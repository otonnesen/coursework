%%%%%%%%%%%%%%%%%%%%%%%%%%%%%%%%%%%%%%%%%
% Memo
% LaTeX Template
% Version 1.0 (30/12/13)
%
% This template has been downloaded from:
% http://www.LaTeXTemplates.com
%
% Original author:
% Rob Oaks (http://www.oak-tree.us) with modifications by:
% Vel (vel@latextemplates.com)
%
% License:
% CC BY-NC-SA 3.0 (http://creativecommons.org/licenses/by-nc-sa/3.0/)
%
%%%%%%%%%%%%%%%%%%%%%%%%%%%%%%%%%%%%%%%%%

\documentclass[letterpaper,11pt]{texMemo}

\usepackage{parskip,booktabs}
\setlength{\parindent}{15pt}
\memoto{Nadia Ariff, Waste Reduction Manager}

\memofrom{Alexander McRae, Oliver Tonnesen}

\memosubject{The elimination of paper cups}

\memodate{Tuesday, Oct 6, 2018}

%\logo{\includegraphics[width=0.3\textwidth]{logo.png}} % Institution logo at the top right of the memo, comment out this line for no logo

\begin{document}

\maketitle

\section*{Objective}
This memo provides a proposal for a study to investigate the feasibility and
effectiveness of eliminating single-use coffee cups and implementing the
use of shared mugs at all UVic food services
locations.
\section*{Background}
Disposable coffee cup waste is a huge problem at UVic. Despite current programs
to increase landfill diversion rate, 32 tonnes of disposable cups sold on campus
still end up in a landfill.\cite{myrefitem} In addition to these non-diverted
paper cups ending up in landfills and not being properly recycled, the
production and transportation of paper cups and their raw materials has been
shown to emit a large amount of CO$_{2}$.\cite{papercupemissions}
% We should probably not mention carbon emissions, since that's not what
% the RFP was about. I only really put it in so the background didn't look
% so empty. Also, if we do keep it we'll need a better reference than the
% one I grabbed.
\section*{Problem Definition}
\subsection*{Need Statement}
Currently, the vast majority of hot beverage purchases at UVic involve a
disposable paper cup. % Cite some UVic resource with some status supporting this
Of UVic's total non-diverted recyclable waste, 6.1\% consists of paper
cups.% More stats!
\subsection*{Goals Statement}
Our goal is to transition to near 100\% sustainable/reusable liquid container
use at UVic.
\subsection*{Objectives}
By implementing a rent-a-mug program similar to those already in place at
other academic institutions,% r e f e r e n c e s
we can immensely reduce the number of paper cups sold on campus, and in
turn reduce the number of paper cups sent to landfills.
The proposed way this would works is at the 
beginning of the year a flat charge is added to everyone's account that purchases 
a mug (~\$5), this mug is not owned however but gives them a credit for a mug to 
use. This credit can be used when getting coffee at any location at UVic and then 
when the mug is returned to a designated location the credit is given back and 
the person can then go to another location on UVic to get coffee again.\\

For example, in the morning I get a cup of coffee at mystic market and bring it to 
class. After class I stop by the Biblio Cafe and drop off the cup to get my credit back 
and then go study in the library. Later that day I get another different cup of coffee using my tab 
at the engineering building then drop it off at mystic before heading off campus.
The cup then stays on campus and gets washed ready for someone the next day.\\

The key thing to note is this is not the same cup each time but a random cup.

\subsection*{Constraints}
Our proposal has a fairly large up-front cost; however, we suspect the initial
price to purchase the reusable mugs will be heavily outweighed by the savings
brought by reducing the amount of paper cups that UVic Food Services has to
purchase continuously. Mugs will undoubtedly become lost or broken, and will so
need to be periodically replaced. We expect this cost to also be significantly
lower than purchasing paper cups in the long run. Industrial dish washers may need 
to be purchased to accommodate the large number of mugs being returned throughout 
the day. Customers of UVic Food Services may be upset that they will have to change 
their routines due to doing away with paper cups.

\section*{Plan of Action}

The plan of action is as follows. First analyse the costs
of the mugs and the opportunity cost associated with staying with paper cups.
Next look at the environment hazards associated with mugs and the possible negative outcomes that could occur due to using the mugs.
Next look at the incentives from all stakeholders perspectives. Then do a trial
in residences gathering data as well as opinions both before and after the trial
has taken place. Based on the trial and residences it is then decided if it is 
suitable for the entire campus.

The research plan provided below gives a brief overview of some of the questions
we aim to investigate in regards to the implementation of reusable mugs across
campus.\\\\

\begin{itemize}
	\item Who uses the most paper cups?
	\item What areas of campus are responsible for the largest amount of
		non-diverted waste?\\

	\item When do people most often drink coffee?
	\item How many mugs will be required during peak hours?
	\begin{itemize}
		\item For how long do people use mugs? Do they bring them off campus?
	\end{itemize}
	\item Are they willing to cooperate with UVic's program, or will they
		get their coffee elsewhere?\\

	\item Is it worth implementing this program all over campus, or would it
		make more sense to initially only target residences?
	\item Is this program any better than those already implemented at UVic?
	\item Investigate other institutions that have implemented similar
		programs.\\

	\item Economic considerations:
	\begin{itemize}
		\item Cost/benefit analysis
		\item Opportunity cost
		\item Short term/long term effects
	\end{itemize}
	\item Environmental considerations:
	\begin{itemize}
		\item How much waste can be eliminated by this program?
		\item How does the manufacturing/transport of the mugs affect the
			environment? Is it better or worse than that of paper cups?
	\end{itemize}

\end{itemize}
\subsection*{Timeline}
Table 1 below shows an overview of the two month timeline we propose to
complete the study and deliver a feasibility report.
\begin{center}
	Table 1: Timeline for project completion
	\begin{tabular}{ccc}
		\toprule
		Description && Dates \\
		Phase One & Gather Data & November 3 --- November 12, 2018 \\
		Phase Two & Analyze Data & November 12 --- November 25, 2018 \\
		Phase Three & Investigate Logistics of Mugs & November 25 --- December 18, 2018 \\
		\bottomrule
	\end{tabular}
\end{center}
\subsection*{Budget}
Table 2 below shows an overview of the costs of putting together this report.
The final cost is projected to be \$2000.
\begin{center}
	Table 2: Budget projection for project completion\\
	\begin{tabular}{ccc}
		\toprule
		Description && Hours \\
		Phase One & Gather Data & 15 \\
		Phase Two & Analyze Data & 10 \\
		Phase Three & Investigate Logistics & 25 \\
		Cost & \$40/hour & \$2000 \\
		\bottomrule
	\end{tabular}
\end{center}

\subsection*{Qualifications}
	The list below outlines how and why we are qualified to undertake this study:
	\begin{itemize}
		\item We are entering our third year studying Computer Science at UVic.
		\item We have proven research skills from our various technical writing
			and research courses here at UVic.
		\item We are able to identify with the UVic student populace and
			understand their sentiments towards various different topics.
	\end{itemize}
\section*{Conclusion}
Disposable coffee cup waste is one of the largest areas of potential
improvement with respect to waste diversion.  Implementing a rent-a-mug
program has the potential to help UVic to reach its goal of 75\% waste diversion
by 2021.
\newline
\newline
Our proposed study would determine the feasibility of implementing such a
program. It would investigate the benefits, drawbacks, and the overall cost,
both in the sort- and in the long-term. Undertaking this study would not cost
much to UVic, but could potentially lead to greatly reduced waste production
all over its campus.

\begin{thebibliography}{9}
\bibitem{myrefitem}
	Nadia Ariff
	\textit{Waste Reduction Problem}.
	available at: https://coursespaces.uvic.ca/
	[Accessed 26 Oct. 2018].
	Waste Management.
\bibitem{myrefite}
	\textit{Waste to Resource Assessment \-- Summary}.
	available at: https://www.uvic.ca/sustainability/assets
		/docs/reports/waste-audit-2014.pdf
	[Accessed 27 Oct, 2018].
\bibitem{papercupemissions}
	EPS Distribution. (2018)
	\textit{The Environmental Effect Of Paper Coffee Cups \-- EPS Distribution.}
	available at:
	http://www.epsdistribution.com.au/the-environmental-effect-of-paper-coffee-cups/
	[Accessed 26 Oct. 2018].

\end{thebibliography}

\end{document}
