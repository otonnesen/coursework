\documentclass[11pt]{article}
\usepackage{listings}
\lstset{language=Matlab,
breaklines=true,
keywordstyle=\color{blue},
identifierstyle=\color{black},
stringstyle=\color{mylilas},
commentstyle=\color{mygreen},
showstringspaces=false,
numbers=left,
numberstyle={\small \color{black}},
numbersep=9pt,
emph=[1]{for,end,break},
emphstyle=[1]\color{red}
}

\usepackage{color}
\definecolor{mygreen}{RGB}{28,172,0}
\definecolor{mylilas}{RGB}{170,55,241}

\usepackage{fancyhdr}
\pagestyle{fancy}
\newcommand\course{CSC 349A}
\newcommand\hwnumber{1}
\newcommand\duedate{September 17, 2019}

\lhead{Oliver Tonnesen\\V00885732}
\chead{\textbf{\Large Assignment \hwnumber}}
\rhead{\course\\\duedate}


\begin{document}
\renewcommand{\thesubsection}{\thesection.\alph{subsection}}
\section{} % Section 1
\subsection{} % Section 1.a
\lstinputlisting{mfiles/Euler.m}


\subsection{} % Section 1.b
\lstinputlisting[numbers=none]{mfiles/Q1b}


\subsection{} % Section 1.c
\lstinputlisting[numbers=none]{mfiles/Q1c}


\subsection{} % Section 1.d
\lstinputlisting[numbers=none]{mfiles/Q1d}


\section{} % Section 2
\subsection{} % Section 2.a
\lstinputlisting{mfiles/Euler2.m}


\subsection{} % Section 2.b
\lstinputlisting[numbers=none]{mfiles/Q2b}


\subsection{} % Section 2.c
\lstinputlisting[numbers=none]{mfiles/Q2c}


\section{} % Section 3
Below is a function to approximate $e^{-x}$ directly from its MacLaurin series
expansion, and a sample output using it to approximate $e^{-3}$..
\lstinputlisting{mfiles/exp_inv_1.m}
\lstinputlisting[numbers=none]{mfiles/Q31}
Below is a function to approximate $e^{-x}$ by taking the inverse of the MacLaurin
series expansion of $e^x$, and a sample output using it to approximate $e^{-3}$..
\lstinputlisting{mfiles/exp_inv_2.m}
\lstinputlisting[numbers=none]{mfiles/Q32}

We can see when using the first method that the relative error did not appear
to begin converging to any value as we increased $n$, while the second method's relative error
appears to be getting smaller as $n$ increases. Note that the first method
always gives a polynomial, so its limits as $x\to\pm\infty$ are $\infty$ or
$-\infty$ (except when $n=0$, of course), but the second method gives the
inverse of a polynomial, so its limits as $x\to\pm\infty$ are 0. Overall,
the series of functions given by the second method look much more similar to
one another than those given by the first, perhaps leading to a more stable
approximation.

\end{document}
