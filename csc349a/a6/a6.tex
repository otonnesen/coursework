\documentclass[11pt]{article}
\usepackage{listings}
\lstset{language=Matlab,
breaklines=true,
keywordstyle=\color{blue},
identifierstyle=\color{black},
stringstyle=\color{mylilas},
commentstyle=\color{mygreen},
showstringspaces=false,
numbers=left,
numberstyle={\small \color{black}},
numbersep=9pt,
emph=[1]{for,end,break},
emphstyle=[1]\color{red}
}

\usepackage{color}
\definecolor{mygreen}{RGB}{28,172,0}
\definecolor{mylilas}{RGB}{170,55,241}

\usepackage{booktabs}

\usepackage{fancyhdr}
\pagestyle{fancy}
\newcommand\course{CSC 349A}
\newcommand\hwnumber{6}
\newcommand\duedate{November 26, 2019}

\lhead{Oliver Tonnesen\\V00885732}
\chead{\textbf{\Large Assignment \hwnumber}}
\rhead{\course\\\duedate}

\usepackage{graphicx}

\usepackage{algpseudocode}

\usepackage{mathtools}
\usepackage{amsmath}
\DeclareMathOperator{\fl}{fl}
\newenvironment{amatrix}[1]{%
  \left(\begin{array}{@{}*{#1}{c}|c@{}}
}{%
  \end{array}\right)
}


\begin{document}
\renewcommand{\thesubsection}{\thesection.\alph{subsection}}
\section{} % Section 1
\subsection{} % Section 1.a
\begin{align*}
	P(x)&=f(0)\frac{(x-h)(x-2h)}{0-h)(0-2h)}+f(h)\frac{(x-0)(x-2h)}{(h-0)(h-2h)}+f(2h)\frac{(x-0)(x-h)}{(2h-0)(2h-h)}\\
	&=f(0)\frac{x^2-3hx+2h^2}{2h^2}+f(h)\frac{x^2-2hx}{-h^2}+f(2h)\frac{x^2-hx}{2h^2}
\end{align*}


\subsection{} % Section 1.b
\begin{align*}
	\int_0^{2h}f(x)dx&\approx\int_0^{2h}P(x)dx\\
	&=\frac{f(x)}{2h^2}\left(\frac{x^3}{3}-\frac{3hx^2}{2}+2h^2x\right)
	+\frac{f(h)}{-h^2}\left(\frac{x^3}{3}-\frac{2hx^2}{2}\right)
	+\frac{f(2h)}{2h^2}\left(\frac{x^3}{3}-\frac{hx^2}{2}\right)\Bigg\rvert_0^{2h}\\
	&=\frac{f(x)}{2h^2}\left(\frac{(2h)^3}{3}-\frac{3h(2h)^2}{2}+2h^2(2h)\right)
	-\frac{f(h)}{h^2}\left(\frac{(2h)^3}{3}-\frac{2h(2h)^2}{2}\right)\\
	&\qquad\qquad+\frac{f(2h)}{2h^2}\left(\frac{(2h)^3}{3}-\frac{h(2h)^2}{2}\right)\\
	&=\frac{f(x)}{2h^2}\left(\frac{8h^3}{3}-\frac{12h^3}{2}+4h^3\right)
	-\frac{f(h)}{h^2}\left(\frac{8h^3}{3}-\frac{8h^3}{2}\right)
	+\frac{f(2h)}{2h^2}\left(\frac{8h^3}{3}-h^3\right)\\
	&=f(0)\left(\frac{4h}{3}-3h+2h\right)-f(h)\left(\frac{8h}{3}-\frac{8h}{2}\right)+f(2h)\left(\frac{4h}{3}-h\right)\\
	&=\frac{hf(0)}{3}+\frac{4hf(h)}{3}+\frac{hf(2h}{3}\\
	&=\frac{h}{3}\left(f(0)+4f(h)+f(2h)\right)
\end{align*}


\subsection{} % Section 1.c
\begin{align*}
	\int_0^{0.2}f(x)dx&\approx\int_0^{2h}P(x)dx\qquad\text{ with $h=0.1$}\\
	&=\frac{h}{3}\left(f(0)+4f(h)+f(2h)\right)\\
	&=\frac{0.1}{3}\left(0.5+4\cdot0.50125+0.50503\right)\\
	&\approx0.1003343
\end{align*}


\section{} % Section 2
\subsection{} % Section 2.a
\begin{tabular}{c|c|c}
	$f(x)$ & $\int_{-1}^{1}f(x)dx$ & $\frac{6}{7}f\left(-\sqrt\frac{2}{5}\right)+\frac{2}{7}f(0)+\frac{6}{7}f\left(\sqrt\frac{2}{5}\right)$ \\
	\hline
	$1$ & $2$ & $2$ \\
	$x$ & $0$ & $0$ \\
	$x^2$ & $\frac{2}{3}$ & $\frac{24}{35}$ \\
\end{tabular}
So the degree of precision of the quadrature formula is 1.


\subsection{} % Section 2.b
If $f(x)=e^{-x}\sqrt{x+1}$, then
$f\left(-\sqrt\frac{2}{5}\right)=1.141108375$, $f(0)=1$, and
$f\left(\sqrt\frac{2}{5}\right)\approx0.6788107828$, so the quadrature formula
gives
\[\frac{6}{7}f\left(-\sqrt\frac{2}{5}\right)+\frac{2}{7}f(0)+\frac{6}{7}f\left(\sqrt\frac{2}{5}\right)\approx1.845644992\]


\section{} % Section 3
\subsection{} % Section 3.a
\lstinputlisting{mfiles/trap.m}

\subsection{} % Section 3.b
\lstinputlisting{mfiles/Q3b}


\end{document}
