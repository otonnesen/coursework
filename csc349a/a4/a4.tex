\documentclass[11pt]{article}
\usepackage{listings}
\lstset{language=Matlab,
breaklines=true,
keywordstyle=\color{blue},
identifierstyle=\color{black},
stringstyle=\color{mylilas},
commentstyle=\color{mygreen},
showstringspaces=false,
numbers=left,
numberstyle={\small \color{black}},
numbersep=9pt,
emph=[1]{for,end,break},
emphstyle=[1]\color{red}
}

\usepackage{color}
\definecolor{mygreen}{RGB}{28,172,0}
\definecolor{mylilas}{RGB}{170,55,241}

\usepackage{fancyhdr}
\pagestyle{fancy}
\newcommand\course{CSC 349A}
\newcommand\hwnumber{4}
\newcommand\duedate{November 5, 2019}

\lhead{Oliver Tonnesen\\V00885732}
\chead{\textbf{\Large Assignment \hwnumber}}
\rhead{\course\\\duedate}

\usepackage{graphicx}

\usepackage{algpseudocode}

\usepackage{mathtools}
\usepackage{amsmath}
\DeclareMathOperator{\fl}{fl}
\newenvironment{amatrix}[1]{%
  \left(\begin{array}{@{}*{#1}{c}|c@{}}
}{%
  \end{array}\right)
}


\begin{document}
\renewcommand{\thesubsection}{\thesection.\alph{subsection}}
\section{} % Section 1
\subsection{} % Section 1.a
The fourth column of $A^{-1}$ is $x$ such that
$Ax=\left(\begin{smallmatrix}0\\0\\0\\1\end{smallmatrix}\right)$, so we solve:\\
$\begin{amatrix}{4}
	0 & -1 & -2 & -4 & 0\\
	-2 & -1 & 1 & 0 & 0\\
	-1 & 2 & -2 & 0 & 0\\
	1 & 3 & -1 & 0.5 & 1
\end{amatrix}$
\newline
\newline
Partial pivoting swap rows 1 and 2:
$\begin{amatrix}{4}
	-2 & -1 & 1 & 0 & 0\\
	0 & -1 & -2 & -4 & 0\\
	-1 & 2 & -2 & 0 & 0\\
	1 & 3 & -1 & 0.5 & 1
\end{amatrix}$
\newline
\newline
Forward elminiation:
$\begin{amatrix}{4}
	-2 & -1 & 1 & 0 & 0\\
	0 & -1 & -2 & -4 & 0\\
	0 & 2.5 & -2.5 & 0 & 0\\
	0 & 2.5 & -0.5 & 0.5 & 1
\end{amatrix}$
\newline
\newline
Partial pivoting: swap rows 2 and 3:
$\begin{amatrix}{4}
	-2 & -1 & 1 & 0 & 0\\
	0 & 2.5 & -2.5 & 0 & 0\\
	0 & -1 & -2 & -4 & 0\\
	0 & 2.5 & -0.5 & 0.5 & 1
\end{amatrix}$
\newline
\newline
Forward elimination:
$\begin{amatrix}{4}
	-2 & -1 & 1 & 0 & 0\\
	0 & 2.5 & -2.5 & 0 & 0\\
	0 & 0 & -3 & -4 & 0\\
	0 & 0 & 2 & 0.5 & 1
\end{amatrix}$
\newline
\newline
Partial pivoting: no swap. Forward elimination:
$\begin{amatrix}{4}
	-2 & -1 & 1 & 0 & 0\\
	0 & 2.5 & -2.5 & 0 & 0\\
	0 & 0 & -3 & -4 & 0\\
	0 & 0 & 0 & -\frac{13}{6} & 1
\end{amatrix}$
\newline
\newline
Back substitution:
\newline
\newline
\begin{align*}
	x_4&=\frac{b_4}{a_{44}}=\frac{1}{-\frac{13}{6}}&=-\frac{6}{13}\\
	x_3&=\frac{b_3-a_{34}x_4}{a_{33}}=\frac{0-(-4)(-\frac{6}{13})}{-3}&=\frac{8}{13}\\
	x_2&=\frac{b_2-a_{23}x_3-a_{24}x_4}{a_{22}}=\frac{0-(-2.5)(\frac{8}{13})-(0)(-\frac{6}{13})}{2.5}&=\frac{8}{21}\\
	x_1&=\frac{b_1-a_{12}x_2-a_{13}x_3-a_{14}x_4}{a_{33}}=\frac{0-(-1)(\frac{8}{21})-(1)(\frac{8}{13})-(0)(-\frac{6}{13})}{-2}&=0
\end{align*}
So we get $x=\left(\begin{smallmatrix}0\\\frac{8}{13}\\\frac{8}{13}\\-\frac{6}{13}\end{smallmatrix}\right)$.


\subsection{} % Section 1.b
We performed an even number of row swaps, so we need not flip the sing of our
number at the end, so we have
\[\det A=(-2)(2.5)(-3)(-\frac{13}{6})=-32.5,\]
the product of the diagonal entries in the upper triangular matrix we obtained
from the Gaussian elimination.  

\section{} % Section 2
\subsection{} % Section 2.a
\begin{algorithmic}[1]
	\Function{SLEDiag}{A, b}
	\State $n \gets \text{rank}\;A$
	\State $x_1 \gets \frac{b_1}{a_{11}}$
	\State $x_2 \gets \frac{b_2-a_{21}x_1}{a_{22}}$
	\For{$i \gets 3,\ldots, n$}
	\State $x_i \gets \frac{b_i-a_{i,i-2}x_{i-2}-a_{i,i-1}x_{i-1}}{a_{ii}}$
	\EndFor
	\State \Return $(x_1,\ldots,x_n)$
	\EndFunction
\end{algorithmic}


\subsection{} % Section 2.b
Computing $x_1$ takes one division, for a total of one flop, and computing
$x_2$ takes one multiplication, one subtraction, and one division, for a total
of three flops. For each $i$ from 3 to $n$, computing $x_i$ takes two
multiplications, two subtractions, and one division, for a total of five
flops. Thus the total flop count of our algorithm is
\[1+3+5(n-2)=5n-6\in\Theta(n),\]
so our algorithm solves such systems in linear time.


\subsection{} % Section 2.c
\lstinputlisting{mfiles/SLEDiag.m}
The MATLAB statements used to call SLEDiag, along with its output:
\lstinputlisting[numbers=none]{mfiles/q2c}


\section{} % Section 3
\subsection{} % Section 3.a
We solve:
$\begin{amatrix}{2}
	3.102 & -0.155 & -2.012\\
	-2.534 & 0.1234 & 1.007\\
\end{amatrix}$
\newline
\newline
Our factor is
$\fl\left(\frac{a_{21}}{a_{11}}\right)=\fl\left(\frac{-2.534}{3.102}\right)
=-0.8169$. After forward elimination, we have
$\begin{amatrix}{2}
	3.102 & -0.155 & -2.012\\
	0 & 0.2500 & 1.644\\
\end{amatrix}$
\newline
\newline
Now, performing back substitution, we get
$x_2=\fl\left(\frac{b_2}{a_{22}}\right)=\fl\left(\frac{1.644}{0.2500}\right)=6.576$
\begin{align*}
	x_1&=\fl\left(\frac{\fl\left(b_1-\fl\left(a_{12}x_2\right)\right)}{a_{11}}\right)\\
	&=\fl\left(\frac{\fl\left(-2.012-\fl\left((-0.155)(6.576)\right)\right)}{3.102}\right)\\
	&=\fl\left(\frac{\fl\left(-2.012-\left(-1.102\right)\right)}{3.102}\right)\\
	&=\fl\left(\frac{-0.9100}{3.102}\right)\\
	&=-0.2934\\
\end{align*}
So we get $a_{22}=0.2500$, $b_2=1.644$, $x_1=-0.2934$, and $x_2=6.576$.


\subsection{} % Section 3.a
\lstinputlisting[numbers=none]{mfiles/q3b}


\end{document}
