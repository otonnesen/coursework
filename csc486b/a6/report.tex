\documentclass[11pt]{article}
\usepackage{listings}
\lstset{
}

\usepackage{fancyhdr}
\pagestyle{fancy}
\newcommand\course{CSC 486B}
\newcommand\hwnumber{6}
\newcommand\duedate{March 29, 2020}

\lhead{Oliver Tonnesen\\V00885732}
\chead{\textbf{\Large Assignment \hwnumber}}
\rhead{\course\\\duedate}

\usepackage{graphicx}
\usepackage{float}

\usepackage{mathtools}
\usepackage{amsmath}


\begin{document}
\section*{Hyperparameters}
Bob set the learning rate too high.
The bug can be fixed by adjusting the learning rate to a lower value, as below:
\begin{lstlisting}
python solution.py --learning-rate=0.01 --mode=train --rep_intv=1
	--conv2d=torch
\end{lstlisting}


\section*{model.py/MyNetwork/forward}
Bob incorrectly normalized his data.
On line 211 in model.py, Bob wrote
\begin{lstlisting}
x = x / 127.5 - 127.5.
\end{lstlisting}
This results in x being between $-127.5$ and $-125.5$, instead of $-1.0$ and $1.0$.
Instead, he should have written
\begin{lstlisting}
x = (x - 127.5) / 127.5,
\end{lstlisting}
resulting in x being in the desired range of $-1.0$ and $1.0$.


\section*{solution.py/test}
Bob forgot to set the model for evaluation before testing it.
After line 301 in solution.py, Bob should have written
\begin{lstlisting}
model.eval()
\end{lstlisting}
to prevent the model from training to attain the desired $\sim67\%$ accuracy.


\end{document}
