\documentclass{article}
\usepackage{amsmath,amsfonts,tikz,pgfplots}
\pgfplotsset{compat=1.16}
\usepgfplotslibrary{fillbetween}
\title{MATH 200 Assignment Three}
\author{%
	Oliver Tonnesen\\
	V00885732\\
	A02 \-- T03}
\date{November 7, 2018}
\begin{document}
\maketitle
\renewcommand{\thesubsection}{\thesection.\alph{subsection}}
\section{} % Question 1
We define the volume, length, width, and height as follows:
\begin{align*}
	l&=l(t)\\
	w&=w(t)\\
	h&=h(t)\\
\end{align*}
We are given the following:
\begin{align*}
	l(t_0)&=1\\
	w(t_0)&=2\\
	h(t_0)&=2\\
	l'(t_0)&=2\\
	w'(t_0)&=2\\
	h'(t_0)&=-3\\
\end{align*}
\subsection{} % Section 1.a
\begin{align*}
	V&=lwh\\\\
	\frac{\partial{V}}{\partial{t}}
	&=\frac{\partial{V}}{\partial{l}}\cdot\frac{dl}{dt}
	+\frac{\partial{V}}{\partial{w}}\cdot\frac{dw}{dt}
	+\frac{\partial{V}}{\partial{h}}\cdot\frac{dh}{dt}\\
	&=2wh+2lh-3lw\\
	&=2(2)(2)+2(1)(2)-3(1)(2)\\
	&=6m^3/s
\end{align*}
\subsection{} % Section 1.b
\begin{align*}
	A&=2lw+2lh+2wh\\\\
	\frac{\partial{A}}{\partial{t}}
	&=\frac{\partial{A}}{\partial{l}}\cdot\frac{dl}{dt}
	+\frac{\partial{A}}{\partial{w}}\cdot\frac{dw}{dt}
	+\frac{\partial{A}}{\partial{h}}\cdot\frac{dh}{dt}\\
	&=(2w+2h)\cdot2+(2l+2h)\cdot2+(2l+2w)\cdot(-3)\\
	&=(4+4)(2)+(2+4)(2)+(2+4)(-3)\\
	&=10m^2/s
\end{align*}
\subsection{} % Section 1.c
We will use the diagonal formed by the length ($l$) and width ($w$) of the box.
\begin{align*}
	D&=\sqrt{l^2+w^2}\\\\
	\frac{\partial{D}}{\partial{t}}
	&=\frac{\partial{D}}{\partial{l}}\cdot\frac{dl}{dt}
	+\frac{\partial{D}}{\partial{w}}\cdot\frac{dw}{dt}
	+\frac{\partial{D}}{\partial{h}}\cdot\frac{dh}{dt}\\
	&=\frac{l}{\sqrt{l^2+w^2}}\cdot2+\frac{w}{\sqrt{l^2+w^2}}\cdot2\\
	&=\frac{1}{\sqrt{5}}\cdot2+\frac{2}{\sqrt{5}}\cdot2\\
	&=\frac{2}{\sqrt{5}}\big(1+2\big)\\
	&=\frac{6}{\sqrt{5}}
\end{align*}
\section{} % Question 2
\begin{align*}
	% \frac{\partial{}}{\partial{}}
	\frac{\partial{z}}{\partial{x}}
	&=\frac{\partial{z}}{\partial{f}}\cdot\frac{df}{dx}
	+\frac{\partial{z}}{\partial{g}}\cdot\frac{dg}{dx}\\
	&=\frac{\partial{z}}{\partial{f}}
	+\frac{\partial{z}}{\partial{g}}\\\\
	\frac{\partial^2{z}}{\partial{x^2}}
	&=\frac{\partial^2{z}}{\partial{f^2}}\cdot\frac{df}{dx}
	+\frac{\partial^2{z}}{\partial{g^2}}\cdot\frac{dg}{dx}\\
	&=\frac{\partial^2{z}}{\partial{f^2}}
	+\frac{\partial^2{z}}{\partial{g^2}}
\end{align*}
\begin{align*}
	\frac{\partial{z}}{\partial{t}}
	&=\frac{\partial{z}}{\partial{f}}\cdot\frac{df}{dt}
	+\frac{\partial{z}}{\partial{g}}\cdot\frac{dg}{dt}\\
	&=a\frac{\partial{z}}{\partial{f}}
	-a\frac{\partial{z}}{\partial{g}}\\\\
	\frac{\partial^2{z}}{\partial{t^2}}
	&=a\frac{\partial^2{z}}{\partial{f^2}}\cdot\frac{df}{dt}
	-a\frac{\partial^2{z}}{\partial{g^2}}\cdot\frac{dg}{dt}\\
	&=a^2\frac{\partial^2{z}}{\partial{f^2}}
	+a^2\frac{\partial^2{z}}{\partial{g^2}}\\
	&=a^2\bigg(\frac{\partial^2{z}}{\partial{f^2}}
	+\frac{\partial^2{z}}{\partial{g^2}}\bigg)\\
	&=a^2\bigg(\frac{\partial^2{z}}{\partial{x^2}}\bigg)\\\\
	\frac{\partial^2{z}}{\partial{t^2}}
	&=a^2\frac{\partial^2{z}}{\partial{x^2}}
\end{align*}
\section{} % Question 3
\begin{align*}
	f(x,y,z)&=3(x-1)^2+2(y+3)^2-z+7\\
	\nabla{f}&=\big\langle6(x-1),4(y+3),-1\big\rangle\\
	\nabla{f}(2,-2,12)&=\big\langle6,4,-1\big\rangle\\\\
	P_0&=\big(3,-2,18\big)\text{\;is a point on the tangent plane}\\
	xf_x+yf_y+zf_z&=3f_x-2f_y+18f_z\\
	6x+4y-z&=-8
\end{align*}
\section{} % Question 4
\begin{align*}
	D_{\vec{PQ}}f(2,8)&=\nabla{f}\cdot\vec{PQ}\\
	\nabla{f}&=\bigg\langle\frac{y}{2\sqrt{xy}},\frac{x}{2\sqrt{xy}}\bigg\rangle\\
	\vec{PQ}&=\big\langle3,4\big\rangle\\
	\hat{PQ}&=\bigg\langle\frac{3}{5},\frac{4}{5}\bigg\rangle\\\\
	\nabla{f}\cdot\hat{PQ}&=\frac{y}{2\sqrt{xy}}\cdot\frac{3}{5}+\frac{x}{2\sqrt{xy}}\cdot\frac{4}{5}\\
	\nabla{f}(2,8)\cdot\hat{PQ}&=\frac{2}{5}\\
\end{align*}
\section{} % Question 5
Recall that $\nabla{f}$ goes in the direction of maximum change.
\begin{align*}
	\nabla{f}&=\big\langle y\cos{xy},x\cos{xy} \big\rangle\\
	\nabla{f}(1,0)&=\big\langle0,1\big\rangle\\
	|\nabla{f}(1,0)|=1
\end{align*}
So the maximum rate of change is 1 and occurs in the direction of $\big\langle0,1\big\rangle$.
\section{} % Question 6
\begin{align*}
	D_{\vec{u}}f(0,2)&=1\\
	\nabla{f}(0,2)\cdot\vec{u}&=1\\
	\nabla{f}&=\big\langle-y^2e^{-xy},e^{-xy},-xye^{-xy}\big\rangle\\
	\nabla{f}(0,2)&=\big\langle-4,1\big\rangle
\end{align*}
So we have the following system of equations:
\begin{align*}
	-4u_1+u_2=1\\
	u_1^2+u_2^2=1
\end{align*}
and must simply solve for $u_1$ and $u_2$.
\begin{align*}
	u_2=4u_1+1\\
	u_1^2+(4u_1+1)^2=1\\
	17u_1^2+8u_1=0\\
	u_1=\frac{-8\pm\sqrt{64}}{34}\\
	u_1=0,-\frac{8}{17}
\end{align*}
So thus
\[\big\langle0,1\big\rangle\]
and
\[\big\langle-\frac{8}{17},-\frac{15}{17}\big\rangle\]
are the two directions in which the directional derivative is 1 at (0,2).
\section{} % Question 7
\begin{align*}
	f_x=e^x\cos{y}\\
	f_y=-e^x\sin{y}
\end{align*}
Critical points are wherever the partial derivatives are 0 or undefined, so
\begin{align*}
	f_x=0:\bigg(k,\frac{\pi}{2}+k\pi\bigg),k\in\mathbb{Z}\\
	f_y=0:\bigg(k,k\pi\bigg),k\in\mathbb{Z}\\
\end{align*}
\section{} % Question 8
\begin{align*}
	f_x=2x-4y\\
	f_y=8x-4y\\
	0=2x-4y\text{\;and\;}0=8y-4x
\end{align*}
Both of these equations are lines, and thus each has infinitely many solutions.
\begin{align*}
	f_{xx}=2\\
	f_{yy}=8\\
	f_{xy}=-4\\
	D=16-16=0\text{\;for all $x$ and $y$}\\
\end{align*}
\end{document}
