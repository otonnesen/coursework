\documentclass[11pt]{article}
\usepackage{fancyhdr}
\pagestyle{fancy}
\newcommand\course{MATH 312}
\newcommand\hwnumber{5}
\newcommand\duedate{December 2, 2019}

\lhead{V00885732}
\chead{\textbf{\Large Assignment \hwnumber}}
\rhead{\course\\\duedate}


\usepackage{amsmath,amssymb,amsfonts,mathtools}


\DeclareMathOperator{\im}{Im}
\DeclarePairedDelimiter\abs{\lvert}{\rvert}%
\makeatletter
\let\oldabs\abs
\def\abs{\@ifstar{\oldabs}{\oldabs*}}


\begin{document}
\renewcommand{\thesubsection}{\thesection.\alph{subsection}}
\section{} % Section 1
\subsection{} % Section 1.a
$\ker\varphi_a=\{x\in R\mid ax=0\}$. $R$ is an integral domain, so since
$a\neq0$, $ax=0$ means that $x=0$, so $\ker\varphi_a=\{0\}$, thus $\varphi_a$
is injective, and since $R$ is finite, this means that $\varphi_a$ is also
surjective, so $\varphi_a$ is bijective.

Let $x,y\in R$. $\varphi_a(x+y)=a(x+y)=ax+ay=\varphi_a(x)+\varphi_a(y)$, so
$\varphi_a$ is a homomorphism. It is bijective, thus an isomorphism, and
$\im\varphi_a=R$, thus it is an automorphism, as desired.


\subsection{} % Section 1.b
Let $a\in R$, $a\neq0$. Since $\varphi_a$ is a bijection, it has an inverse,
$\varphi_a^{-1}$. Consider $\varphi_a^{-1}$.
$1=\varphi_a(\varphi_a^{-1}(1))=a\varphi_a^{-1}(1)$, so
$\varphi_1^{-1}(1)=a^{-1}$. Thus any $0\neq a\in R$ has an inverse, so $R$ is
a field.


\section{} % Section 2
$(a)\implies(b)$: $y=ux$, so $x\mid y$. $u$ is a unit, so since $y=ux$, $x=u^{-1}y$, hence $y\mid x$.
\newline
\newline
$(b)\implies(c)$: We know that $a\mid b\iff b\in \langle a\rangle$, so
$y\in\langle x\rangle$ and $x\in\langle y\rangle$. $y\in\langle x\rangle$, so
$\langle y\rangle\subseteq\langle x\rangle$, and $x\in\langle y\rangle$, so
$\langle x\rangle\subseteq\langle y\rangle$. Thus
$\langle x\rangle=\langle y\rangle$.
\newline
\newline
$(c)\implies(a)$: $y\in\langle y\rangle$, and
$\langle y\rangle=\langle x\rangle$, so $y\in\langle x\rangle$. Thus $y=ux$
for some $u\in R$. Similarly, $x\in\langle y\rangle$, so $x=vy$ for some
$v\in R$. Then $y=u(vy)=(uv)y$, so $uv=1$, thus $u$ is a unit.


\section{} % Section 3
\subsection{} % Section 3.a
$J$ is an ideal of $R_P$, so for any $\frac{i}{j}\in J$, $\frac{a}{b}\in R_P$,
$\frac{ai}{bj}\in J$. Let $i\in I$, $r\in R$. $1\not\in P$ since $P$ is prime,
so $\frac{r}{1}\in R_P$. We know that there is a $p\in P$ such that
$\frac{i}{p}\in J$, so $\frac{r}{1}\cdot\frac{i}{p}=\frac{ri}{p}\in J$, then
$ri\in I$, so since $i$ and $r$ were arbitrary, $I$ is an ideal of $R$.


\subsection{} % Section 3.b
Let $J$ be an ideal of $R_P$. Let $I$ be the set of numerators of elements of
$J$. We just showed that $I$ is an ideal of $R$. Let $I=\langle a\rangle$.
Then any element of $J$ has the form $\frac{ra}{b}$, $b\not\in P$ for some
$r\in R$. $\frac{ra}{b}=\frac{r}{b}\cdot\frac{a}{1}$, so
$J=\langle\frac{a}{1}\rangle$, thus $R_P$ is a PID.


\section{} % Section 4
\subsection{} % Section 4.a
Let $a_1\in R$ be a nonzero non-unit, and assume for a contradiction that
$a_1$ cannot be written as a product of irreducibles. $a_1$ is not
irreducible, otherwise $a_1=a_1$ would be $a_1$ written as a product of
irreducibles. So $a_1=bc$, for non-units $b,c\in R$. $b$ or $c$ must not be
able to be written as a product of irreducibles (otherwise we could use their
decomposition into irreducibles to write $a_1$ as a product of irreducibles)
so WLOG $b$ cannot be written as a product of irreducibles. Let $a_2=b$.
$a_1\in\langle a_2\rangle$, so $\langle a_1\rangle\subseteq\langle
a_2\rangle$. $\langle a_1\rangle\neq\langle a_2\rangle$, otherwise $a_2=ra_1$,
then $a_1=(ra_1)c=(rc)a_1$, so $rc=1$, but $c$ was not a unit. We can continue
this process to construct an infinite chain $\langle
a_1\rangle\subsetneq\langle a_2\rangle\subsetneq\cdots$, a contradiction, so
$a_1$ can be written as a product of irreducibles, as desired.


\subsection{} % Section 4.b
Let $r=p_1\cdots p_k$. If $d\mid r$, $d$ irreducible, then there is some $q\in
R$ such that $r=qd$, so $qd=p_1\cdots p_k$. $p_1\cdots p_k$ is unique up to
order and units, so we can assume $ud=pk$, $vq=p_1\cdots p_{k-1}$ for some
units $u,v\in R$. So any divisor $d$ of $r$ must be one of $p_1,\ldots,p_k$
multiplied by a unit, thus $r$ has, up to units, $k$ divisors.

So any element $r\in R$ has a finite number of divisors up to units. That is
to say, any element dividing $r$ lies in one of finitely many principal ideals
$\langle p_1\rangle,\ldots,\langle p_k\rangle$.

Now suppose for a contradiction that there exists a sequence
$a_1,a_2,\ldots\in R$ such that $\langle a_1\rangle\subsetneq\langle
a_2\rangle\subsetneq\cdots$. Consider the element $s=a_1\cdot a_2\cdot\ldots$.
$s$ has only finitely many divisors, so somewhere in the sequence $\langle
a_i\rangle$ must become equal to $\langle a_{i+1}\rangle$, a contradiction, so
no such sequence exists, and $R$ thus has the $\heartsuit$ property.


\section{} % Section 5
$R$ is a PID, so $\langle a,b\rangle=\langle d\rangle$ for some $d\in R$.
$a,b\in\langle a,b\rangle$, so $a,b\in\langle d\rangle$, and thus $d\mid a$
and $d\mid b$. We claim that $d$ is a $\gcd$ of $a$ and $b$.

Let $q\in R$ such that $q\mid a$ and $q\mid b$. Then $a=nq$ and $b=mq$ for
some $n,m\in R$. $\langle d\rangle=\langle a,b\rangle$, so $d\in\langle
a,b\rangle$, thus $d=xa+yb$ for some $x,y\in R$. Then
$d=x(nq)+y(mq)=(xn+ym)q$, so $q\mid d$. Thus $d$ is a $\gcd$ of $a$ and $b$.

Now let $c$ be any arbitrary $\gcd$ of $a$ and $b$. We know, since $d\mid a$
and $d\mid b$, that $d\mid c$, so $c\in\langle d\rangle=\langle a,b\rangle$,
as desired.


\section{} % Section 6
\subsection{} % Section 6.a
Let $I=\langle a_1,\ldots,a_k\rangle\subseteq R$ be a finitely generated
ideal. We prove $I$ is principal by induction on $k$:

This is true by definition when $k=1$, and when $k=2$, let $r\in\langle
a,b\rangle$, $d$ a $\gcd$ of $a$ and $b$. $r=xa+yb$ for some $x,y\in R$, and
$d\mid a$ and $d\mid b$, so $a=nd$, $b=md$ for some $n,m\in R$. Thus
$r=x(nd)+y(md)=(xn+ym)d$, so $r\in\langle d\rangle$. Thus $\langle
a,b\rangle\subseteq\langle d\rangle$. Now let $r\in\langle d\rangle$. Then
$r=zd$ for some $z\in R$. $d\in\langle a,b\rangle$, so $d=ia+jb$ for some
$i,j\in R$. So $r=z(ia+jb)=(zi)a+(zj)b\in\langle a,b\rangle$, so
$\langle d\rangle\subseteq\langle a,b\rangle$, and so
$\langle a,b\rangle=\langle d\rangle$, and thus $\langle a,b\rangle$ is
principal.

Assume $\langle a_1,\ldots,a_k$ is principal for any $k\le n$, and consider
$\langle a_1,\ldots,a_{n+1}\rangle$. Let
$r=c_1a_1+\cdots+c_{n+1}a_{n+1}\in\langle a_1,\ldots,a_{n+1}\rangle$. By our
hypothesis, $\langle a_1,\ldots,a_n\rangle$ is principal, say generated by
$d$. Then $r-c_{n+1}a_{n+1}\in\langle d\rangle$, so $r\in\langle
d,a_{n+1}\rangle$. Again by our hypothesis, $\langle d,a_{n+1}\rangle$ is
principal, as desired. So by induction, every finitely generated ideal of $R$
is principal.


\subsection{} % Section 6.b
Assume for a contradiction that $I\subseteq R$ is a non finitely generated
ideal, say $I=\langle a_1,a_2,\ldots\rangle$. Every finitely generated ideal
of $R$ is principal, so $\langle a_1\rangle\subsetneq\langle
a_1,a_2\rangle\subsetneq\cdots$ is a sequence of principal ideals, a
contradiction since $R$ is a UFD, and therefore has the $\heartsuit$ property.
So no such $I$ exists, and thus every ideal is finitely generated, and thus
principal. So $R$ is a PID, as desired.


\end{document}
