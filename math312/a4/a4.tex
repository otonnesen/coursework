\documentclass[11pt]{article}
\usepackage{fancyhdr}
\pagestyle{fancy}
\newcommand\course{MATH 312}
\newcommand\hwnumber{4}
\newcommand\duedate{November 7, 2019}

\lhead{Oliver Tonnesen\\V00885732}
\chead{\textbf{\Large Assignment \hwnumber}}
\rhead{\course\\\duedate}


\usepackage{amsmath,amsfonts,mathtools}


\DeclarePairedDelimiter\abs{\lvert}{\rvert}%
\makeatletter
\let\oldabs\abs
\def\abs{\@ifstar{\oldabs}{\oldabs*}}


\begin{document}


\renewcommand{\thesubsection}{\thesection.\alph{subsection}}


\section{} % Section 1
$H\cap K\le G$, so since $H\cap K\subseteq H$, $H\cap K\le H$. By Lagrange's
theorem, we have that $\abs{H\cap K}\mid\abs{H}$, but $\abs{H}$ is
prime, so either $\abs{H\cap K}=1$ or $p$. In the first case, $H\cap
K=\{1_G\}$, and in the second case $H\cap K=H=K$, as desired.


\section{} % Section 2
$(\Longrightarrow)$: Assume for a contradiction that $\abs{G}$ is not prime. Let
$p\mid\abs{G}$, with $p\neq1$ or $\abs{G}$. Then there exists some
subgroup $P\le G$ with $\abs{P}=p$, but $G$ is abelian, so $P\trianglelefteq
G$, a contradiction.  Thus $\abs{G}$ must be prime.
\newline
\newline
$(\Longleftarrow)$: Let $H\le G$. $\abs{H}\mid\abs{G}$, but $\abs{G}$
is prime, so $\abs{H}=1$ or $\abs{G}$, thus any subgroup (hence any normal
subgroup) of $G$ is trivial, so $G$ is simple.


\section{} % Section 3
\subsection{} % Section 3.a
By the third Sylow theorem, $n_{11}\equiv1\pmod{11}$ and $n_{11}\mid\abs{G}$,
so $n_{11}=1,11,11^2,17,17^2,11\cdot17,11^2\cdot17,11\cdot17^2$, or
$11^2\cdot17^2$. Obviously any number divisible by $11$ is not congruent to
$1\pmod{11}$, so $n_{11}=1,17$, or $17^2$. $17\equiv6\pmod{11}$ and
$17^2\equiv3\pmod{11}$, so it must be the case that $n_{11}=1$. Thus there is
only one Sylow 11-subgroup of $G$, and it follows then from the second Sylow
theorem that this subgroup must be normal. By a similar argument,
$n_{17}=1,11$, or $11^2$, but $11\equiv11\pmod{17}$ and
$11^2\equiv2\pmod{17}$, so $n_{17}=1$, and therefore the unique Sylow
17-subgroup is also normal in $G$.


\subsection{} % Section 3.b
We know that any group with order $p^2$, $p$ a prime is abelian, so both the
Sylow 11-subgroup and the Sylow 17-subgroup are abelian (since they have
orders $11^2$ and $17^2$, respectively). Let $N_{11},N_{17}\trianglelefteq G$
be the Sylow 11- and Sylow 17-subgroups, respectively. We show that
$G=N_{11}\times N_{17}$:
\newline
\newline
\underline{$N_{11},N_{17}\trianglelefteq G$}: Showed in part 3.a).
\newline
\newline
\underline{$N_{11}\cap N_{17}=\{0\}$}: Let $g\in N_{11}\cap N_{17}$.
$\abs{g}\mid\abs{N_{11}}$ and $\abs{g}\mid\abs{N_{17}}$, so $\abs{g}\mid11^2$
and $\abs{g}\mid17^2$, thus $\abs{g}=1$, so $g=0$.
\newline
\newline
\underline{$G=N_{11}N_{17}$}: $N_{11}\cap N_{17}=\{0\}$, so since
$\abs{G}=11^2\cdot17^2=\abs{N_{11}}\abs{N_{17}}$, $G=N_{11}N_{17}$.

Thus $G=N_{11}\times N_{17}$, so $G\cong N_{11}\oplus N_{17}$. $N_{11}$ and
$N_{17}$ are abelian, so their direct product is abelian, and thus $G$ is
abelian, as desired.


\subsection{} % Section 3.c
$N_{11}\cong\mathbb{Z}_{11^2}$, or $\mathbb{Z}_{11}\oplus\mathbb{Z}_{11}$, and
$N_{17}\cong\mathbb{Z}_{17^2}$, or $\mathbb{Z}_{17}\oplus\mathbb{Z}_{17}$, so
since $G\cong N_{11}\oplus N_{17}$, any group of order 34969 is isomorphic to
exactly one of the following:
\begin{align*}
	&\mathbb{Z}_{11^2}\oplus\mathbb{Z}_{17^2}\\
	&\mathbb{Z}_{11^2}\oplus\mathbb{Z}_{17}\oplus\mathbb{Z}_{17}\\
	&\mathbb{Z}_{11}\oplus\mathbb{Z}_{11}\oplus\mathbb{Z}_{17^2}\\
	&\mathbb{Z}_{11}\oplus\mathbb{Z}_{11}\oplus\mathbb{Z}_{17}\oplus\mathbb{Z}_{17}\\
\end{align*}


\section{} % Section 4
$N_1=4$. We know that there are exactly two groups of order 4: $\mathbb{Z}_4$
and $\mathbb{Z}_2\oplus\mathbb{Z}_2$. $Z_4$ is abelian and
$\langle[2]_4\rangle\le\mathbb{Z}_4$, so
$\langle[2]_4\rangle\trianglelefteq\mathbb{Z}_4$.
$\mathbb{Z}_2\oplus\mathbb{Z}_2$ is also abelian, and
$\langle([1]_2,[0]_2\rangle\le\mathbb{Z}_2\oplus\mathbb{Z}_2$, so
$\langle([1]_2,[0]_2\rangle\trianglelefteq\mathbb{Z}_2\oplus\mathbb{Z}_2$.
Thus, neither group of order 4 is simple.


\section{} % Section 5
$N_2=46$. $46=2\cdot23$. Let $G$ be a group of order 46. We know by the first
Sylow theorem that there exists a subgroup of $G$ with order 23. This subgroup
has order half that of $G$, so $H$ is normal, and hence $G$ is not simple.


\section{} % Section 6
$N_3=30$. $30=2\cdot3\cdot5$. By the third Sylow theorem,
$n_5\equiv1\pmod{5}$, and $n_5\mid\abs{G}$, so $n_5=1$ or $6$. Similarly,
$n_3\equiv1\pmod{3}$, and $n_3\mid\abs{G}$, so $n_3=1$ or $10$. If $n_5=1$, or
$n_3=1$, then $G$ has a normal Sylow 5- or Sylow 3-subgroup, and hence is not
simple. So assume $n_5=6$ and $n_3=10$. 5 and 3 are prime, so all of these
subgroups are cyclic and every nonidentity element generates the entire
subgroup. This means that any pair of these 16 subgroups intersects only at
the identity. So $G$ is the union of 6 5-subgroups and 10 3-subgroups, and this
union is disjoint when we remove the identity from each subset. So we have
\[\abs{G}=1+6(5-1)+10(3-1)=45>30=\abs{G},\]
a contradiction, and so our assumption that $n_5=6$ and $n_3=10$ was false, so
$n_5=1$ or $n_3=1$, meaning $G$ has a normal subgroup, and is therefore not
simple.


\end{document}
