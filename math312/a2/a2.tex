\documentclass[11pt]{article}
\usepackage{fancyhdr}
\pagestyle{fancy}
\newcommand\course{MATH 312}
\newcommand\hwnumber{2}
\newcommand\duedate{October 3, 2019}

\lhead{Oliver Tonnesen\\V00885732}
\chead{\textbf{\Large Assignment \hwnumber}}
\rhead{\course\\\duedate}


\usepackage{amsmath,amsfonts}


\DeclareMathOperator{\im}{Im}
\DeclareMathOperator{\Inn}{Inn}
\DeclareMathOperator{\Aut}{Aut}


\begin{document}


\renewcommand{\thesubsection}{\thesection.\alph{subsection}}


\section{} % Section 1
$G=\mathbb{Z}_1$, $A=\emptyset$. The empty set is indeed a subset of
$\mathbb{Z}_1$, and the statement ``for all $g$ in $G$ and $a$ in $A$,
$gag^{-1}$ is in $A$'' is vacuously true. $A$ is not a normal subgroup of $G$
since it isn't a group, as it's not nonempty.


\section{} % Section 2
Define $\varphi:G\rightarrow G$ by $\varphi(g)=2g$. Let $g,h\in G$.
$\varphi(g+h)=2(g+h)=2g+2h=\varphi(g)+\varphi(h)$, so $\varphi$ is a
homomorphism.
\begin{align*}
	\ker\varphi&=\varphi^{-1}(\{0\})\\
	&=\{g\in G\mid2g=0\}\\
	&=\{g\in G\mid|g|\Big\vert2\}
\end{align*}
$G$ has odd order, and the order of any element of $G$ must divide its order,
so there are no elements of order 2, thus $\ker\varphi=\{0\}$. Then by the
first isomorphism theorem, we have that $G\mathbin/\ker\varphi\cong\im\varphi$,
but $G\mathbin/\{0\}\cong G$, so $G\cong\im\varphi$, thus $\varphi$ is an
isomorphism.
\newline
\newline
Most importantly, since $\varphi$ is an isomorphism, it is injective, so for
any $y\in G$, $\varphi$ \textit{uniquely} maps it to the elementy $2y$ (or $x$),
as desired.


\section{} % Section 3
\subsection{} % Section 3.a
We proved in class that this map $\varphi$ is a homomorphism. We know that for
any $g\in G$, $|\varphi(g)|\mathbin\Big\vert|g|$. We also know that for any
$g\in G$, $|g|\mathbin\Big\vert|G|$, so since $\varphi(g)\in H$,
$|\varphi(g)|\mathbin\Big\vert|H|$. Since $\gcd(|G|,|H|)=1$, $|\varphi(g)|$ and
$|g|$ share no common factors (other than 1), but
$|\varphi(g)|\mathbin\Big\vert|g|$, so $|\varphi(g)|$ must be 1, thus
$\varphi(g)$ must be $1_H$ for any choice of $g$. Thus the only possible
homomorphism $\varphi:G\rightarrow H$ is $\varphi(g)=1_H$.


\subsection{} % Section 3.b
Suppose for a contradiction that another such $\varphi$ exists. $\varphi$ is a
homomorphism, so we have that $\im\varphi\le\mathbb{Z}$. The only subgroups of
$\mathbb{Z}$ are of the form $n\mathbb{Z}$, all of which (other than
$0\mathbb{Z}$) are isomorphic to $\mathbb{Z}$. So if we assume $\varphi$ is
nontrivial, then its image is isomorphic to $\mathbb{Z}$.  If we let
$f:\im\varphi\rightarrow\mathbb{Z}$ be an isomorphism, then there exists a
$q\in\mathbb{Q}$ such that $f\circ\varphi(q)=1$. But
\begin{align*}
	1&=f\circ\varphi(q)\\
	&=f\circ\varphi(\frac{q}{2}+\frac{q}{2})\\
	&=f\circ\varphi(\frac{q}{2})+f\circ\varphi(\frac{q}{2})\\
	&=z+z\\
	&=2z
\end{align*}
for some $z\in\mathbb{Z}$, a contradiction, since 1 cannot be written as the
sum of any two integers. So the only possible homomorphism from $\mathbb{Q}$ to
$\mathbb{Z}$ is the trivial homomorphism, $\varphi(z)=0$.


\section{} % Section 4
Let $(g',h')\in G\oplus H$. $(g',h')H^*=\{(g',h'h)\mid h\in H\}$.
$H^*(g',h')=\{(g',hh')\mid h\in H\}$. But $h'h$ and $hh'$ are both already in
$H$, so both $(g',h')H^*$ and $H^*(g',h')$ can be rewritten as
$\{(g', h^*)\mid h^*\in H\}$. Thus $H^*\trianglelefteq G\oplus H$.
\newline
\newline
Define $\varphi:G\oplus H\rightarrow G$ by $\varphi((g,h))=g$.
\begin{align*}
	\ker\varphi&=\{(g,h)\in G\oplus H\mid\varphi((g,h))=1_G\}\\
	&=\{(1_G,h)\mid h\in H\}\\
	&= H^*
\end{align*}
By the first isomorphism theorem, we have that
$G\oplus h\mathbin/\ker\varphi\cong\im\varphi$. Let $g\in G$. Then
$\varphi((g,1_H))=g$, so $\varphi$ is surjective and $\im\varphi=G$. Thus we
have the following three identities:
\begin{align*}
	G\oplus H\mathbin/H^*&=G\oplus H\mathbin/\ker\varphi\\
	\im\varphi&=G\\
	G\oplus H\mathbin/\ker\varphi&\cong\im\varphi
\end{align*}
and so
\begin{align*}
	G\oplus H\mathbin/H^*\cong G
\end{align*}
as desired.


\section{} % Section 5
\subsection{} % Section 5.a
We use the one step subgroup test.
\newline
\newline
$\Inn(G)\neq\emptyset$, since if $\varphi:G\rightarrow G$,
$\varphi(a)=1a1^{-1}=a$, then $\varphi\in\Inn(G)$.
\newline
\newline
Let $\varphi,\psi\in\Inn(G)$. We show that $\varphi\circ\psi^{-1}\in\Inn(G)$.
There are $g,h\in G$ such that $\varphi(a)=gag^{-1}$ and $\psi(a)=hah^{-1}$.
Then $\psi^{-1}(a)=h^{-1}ah$, since $\psi(\psi^{-1}(a))=h(h^{-1}ah)h^{-1}=a$,
so $\psi^{-1}\in\Inn(G)$. Then $\varphi\circ\psi^{-1}(a)=gh^{-1}ahg^{-1}$.
$(gh^{-1})^{-1}=(hg^{-1})$, so $\varphi\circ\psi^{-1}\in\Inn(G)$, and thus
$\Inn(G)\le\Aut(G)$.


\subsection{} % Section 5.b
Let $g_1,g_2\in G$. Then
\begin{align*}
	\rho(g_1g_2)(h)&=g_1g_2hg_2^{-1}g_1^{-1}\\
	&=g_1(g_2hg_2^{-1})g_1^{-1}\\
	&=g_1(\rho(g_2)(h))g_1^{-1}\\
	&=\rho(g_1)(\rho(g_2)(h))\\
	&=\rho(g_1)\circ\rho(g_2)(h).
\end{align*}
So $\rho$ is a homomorphism.


\subsection{} % Section 5.c
\begin{align*}
	\ker\rho&=\{g\in G\mid\rho(g)(h)=ghg^{-1}=h\}\\
	&=\{g\in G\mid ghg^{-1}=h\}\\
	&=Z(G)
\end{align*}
By the first isomorphism theorem, we have that $G\mathbin/\ker\rho\cong\im\rho$.\\
So $G\mathbin/\ker\rho=G\mathbin/Z(G)\cong\im\rho$. We now show that
$\im\rho=\Inn(G)$.
\begin{align*}
	\im\rho&=\{\rho(g)(h)=ghg^{-1}\mid g\in G\}\\
	&=\{\varphi:G\rightarrow G\mid\text{there is some $g\in G$ such that for
		all $a\in G$, }\varphi(a)=gag^{-1}\}\\
	&=\Inn(G)
\end{align*}
So $G\mathbin/Z(G)\cong\im\rho$, and $\im\rho=\Inn(G)$, thus
$G\mathbin/Z(G)\cong\Inn(G)$.


\subsection{} % Section 5.d
Consider $1,r^2\in D_4$. $\rho(1)(h)=1h1^{-1}=h$, and $\rho(r^2)(h)=r^2hr^2=h$.
So $\rho$ is not injective, and is thus not an isomorphism.


\subsection{} % Section 5.e
Define $\varphi:D_4\rightarrow\Aut(D_4)$ by $\varphi(g)(h)=gh$. If
$g_1,g_2\in D_4$, then
\begin{align*}
	\varphi(g_1g_2)(h)&=g_1g_2h\\
	&=\varphi(g_1)(g_2h)\\
	&=\varphi(g_1)(\varphi(g_2)(h))\\
	&=\varphi(g_2)\circ\varphi(g_2)(h)
\end{align*}
so $\varphi$ is a homomorphism.  $\ker\varphi=\{g\in D_4\mid\varphi(g)(h)=h\}$.
Clearly $\ker\varphi=\{1\}$, so $\varphi$ is injective. Note that any element
of $D_4$ is of the form $r^aj^b$, for some integers $a$ and $b$. Define
$\psi:D_4\rightarrow D_4$ by $\psi(r^mj^q)=r^nj^r$, so that $\psi\in\Aut(D_4)$.
Then $\varphi(r^nj^{r+q}r^{4-m})=\psi$, since
\begin{align*}
	\varphi(r^nj^{r+q}r^{4-m})(r^mj^q)&=(r^nj^{r+q}r^{4-m})(r^mj^q)\\
	&=r^nj^{r+q}r^{4-m+m}j^q\\
	&=r^nj^{r+q}r^4j^q\\
	&=r^nj^{r+q}j^q\\
	&=r^nj^{r+2q}\\
	&=r^nj^r\\
	&=\psi(r^mj^q)
\end{align*}
So $\varphi$ is also surjective, and thus $\varphi$ is an isomorphism.

\end{document}
