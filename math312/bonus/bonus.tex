\documentclass[11pt]{article}
\usepackage{fancyhdr}
\pagestyle{fancy}
\newcommand\course{MATH 312}
\newcommand\duedate{November 14, 2019}

\lhead{V00885732}
\chead{\textbf{\Large Bonus Assignment}}
\rhead{\course\\\duedate}


\usepackage{amsmath,amssymb,mathtools}


\DeclareMathOperator{\im}{im}


\DeclarePairedDelimiter\abs{\lvert}{\rvert}%
\makeatletter
\let\oldabs\abs
\def\abs{\@ifstar{\oldabs}{\oldabs*}}


\begin{document}
\renewcommand{\thesubsection}{\thesection.\alph{subsection}}
\section{} % Section 1
$G$ is a finite abelian group, so it is isomorphic to the direct product of
finite cyclic groups of prime power order, and can thus be written as
$\bigoplus_i\mathbb{Z}_{p_i^{k_i}}$, where each $p_i$ is prime and each $k_i$
is an integer.

We know that every group of the form $\mathbb{Z}_n$ is a ring when equipped
with multiplication modulo $n$, so the direct product of finite cyclic groups
above is also a direct product of rings, giving us a ring whose additive group
is isomorphic to $G$, as desired.


\section{} % Section 2
Let $\frac{a}{b}+\mathbb{Z}\in\mathbb{Q}/\mathbb{Z}$, and suppose WLOG that
$\gcd(a,b)=1$. Then $\abs{\frac{a}{b}+\mathbb{Z}}=b$, since
$b\cdot\frac{a}{b}\in\mathbb{Z}$, and $a$ and $b$ are coprime. Thus any
element in $\mathbb{Q}/\mathbb{Z}$ has finite order $b$, but since
$b\in\mathbb{Z}$ and $\mathbb{Z}$ has no maximum element, this order can be
arbitrarily large.

Suppose for a contradiction that $R$ is a commutative unital ring whose
additive group is isomorphic to $\mathbb{Q}/\mathbb{Z}$. Let 1 be $R$'s
multiplicative identity. $(R,+)\cong\mathbb{Q}/\mathbb{Z}$, so 1 has finite
order, say $n$. Then for any $r\in R$, we have
\begin{align*}
	n\cdot r&=n\cdot(1\cdot r)\\
	&=(n\cdot1)\cdot r\\
	&=0\cdot r\\
	&=0
\end{align*}
so $\abs{r}\le n$, a contradiction, since the order of an element in
$\mathbb{Q}/\mathbb{Z}$ (and hence $(R,+)$) can be arbitrarily large, thus
greater than $n$, and so no such ring $R$ exists, as desired.


\section{} % Section 3
By the first isomorphism theorem, we know that $F/\ker\varphi\cong\im\varphi$.
We also know that the kernel of a homomorphism is an ideal. $F$ is a field, so
its only ideals are $\{0\}$ and itself. Suppose $\ker\varphi=F$. Then
$\im\varphi\cong F/F\cong\{0\}$. This is not possible since $\varphi$ is
unital, so it must be the case that $\ker\varphi=\{0\}$, and thus $\varphi$ is
injective, as desired.


\end{document}
