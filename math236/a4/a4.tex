\documentclass[11pt]{article}
\usepackage{fancyhdr}
\pagestyle{fancy}
\newcommand\course{MATH 236}
\newcommand\hwnumber{4}
\newcommand\duedate{March 13, 2020}

\lhead{Oliver Tonnesen\\V00885732}
\chead{\textbf{\Large Assignment \hwnumber}}
\rhead{\course\\\duedate}

\usepackage[left=1.5in,right=1.5in]{geometry}

\usepackage{amsmath, amssymb, mathtools}

\DeclarePairedDelimiter\abs{\lvert}{\rvert}%
\makeatletter
\let\oldabs\abs
\def\abs{\@ifstar{\oldabs}{\oldabs*}}
\let\ep\varepsilon
\DeclarePairedDelimiter\set{\left\{}{\right\}}%

\begin{document}
\renewcommand{\thesubsection}{\thesection.\alph{subsection}}
\section{} % Section 1
Let $\ep>0$.
We show that $\abs{b_n}<\ep$.
\newline
Since $a_n\rightarrow0$, there exists $N$ such that for any $n>N$, $\abs{a_n}<\ep$.
\newline
If we let $n>N$, then $\abs{b_n}\le a_n=\abs{a_n}<\abs{a_N}<\ep$, and so $b_n\rightarrow0$.


\section{} % Section 2
Let $A_n=a_1+\cdots+a_n$, $B_n=b_1+\cdots+b_n$.
Then $A_n\rightarrow\alpha$ and $B_n\rightarrow\beta$.
We know that the limit of the sum of two convergent sequences is the sum of their limits, so $A_n+B_n\rightarrow\alpha+\beta$.
\newline
\newline
Similarly, we know that the limit of the product of two convergent sequences is the product of their limits, so also $A_nB_n\rightarrow\alpha\beta$.


\section{} % Section 3
($\Longrightarrow$): We prove the contrapositive.
Let $a=1$, and let $s_n=1+\cdots+\frac{1}{n}$ be the partial sums of the series.
\begin{align*}
	s_n&=1+\cdots+\frac{1}{n}\\
	   &\ge1+\frac{1}{2}+\frac{1}{4}+\frac{1}{4}+\cdots+\frac{1}{2^k}
	   \qquad\text{Where the terms are grouped by powers of 2}\\
	   &\ge\frac{1}{2}+\frac{1}{2}+\cdots+\frac{1}{2}
	   \qquad\text{the constant sequence of $\frac{1}{2}$}
\end{align*}
As $n\rightarrow\infty$, the above sequence of $\frac{1}{2}$ converges to $\infty$, so by the comparison test, $s_n\rightarrow\infty$.
Clearly if $a<1$, then $\frac{1}{n^a}\le\frac{1}{n}$, so again by the comparison test $s_n=1+\cdots+\frac{1}{n^a}\rightarrow\infty$.
\newline
\newline
($\Longleftarrow$): Let $\ep>0$, and let $s_n$ be the sequence of partial sums for the series.
Pick $N$ such that $\frac{2}{2^N}<\ep$, and let $2^N<m<n<2^L$.
\begin{align*}
	\abs{s_n-s_m}&=\abs{\frac{1}{(m+1)^a}+\cdots+\frac{1}{n^a}}\\
				 &<\abs{\frac{1}{(2^N)^a}+\frac{1}{(2^N+1)^a}+\cdots+\frac{1}{(2^L)^a}}\\
				 &=\abs{\left(\frac{1}{(2^N)^a}+\cdots+
					 \frac{1}{(2^{N+1}-1)^a}\right)+\cdots+
					 \left(\frac{1}{(2^{L-1})^a}+\cdots+
				 \frac{1}{(2^L-1)^a}\right)}\\
				 &<\abs{\frac{1}{2^N}+\frac{1}{2^{N+1}}+\cdots+\frac{1}{2^{L-1}}}\\
				 &=\frac{2}{2^N}-\frac{1}{2^L}\\
				 &<\frac{2}{2^N}\\
				 &<\ep
\end{align*}
So by the Cauchy criterion for convergence, $s_n$ converges to a real number, as desired.


\section{} % Section 4
($\Longrightarrow$): We prove the contrapositive.
Let $a=1$.
For $n\ge3$, $\log n>1$, so it's clear to see that $\frac{1}{n\log n}\ge\frac{1}{n}$.
$\sum_{n=3}^\infty\frac{1}{n}=\infty$, so by the comparison test, $\sum_{n=3}^\infty\frac{1}{n\log n}=\infty$.
When $a<1$, $\sum_{n=3}^\infty\frac{1}{n(\log n)^a}\ge\sum_{n=3}^\infty\frac{1}{n\log n}$, so again by the comparison test, $\sum_{n=3}^\infty\frac{1}{n(\log n)^a}=\infty$.
\newline
\newline
($\Longleftarrow$): Let $a>1$, and $s_n$ be the sequence of partial sums for $\sum_{n=2}^\infty\frac{1}{n(\log n)^a}$.
Let $2^N<m<n<2^L$.
Then there exists some $c$ such that $cN\le\log m<\log n\le cL$.
So
\begin{align*}
	\abs{s_n-s_m}&=\abs{\frac{1}{(m+1)(\log(m+1))^a}+\cdots+
				 \frac{1}{n(\log n)^a}}\\
				 %
				 &<\Bigg\lvert\left(\frac{1}{2^N(\log 2^N)^a}+\cdots+
				 \frac{1}{(2^{N+1}-1)(\log(2^{N+1}-1))^a}\right)+\cdots\\
				 &+\left(\frac{1}{2^{L-1}(\log 2^{L-1})^a}+\cdots+
				 \frac{1}{(2^L-1)(\log(2^L-1))^a}\right)\Bigg\rvert\\
				 %
				 &<k\Bigg\lvert\left(\frac{1}{2^N(2^{N+1})^a}+\cdots
				 +\frac{1}{(2^{N+1}-1)(2^{N+1})^a}\right)+\cdots\\
				 &+\left(\frac{1}{2^{L-1}(2^L)^a}+\cdots+
				 \frac{1}{(2^L-1)(2^L)^a}\right)\Bigg\rvert
				 \qquad\text{where $k$ is a constant depending on $\ep$}\\
				 %
				 &<k\abs{\frac{1}{(2^{N+1})^a}+\cdots+\frac{1}{(2^{L-1})^a}}\\
				 %
				 &<k\abs{\frac{1}{2^{N+1}}+\cdots+\frac{1}{2^{L-1}}}\\
				 %
				 &=k\left(\frac{2}{2^{N+1}}-\frac{1}{2^L}\right)\\
				 %
				 &<\frac{k}{2^N}\\
				 %
				 &=\frac{k}{2^{\log_2\frac{k}{\ep}}}\\
				 %
				 &=\ep
\end{align*}
So $s_n$ is Cauchy and thus convergent to a real, so $\sum_{n=2}\frac{1}{n(\log n)^a}$ converges to a real.


\section{} % Section 5
($\Longrightarrow$): Let $a=1$.
When $n>321$, $\log n\log\log n\ge1$, so by the same argument as above, $\sum_{n=3}\frac{1}{n\log n\log\log n}$ converges, and so does $\sum_{n=3}\frac{1}{n\log n(\log\log n)^a}$ when $a<1$.
\newline
\newline
($\Longleftarrow$): 


\section{} % Section 6
Let $s_n$ be the sequence of partial sums of $\sum_{n=1}^\infty a_n$.
We $s_n$ converges, say $\lim_{n\rightarrow\infty}s_n=\alpha$.
Then
\begin{align*}
	\sum_{n=k}^\infty a_n&=\left(\lim_{n\rightarrow\infty}s_n\right)-s_{k-1}\\
						 &=\alpha-s_{k-1}
\end{align*}
and so
\begin{align*}
	b_k&=\lim_{k\rightarrow\infty}\sum_{n=k}^\infty a_n\\
	&=\lim_{k\rightarrow\infty}\left(\alpha-s_{k-1}\right)\\
	&=\alpha-\alpha\\
	&=0
\end{align*}


\section{} % Section 7
Let $x_0\in\mathbb{R}$.
Let $\ep>0$, and let $\delta=\ep$.
Then if $\abs{x-x_0}<\delta$, $\abs{x-x_0}<\ep$.
It's clear to see from the definition of $g(x)$ that $\abs{g(x)-g(x_0)}$ can be no greater than $\abs{x-x_0}$, so we have
\[\abs{g(x)-g(x_0)}\le\abs{x-x_0}<\ep\],
and so $g(x)$ is continuous.


\section{} % Section 8
\subsection{} % Section 8.a
We first show by induction that $\prod_{i=1}^m(1-a_i)\ge1-\sum_{i=1}^m a_i$:
\newline
When $m=1$, equality holds.
Assume the inequality holds for some $N$.
\begin{align*}
	\prod_{i=1}^{N+1}(1-a_i)&=(1-a_{N+1})\prod_{i=1}^N(1-a_i)\\
							&\ge(1-a_{N+1})\left(1-\sum_{i=1}^N a_i\right)\qquad\text{By the hypothesis}\\
							&=1-a_{N+1}-\sum_{i=1}^N a_i+a_{N+1}\sum_{i=1}^N a_i\\
							&\ge1-\sum_{i=1}^{N+1}a_i
\end{align*}
So by induction, the inequality holds.
\newline
\newline
Now,
\begin{align*}
	\lim_{n\rightarrow\infty}\prod_{i=1}^n(1-a_i)&\ge\lim_{n\rightarrow\infty}\left(1-\sum_{i=1}^n a_i\right)\\
												 &=1-\lim_{n\rightarrow\infty}\sum_{i=1}^n a_i,
\end{align*}
so $\prod_{i=1}^\infty(1-a_i)\ge1-\sum_{i=1}^\infty a_i$, as desired.


\subsection{} % Section 8.b
Let $a_i$ be non-negative.
Then $1-a_i^2\le1$.
Dividing both sides by $1-a_i$, we get $1+a_i\le\frac{1}{1-a_i}$.
We know, since $\sum_{i=1}^\infty a_i<\infty$, that $\sum_{i=k}^\infty a_i\rightarrow0$.
Consider $\prod_{i=k}^\infty(1+a_i)\le\prod_{i=k}^\infty\frac{1}{1-a_i}$.
$a_n\le a_k$ for all $n>k$, so $\prod_{i=k}^\infty\frac{1}{1-a_i}\le\left(\frac{1}{1-a_k}\right)\left(\frac{1}{1-a_k}\right)\cdots$.
As $k\rightarrow\infty$, $1-a_k\rightarrow1$, so $\left(\frac{1}{1-a_k}\right)\left(\frac{1}{1-a_k}\right)\cdots\rightarrow1$.
$\prod_{i=1}^k$ is finite, so $\prod_{i=1}^\infty\frac{1}{1-a_i}$ is too, and so $\prod_{i=1}^\infty(1+a_i)$ exists and is finite.


\subsection{} % Section 8.c
Each of $a_1,a_2,\ldots$ appears as a term in the expansion of $(1+a_1)(1+a_2)\cdots$, so $\sum_{i=1}^\infty a_i\le\prod_{i=1}^\infty(1+a_i)$.
Thus by the comparison test, since $\prod_{i=1}^\infty(1+a_i)$ converges, so does  $\sum_{i=1}^\infty a_i$.


\end{document}
