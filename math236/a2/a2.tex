\documentclass[11pt]{article}
\usepackage{fancyhdr}
\pagestyle{fancy}
\newcommand\course{MATH 236}
\newcommand\hwnumber{2}
\newcommand\duedate{February 7, 2020}

\lhead{Oliver Tonnesen\\V00885732}
\chead{\textbf{\Large Assignment \hwnumber}}
\rhead{\course\\\duedate}

\usepackage[left=1.5in,right=1.5in]{geometry}

\usepackage{amsmath, amssymb, mathtools}

\DeclarePairedDelimiter\abs{\lvert}{\rvert}%
\makeatletter
\let\oldabs\abs
\def\abs{\@ifstar{\oldabs}{\oldabs*}}
\let\ep\varepsilon
\DeclarePairedDelimiter\set{\left\{}{\right\}}%

\begin{document}
\renewcommand{\thesubsection}{\thesection.\alph{subsection}}
\section{} % Section 1
Let $\ep>0$, and let $N=\frac{1/\ep+3}{2}$.
Then for any $n>N$,
\begin{align*}
	\abs{\frac{1}{2n-3}}&<\abs{\frac{1}{2N-3}}\\
						&=\abs{\frac{1}{2\left(\frac{1/\ep+3}{2}\right)-3}}\\
						&=\abs{\ep}\\
						&=\ep
\end{align*}
and so, indeed, $a_n\rightarrow0$, as desired.


\section{} % Section 2
We claim the limit is $b=\frac{3}{2}$.
Note first that $\abs{\frac{3n^2+4n+5}{2n^2+6n+7}-\frac{3}{2}}=\abs{\frac{5+11/2n}{2n+6+7/n}}$.
Let $\ep>0$, and let $N=\max\left\{7,\frac{3}{\ep}\right\}$.
If $n>7$, then $5+\frac{11}{2n}<6$ and $2n+6+\frac{7}{n}<2n+7$.
Then for any $n>N$,
\begin{align*}
	\abs{\frac{5+11/2n}{2n+6+7/n}}&<\abs{\frac{6}{2n+6+7/n}}\\
								  &<\abs{\frac{6}{2n+7}}\\
								  &<\abs{\frac{3}{n}}\\
								  &<\abs{\frac{3}{N}}\\
								  &=\abs{\frac{3}{3/\ep}}\\
								  &=\ep
\end{align*}
and so $b_n\rightarrow\frac{3}{2}$, as desired.


\section{} % Section 3
Let $M>0$.
$y_n\rightarrow\infty$, so we know there is some $N$ such that $y_n>\frac{M}{\inf\left\{a_n\right\}}$ for any $n>N$.
Then $\inf\left\{a_n\right\}y_n>M$, so $a_ny_n>M$, and thus $a_ny_n\rightarrow\infty$, as desired.


\section{} % Section 4
We claim the limit is $0$.
Let $\ep>0$, and let $N=\max\left\{\frac{5}{\ep},\frac{1}{\sqrt[5]{\ep}}\right\}$.
Then for any $n>N$:
\newline
if $n$ is even, then $\abs{s_n}=\abs{\frac{5}{n}}<\abs{\frac{5}{N}}\le\abs{\frac{5}{5/\ep}}=\ep$, and
\newline
if $n$ is odd, then $\abs{s_n}=\abs{\frac{1}{n^5}}<\abs{\frac{1}{N^5}}\le\abs{\frac{1}{1/\sqrt[5]{\ep}}}=\ep$.
\newline
Thus $s_n\rightarrow0$, as desired.


\section{} % Section 5
We claim the limit is $0$.
Let $\ep>0$, and let $N=\frac{1}{2\ep}$.
Then
\begin{align*}
	\abs{\sqrt{n}-\sqrt{n-1}}&=\abs{\frac{1}{\sqrt{n}+\sqrt{n-1}}}\\
							 &<\abs{\frac{1}{2\sqrt{n}}}\\
							 &<\abs{\frac{1}{2n}}\\
							 &<\abs{\frac{1}{2N}}\\
							 &=\abs{\frac{1}{2/2\ep}}\\
							 &=\ep
\end{align*}
and so $u_n\rightarrow\infty$, as desired.


\section{} % Section 6
We claim the sequence converges to $\infty$.
Let $M>0$, and let $N=\max\left\{3,M^2\sqrt{32}\right\}$.
If $n>3$, then $3+\frac{3}{n}+\frac{1}{n^2}<4$.
\begin{align*}
	\abs{(n+1)^{3/2}-n^{3/2}}&=\abs{\frac{(n+1)^3-n^3}{(n+1)^{3/2}+n^{3/2}}}\\
							 &=\abs{\frac{n^3+3n^2+3n+1-n^3}{(n+1)^{3/2}+n^{3/2}}}\\
							 &=\abs{\frac{3n^2+3n+1}{(n+1)^{3/2}+n^{3/2}}}\\
\end{align*}
\begin{align*}
							 &>\abs{\frac{3n^2+3n+1}{2(n+1)^{3/2}}}\\
							 &=\abs{\frac{3n^2+3n+1}{2\sqrt{n^3+3n^2+3n+1}}}\\
							 &>\abs{\frac{3n^2+3n+1}{2\sqrt{8n^3}}}\\
							 &=\abs{\frac{3n^2+3n+1}{\sqrt{32}n^{3/2}}}\\
							 &=\abs{\frac{3\sqrt{n}+3/\sqrt{n}+1/n^{3/2}}{\sqrt{32}}}\\
							 &>\abs{\frac{3\sqrt{n}}{\sqrt{32}}}\\
							 &>\abs{\frac{3\sqrt{N}}{\sqrt{32}}}\\
							 &=3M\\
							 &>M
\end{align*}
and so $v_n\rightarrow\infty$, as desired.


\section{} % Section 7
We know that $x_n\rightarrow\infty$, so for any $\ep>0$, there is some $N$ such that $\abs{x_n-15}<\ep$ for all $n>N$.
If we take $\ep$ to be less than $5$, then all $\abs{x_n}$, $n>N$, will lie between $10$ and $20$.
This means that all $x_n$ lying outside this range occur before $N$.
All $x_n$ with $\abs{x_n}>20$ will clearly be a subset of these.
More formally, $\left\{n\mid\abs{x_n}>20\right\}\subseteq\left\{n\mid n\le N\right\}$, and thus is finite, as desired.


\section{} % Section 8
\subsection{} % Section 8.a
Suppose it did, and let $N$ be such a number that the condition holds.
Let $\ep=\frac{1}{N+2}$.
It is clearly not the case that $\abs{s_n}<\ep$ for all $n>N$, since $N+1>N$, but $s_{N+1}=\frac{1}{N+1}\ge\ep$.
Thus forms a contradiction, and so it is not the case that $s\rightsquigarrow0$.


\subsection{} % Section 8.b
Let $N$ be a number satisfying the conditions for $s_n\rightsquigarrow s$, and let $\ep>0$.
By the definition of $s_n\rightsquigarrow s$, if $n>N$, then $\abs{s_n-s}<\ep$, so indeed $s_n\rightarrow s$, as desired.


\subsection{} % Section 8.c
Let $a_n=\begin{cases}
	1 & \text{if there is a solution to }x^n+y^n=z^n\\
	0 & \text{otherwise}
\end{cases}$
\newline
\newline
It is the case that $a_n\rightsquigarrow0$.
I have a truly marvelous proof this, which this homework is too narrow to contain.


\subsection{} % Section 8.d
The condition $s_n\rightsquigarrow s$ holds when $s_n$ eventually becomes constant.
That is, there is some $N$ such that $s_N=s_{N+1}=\cdots=s$


\end{document}
