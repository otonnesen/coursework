\documentclass[11pt]{article}
\usepackage{fancyhdr}
\pagestyle{fancy}
\newcommand\course{MATH 236}
\newcommand\hwnumber{5}
\newcommand\duedate{March 27, 2020}

\lhead{Oliver Tonnesen\\V00885732}
\chead{\textbf{\Large Assignment \hwnumber}}
\rhead{\course\\\duedate}

\usepackage[left=1.5in,right=1.5in]{geometry}

\usepackage{amsmath, amssymb, mathtools}

\DeclarePairedDelimiter\abs{\lvert}{\rvert}%
\makeatletter
\let\oldabs\abs
\def\abs{\@ifstar{\oldabs}{\oldabs*}}
\let\ep\varepsilon
\DeclarePairedDelimiter\set{\left\{}{\right\}}%

\begin{document}
\renewcommand{\thesubsection}{\thesection.\alph{subsection}}
\section{} % Section 1
We already know by its definition that $f$ is continuous at each point in $\left[a,c\right]\setminus\{b\}$, so we need only show that $f$ is also continuous at $b$.
Let $\ep>0$, and WLOG let $x\in\left[a,b\right]$.
We know that $f$ is continuous over $\left[a,b\right]$, so there is some $\delta$ such that if $\abs{x-b}<\ep$, then $\abs{f(x)-f(b)}<\delta$.
So $f$ is also continuous at $b$, and is thus continuous on all of $\left[a,c\right]$.


\section{} % Section 2
$f$ is continuous at $x=0$, and discontinuous elsewhere.
\newline
\newline
\underline{$f$ continuous at 0}:
Let $\ep>0$, and let $\delta=\ep$.
If $x\in\mathbb{R}\setminus\mathbb{Q}$, then $f(x)=x$, and so $\abs{f(x)}=\abs{x}<\ep=\delta$, so $\abs{f(x)}<\delta$.
If $x\in\mathbb{Q}$, then $\abs{f(x)}=0<\ep=\delta$, so again $f(x)<\delta$, as desired, and so $f$ is continuous at $x=0$.
\newline
\newline
\underline{$f$ discontinuous elsewhere}:
Let $0\neq\alpha\in\mathbb{Q}$, and let $x_n\rightarrow\alpha$ be a sequence of irrational numbers.
Then $f(x_n)=0$ for all $n$, so $lim_{n\rightarrow\infty}f(x_n)=0$, but $f(\alpha)=\alpha\neq0$, defying the limit definition of continuity, so $f$ is discontinuous at $x\neq0$.


\section{} % Section 3
Let $\ep>0$.
$S_n$ converges to $L\in\mathbb{R}$, so there exists some $N$ such that for all $n>N$, $\abs{S_n-L}<\ep$.
Note that the sequence of partial sums of $a_n$, say $T_n$, corresponds to $S_\frac{n}{2}$.
Let $N'=2N$.
Then for any $n>N'$, we have
\begin{align*}
	\abs{T_n-L} &= \abs{S_\frac{n}{2}-L}\\
				&< \abs{S_\frac{N'}{2}-L}\\
				&= \abs{S_\frac{2N}{2}-L}\\
				&= \abs{S_N-L}\\
				&< \ep
\end{align*}
So $T_n\rightarrow L$, and thus $\sum_{k=1}^\infty a_k=L$, as desired.


\section{} % Section 4
\subsection{} % Section 4.a
We show that $a_{n+1}-a_n>0$ for any $n$.
\begin{align*}
	a_{n+1}-a_n &= \left(1-\frac{1}{2}+\cdots+\frac{1}{2n+1}-\frac{1}{2n+2}\right)-\left(1-\frac{1}{2}+\cdots+\frac{1}{2n-1}-\frac{1}{2n}\right)\\
			    &= \frac{1}{2n+1}-\frac{1}{2n+2}\\
				&> 0
\end{align*}
So $(a_n)$ is monotonic increasing.


\subsection{} % Section 4.b
We show that $b_{n+1}-b_n<0$ for any $n$.
\begin{align*}
	b_{n+1}-b_n &= \left(1-\cdots-\frac{1}{2n}+\frac{1}{2n+1}\right)-\left(1-\cdots-\frac{1}{2n-2}+\frac{1}{2n-1}\right)\\
				&= \frac{1}{2n+1}-\frac{1}{2n+2}\\
				&> 0
\end{align*}
So $(b_n)$ is monotonic decreasing.


\subsection{} % Section 4.c
It's clear to see that $a_m<a_N$ and $b_N<b_n$, so we need only show that $a_N<b_N$.
Note first that by definition, $a_N = b_N + \frac{(-1)^{2N+1}}{2N}$.
$2N+1$ is always odd, so we have $a_N = b_N - \frac{1}{2N}$.
$\frac{1}{2N}>0$, so $a_N<b_N$, as desired.


\subsection{} % Section 4.d
$(a_n)$ is monotonic increasing and as we saw in part (c), it is bounded above by all terms of $(b_n)$, so it converges to a real number $a$.
Similarly, $(b_n)$ is monotinic decreasing and is bounded below by all terms of $(a_n)$, so it converges to a real number $b$.


\subsection{} % Section 4.e
Let $\ep>0$, and let $N=\frac{1}{2\ep}$.
Then for any $n>N$,
\begin{align*}
	\abs{b_n-a_n} &= \frac{1}{2n}\\
				  &< \frac{1}{2N}\\
				  &= \frac{1}{2\frac{1}{2\ep}}\\
				  &= \ep
\end{align*}
The terms of $(a_n)$ and $(b_n)$ get arbitrarily close, so indeed $\lim_{n\rightarrow\infty}a_n=a=b=\lim_{n\rightarrow\infty}b_n$, as desired.


\subsection{} % Section 4.f
Let $s_n=\sum_{k=1}^n\frac{(-1)^k+1}{k}$.
$(a_n)$ and $(b_n)$ represent the even and odd terms of $(s_n)$, respectively.
Both converge to $a$, and so the whole sequence $(s_n)$ must converge to $a$.


\subsection{} % Section 4.g
Every $3n$th partial sum of $(z_n)$ is simply the $3(n-1)$th partial sum of $(z_n)$ added to $\frac{1}{4n-3}+\frac{1}{4n-1}-\frac{1}{2n}$.
In other words, the $3n$th partial sum of $(z_n)$ is $\sum_{j=1}^n\left(\frac{1}{4j-3}+\frac{1}{4j-1}-\frac{1}{2j}\right)$.
\begin{align*}
	&\sum_{j=1}^n\left(\frac{1}{4j-3}-\frac{1}{4j-2}+\frac{1}{4j-1}-\frac{1}{4j}\right) + \sum_{j=1}^n\left(\frac{1}{4j-2}-\frac{1}{4j}\right)\\
	= &\sum_{j=1}^n\left(\frac{1}{4j-3}-\frac{1}{4j-2}+\frac{1}{4j-1}-\frac{1}{4j}+\frac{1}{4j-2}-\frac{1}{4j}\right)\\
	= &\sum_{j=1}^n\left(\frac{1}{4j-3}+\frac{1}{4j-1}-\frac{2}{4j}\right)\\
	= &\sum_{j=1}^n\left(\frac{1}{4j-3}+\frac{1}{4j-1}-\frac{1}{2j}\right)
\end{align*}
So indeed the partial sums of $(z_n)$ are equal to the given expression.


\subsection{} % Section 4.h
Note first that the first term of the expression is just the $(3n)$th partial sum of $\sum_{k=1}^\infty\frac{(-1)^{k+1}}{k}$, and that the second term of the expression is half that.
So, since $\sum_{k=1}^\infty\frac{(-1)^{k+1}}{k}$, as we saw in (f), it must be the case that $\sum_{n=1}^\infty z_n=\frac{3a}{2}$.


\section{} % Section 5
Let $a_n=\frac{1}{2n}$.
We know that this subsequence consisting of the even terms $\frac{(-1)^n}{n}$ converges to $\infty$, so for any $M>0$, there exists $N$ such that $s_n=\sum_{i=1}^na_n>M$ for all $n>N$.
We construct our sequence as follows:
\newline
\newline
Beginning with an empty sequence, we choose $N_1$ such that $s_n>2$ for all $n>N_1$.
We append the first $N_1$ terms of $a_n$ to our sequence, followed by the first odd term.
Now our sequence sums to at least 1.
\newline
\newline
Next, we choose $N_2$ such that $s_n>3$ for all $n>N_2$.
We append the next even terms of $a_n$ up to $N_2$ to our sequence, followed by the second odd term.
\newline
\newline
At each step $i$, we choose $N_i$ such that $s_n>i$ for all $n>N_i$, and append to our sequence from the $N_{i-1}$ to $N_i$th even terms, followed by the $i$th odd term.
In the end, we end up with the sequence
\[\left(\frac{1}{2}+\frac{1}{4}+\cdots+\frac{1}{2N_1}-1\right)+\Big(\cdots\Big)+\cdots+\Big(\cdots\Big)+\left(\frac{1}{2N_{i-1}+2}+\frac{1}{2N_{i-1}+3}+\cdots+\frac{1}{2N_i}-\frac{1}{i}\right)+\cdots.\]
At each step in this process, our sequence sums to at least $i$, so it is clear to see that it indeed converges to $\infty$ as $n$ grows large.


\section{} % Section 6


\end{document}
