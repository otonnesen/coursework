\documentclass[11pt]{article}
\usepackage{fancyhdr}
\pagestyle{fancy}
\newcommand\course{MATH 236}
\newcommand\hwnumber{1}
\newcommand\duedate{January 24, 2020}

\lhead{Oliver Tonnesen\\V00885732}
\chead{\textbf{\Large Assignment \hwnumber}}
\rhead{\course\\\duedate}

\usepackage{amsmath, amssymb, mathtools}

\DeclarePairedDelimiter\abs{\lvert}{\rvert}%
\makeatletter
\let\oldabs\abs
\def\abs{\@ifstar{\oldabs}{\oldabs*}}

\begin{document}
\renewcommand{\thesubsection}{\thesection.\alph{subsection}}
\section{} % Section 1
We know $0<a<b$.
$a$ and $b$ are nonzero, so $ab$ is nonzero, and thus has an inverse with respect to multiplication, $b^{-1}a^{-1}$.
$b^{-1}a^{-1}>0$ since $ab>0$, so since if $a<b$ and $c>0$, $ac<bc$, we can conclude that since $0<a<b$, $0\cdot b^{-1}a^{-1}<a\cdot b^{-1}a^{-1}<b\cdot b^{-1}a^{-1}$.
Multiplication is commutative over $\mathbb{R}$, so this means $0<b^{-1}<a^{-1}$, or $0<\frac{1}{b}<\frac{1}{a}$, as desired.


\section{} % Section 2
[Induction on $n$]: When $n=2$, this is simply the triangle inequality.
Suppose the claim holds for all $n\le N$ for some $N$.
Let $a_1,\ldots,a_{N+1}\in\mathbb{Q}$. Consider $\abs{a_1+\ldots+a_{N+1}}$, and denote $b=a_1+\ldots+a_N$. Then $b\in\mathbb{Q}$, and $\abs{a_1+\ldots+a_{N+1}}=\abs{b+a_{N+1}}$.
By the hypothesis, $\abs{b+a_{N+1}}\le\abs{b}+\abs{a_{N+1}}$.
We know $\abs{b}=\abs{a_1+\ldots+a_N}\le\abs{a_1}+\ldots+\abs{a_N}$, so
\begin{align*}
	\abs{a_1+\ldots+a_{N+1}}&=\abs{b+a_{N+1}}\\
	&\le\abs{b}+\abs{a_{N+1}}\\
	&=\abs{a_1+\ldots+a_N}+\abs{a_{N+1}}\\
	&\le\abs{a_1}+\ldots+\abs{a_N}+\abs{a_{N+1}}
\end{align*}
So by induction, the claim holds.


\section{} % Section 3
\subsection{} % Section 3.a
(i). No.
(ii). Yes, 7.


\subsection{} % Section 3.b
(i). Yes.
(ii). Yes, 2.


\subsection{} % Section 3.c
(i). No.
(ii). Yes, 90.


\subsection{} % Section 3.d
(i). No.
(ii). Yes, 10.


\subsection{} % Section 3.e
(i). No.
(ii). Yes, 1.


\section{} % Section 4
\subsection{} % Section 4.a
(i). No.
(ii). Yes, -1.


\subsection{} % Section 4.b
(i). Yes.
(ii). Yes, 1.


\subsection{} % Section 4.c
(i). No.
(ii). No.


\subsection{} % Section 4.d
(i). No.
(ii). No.


\subsection{} % Section 4.e
(i). No.
(ii). Yes, 1.


\section{} % Section 5
Suppose not.
Then there is some $x\in S\cap T$.
$x\in S$, so $x\ge\inf S$.
Similarly, $s\in T$, so $x\le\sup T$.
Thus we have $x\le\sup T<\inf S\le x$, so $x<x$, a contradiction, and so $S\cap T=\emptyset$, as desired.


\section{} % Section 6
Yes: $S=\left(0,1\right)$, $T=\{0,1\}$.


\section{} % Section 7
We know that if $a,b\in\mathbb{R}$ such that $a<b$, then there exists $q\in\mathbb{Q}$ with $a<q<b$.

Suppose $S$ did have a maximum element, say $s$.
$s\in\mathbb{Q}\subseteq\mathbb{R}$, so by the denseness of $\mathbb{Q}$ in $\mathbb{R}$, there exists $r\in\mathbb{Q}$ such that $q<r<\sqrt{2}$.
We see by the definition of $s$ that it contains $r$, so $q$ is not the maximum element of $S$, a contradiction, and so $S$ has no maximum element.


\section{} % Section 8
We consider two cases: $a<1$, and $a\ge1$.

\noindent
\underline{Case $a<1$}: $a\neq0$, so there must be some $n\in\mathbb{N}$ such that $an>1$.
Then $a>\frac{1}{n}$.
$n\ge1$, and $a<1$, so we also have $n>a$.
Thus $\frac{1}{n}<a<n$, as desired.

\noindent
\underline{Case $a\ge1$}: Simply choose $n=\lceil(a)\rceil+1$.
$n\ge2$, and $a\ge1$, so $\frac{1}{n}\le\frac{1}{2}<a$, so we have $\frac{1}{n}<a<n$, as desired.


\section{} % Section 9
\underline{$\left[\frac{1}{2}\left(a+\frac{2}{a}\right)\right]^2>2$}:
\begin{align*}
	\left[\frac{1}{2}\left(a+\frac{2}{a}\right)\right]^2
	&=\left[\frac{1}{2a}\left(a^2+2\right)\right]^2\\
	&=\frac{1}{4a^2}\left(a^2+2\right)^2\\
	&=\frac{1}{4a^2}\left(a^4+4a^2+4\right)\\
	&=\frac{a^2}{4}+1+\frac{1}{a^2}\\
	&>2+\frac{1}{a^2}\\
	&>2
\end{align*}

\noindent
\underline{$\frac{1}{2}\left(a+\frac{2}{a}\right)<a$}:
\begin{align*}
	\frac{1}{2}\left(a+\frac{2}{a}\right)
	&=\frac{a}{2}+\frac{1}{a}\\
	&=\frac{a}{2}+\frac{a}{a^2}\\
	&<\frac{a}{2}+\frac{a}{2}\\
	&=a
\end{align*}

\noindent
\underline{$\left[4/\left(a+\frac{2}{a}\right)\right]^2<2$}:
\begin{align*}
	\left[4/\left(a+\frac{2}{a}\right)\right]^2
	&=\left(\frac{4a}{a^2+2}\right)^2\\
	&=\frac{16a^2}{a^4+4a^2+4}\\
	&<\frac{32}{4+8+4}\\
	&=2
\end{align*}

\noindent
\underline{$a<4/\left(a+\frac{2}{a}\right)$}:
\begin{align*}
	\frac{4}{a+\frac{2}{a}}
	&=\frac{4}{a+\frac{2}{a}}\\
	&=\frac{4a}{a^2+2}\\
	&>\frac{4a}{2+2}\\
	&=a
\end{align*}

\noindent
If $a^2>2$, then $\left[\frac{1}{2}\left(a+\frac{2}{a}\right)\right]^2>2$, so it is not in $S$, a contradiction, since $0\le\frac{1}{2}\left(a+\frac{2}{a}\right)<a=\sup S$.

\noindent
If $a^2<2$, then $0\le\left[4/\left(a+\frac{2}{a}\right)\right]^2<2$, so it is in $S$, a contradiction, since $\sup S=a<4/\left(a+\frac{2}{a}\right)$

\noindent
So it must be the case that $a^2=2$, as desired.


\end{document}
