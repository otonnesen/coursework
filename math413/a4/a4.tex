\documentclass[11pt]{article}
\usepackage{fancyhdr}
\pagestyle{fancy}
\newcommand\course{MATH 413}
\newcommand\hwnumber{4}
\newcommand\duedate{March 20, 2020}

\lhead{Oliver Tonnesen\\V00885732}
\chead{\textbf{\Large Assignment \hwnumber}}
\rhead{\course\\\duedate}


\usepackage{amsmath, amssymb, mathtools}

\DeclareMathOperator{\Span}{Span}
\def\F{\mathbb{F}}
\def\a{\alpha}
\DeclarePairedDelimiter\abs{\lvert}{\rvert}%
\makeatletter
\let\oldabs\abs
\def\abs{\@ifstar{\oldabs}{\oldabs*}}

\begin{document}
\renewcommand{\thesubsection}{\thesection.\alph{subsection}}
\section{} % Section 1
\begin{align*}
	C&=\left\{u_1u_2u_3u_4\in\F_3^4\mid u_1+u_2=u_3+u_4=0\right\}\\
	 &=\left\{0000,0012,0021,1200,1212,1221,2100,2112,2121\right\}
\end{align*}
The coset containing 1000 also contains 0100.
The same is true of 0010 and 0001, 2000 and 0200, and 0020 and 0002, so each of the four remaining cosets' least weight representatives have weight at least 2.
It's easy to see that 1010, 2020, 2001, and 1002 all represent different cosets, so the least weight representatives for each coset are the following:
0000, 1000, 0010, 2000, 0020, 1010, 2020, 2001, 1002.


\section{} % Section 2
\subsection{} % Section 2.a
$q^n=243>233=\sum_{i=0}^{d-2}\binom{n-1}{i}\left(q-1\right)^i$, so we can use the construction given in the proof of proposition 5.13 in the course notes to get
\[H=\begin{pmatrix}
	1 & 0 & 0 & 0 & 0 & 1 & 2\\
	0 & 1 & 0 & 0 & 0 & 1 & 2\\
	0 & 0 & 1 & 0 & 0 & 1 & 1\\
	0 & 0 & 0 & 1 & 0 & 1 & 1\\
	0 & 0 & 0 & 0 & 1 & 0 & 1\\
\end{pmatrix}.\]

\subsection{} % Section 2.b
We know if $H$ is the parity check matrix for $C$ such that $H^\top=\begin{pmatrix}-X\\\hline I_{n-k}\end{pmatrix}$, then $G=\begin{pmatrix}I_k\mid X\end{pmatrix}$ is its generator matrix.
Permuting $H$, we get
\[H'=\begin{pmatrix}
	1 & 2 & 1 & 0 & 0 & 0 & 0\\
	1 & 2 & 0 & 1 & 0 & 0 & 0\\
	1 & 1 & 0 & 0 & 1 & 0 & 0\\
	1 & 1 & 0 & 0 & 0 & 1 & 0\\
	0 & 1 & 0 & 0 & 0 & 0 & 1\\
\end{pmatrix}.\]
Then 
\[H'^\top=\begin{pmatrix}
	1 & 1 & 1 & 1 & 0\\
	2 & 2 & 1 & 1 & 1\\
	1 & 0 & 0 & 0 & 0\\
	0 & 1 & 0 & 0 & 0\\
	0 & 0 & 1 & 0 & 0\\
	0 & 0 & 0 & 1 & 0\\
	0 & 0 & 0 & 0 & 1\\
\end{pmatrix}\]
is in the desired form, so we get
\[-X=\begin{pmatrix}
	1 & 1 & 1 & 1 & 0\\
	2 & 2 & 1 & 1 & 1\\
\end{pmatrix},\]
and so
\[G=\begin{pmatrix}
	1 & 0 & 2 & 2 & 2 & 2 & 0\\
	0 & 1 & 1 & 1 & 2 & 2 & 2\\
\end{pmatrix},\]
$G\in\F_3^{2\times7}$, so the code it generates, $C$, has dimension 2 over $\F_3^7$.
This means $\abs{C}=3^2$.


\section{} % Section 3
\subsection{} % Section 3.a
For $C$, $w_0=1$, $w_1=0$, $w_2=0$, $w_3=0$, and $w_4=1$.
$\left\{1111\right\}$ is a basis for $C$, so $G=\begin{pmatrix}1&1&1&1\end{pmatrix}$ is a generator matrix for $C$.
This gives us a parity check matrix,
\[H=\begin{pmatrix}
	1 & 1 & 0 & 0\\
	1 & 0 & 1 & 0\\
	1 & 0 & 0 & 1
\end{pmatrix}.\]
So $C^\perp=\left\{0000,1001,1010,0011,1100,0101,0110,1111\right\}$.
This means for $C^\perp$, $w_0=1$, $w_1=0$, $w_2=6$, $w_3=0$, and $w_4=1$.
So using the above coefficients, we get
\newline
\newline
$A(x,y)=y^4+x^4$
\newline
\newline
and
\newline
\newline
$A^\perp(x,y)=y^4+6x^2y^2+x^4$.


\subsection{} % Section 3.b
\begin{align*}
	A(y-x,x+y)&=(x+y)^4+(y-x)^4\\
			  &=x^4+4x^3y+6x^2y^2+4xy^3+y^4\\
			  &+y^4-4y^3x+6y^2x^2-4yx^3+y^4\\
			  &=2x^4+12x^2y^2+2y^4
\end{align*}
We saw that $k=\dim C=1$, so in this case, $\frac{1}{2^k}A(y-x,x+y)=\frac{1}{2}A(y-x,x+y)=x^4+6x^2y^2+y^4=A^\perp(x,y)$, as desired.


\section{} % Section 4
$C_{23}$ is perfect, so it attains equality in the Hamming bound -- that is, the balls of radius $\left\lfloor\frac{d-1}{2}\right\rfloor$ partition $\F_2^{23}$.
In this case, $d=d_\text{min}(C_{23})=7$, so $\left\lfloor\frac{d-1}{2}\right\rfloor=3$.
This means each word of weight 4 in $\F_2^{23}$ is contained in exactly one ball around a codeword of weight 7.
There are $\binom{7}{3}$ words of weight 4 at distance 3 from any given codeword of weight 7, so we can count the number of code words of weight 7 by counting the number of balls needed to contain all of the weight 4 words in $\F_2^{23}$: $\frac{\binom{23}{4}}{\binom{7}{3}}=253$.


\section{} % Section 5
When $t=1$, we know that the only nontrivial $t$-error-correcting perfect codes are the Golay codes, which certainly fail to exist for sufficiently large primes.
\newline
\newline
When $t>1$, the only perfect codes are the Hamming codes.
These codes have $n=\frac{q^r-1}{q-1}$, so if $n$ is fixed, then $\frac{q^r-1}{q-1}$ is increasing in $q$, so for larger $q$, no Hamming codes of length $n$ can exist.


\section{} % Section 6
Consider the $k$-th term of $L_t(x)$, $(-1)^k(q-1)^{t-k}\binom{n-x}{t-k}\binom{x-1}{k}$.
When $x=0$,
\begin{align*}
	\binom{n-x}{t-k}&=\frac{(n-x)!}{(t-k)!((n-x)-(t-k))!}\\
					&=\frac{n!}{(t-k)!(n-(t-k))!}\\
					&=\binom{n}{t-k}
\end{align*}
and
\begin{align*}
	\binom{x-1}{k}&=\frac{(x-1)!}{k!(x-1-k)!}\\
				  &=\frac{(x-1)(x-2)\cdots(x-k)}{k!}\\
				  &=\frac{(-1)^kk!}{k!}\\
				  &=(-1)^k.
\end{align*}
So
\begin{align*}
	L_t(0)&=(-1)^k(q-1)^{t-k}\binom{n}{t-k}(-1)^k\\
		  &=(-1)^{2k}(q-1)^{t-k}\binom{n}{t-k}\\
		  &=(q-1)^{t-k}\binom{n}{t-k}.
\end{align*}
If we now compare $L_t(0)$ and $\abs{B_t}=\sum_{i=0}^t\binom{n}{i}(q-1)^i$, we can see that the $k$-th term of $L_t(0)$ is the same as the $(t-k)$-th term of $\abs{B_t}$.
So the constant term in the Lloyd polynomial equals the size of the ball of radius $t$, as desired.


\section{} % Section 7
\subsection{} % Section 7.a
\[2A_1+3A_3\]


\subsection{} % Section 7.b
\[3A_0+2A_2\]


\end{document}
