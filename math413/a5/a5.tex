\documentclass[11pt]{article}
\usepackage{fancyhdr}
\pagestyle{fancy}
\newcommand\course{MATH 413}
\newcommand\hwnumber{5}
\newcommand\duedate{April 3, 2020}

\lhead{Oliver Tonnesen\\V00885732}
\chead{\textbf{\Large Assignment \hwnumber}}
\rhead{\course\\\duedate}


\usepackage{amsmath, amssymb, mathtools}

\DeclareMathOperator{\Span}{Span}
\DeclareMathOperator{\lcm}{lcm}
\def\F{\mathbb{F}}
\def\a{\alpha}
\DeclarePairedDelimiter\abs{\lvert}{\rvert}%
\makeatletter
\let\oldabs\abs
\def\abs{\@ifstar{\oldabs}{\oldabs*}}

\begin{document}
\renewcommand{\thesubsection}{\thesection.\alph{subsection}}
\section{} % Section 1
\begin{align*}
	L_1(x) &= (q-1)\binom{n-x}{1}\binom{x-1}{0} + (-1)\binom{n-x}{0}\binom{x-1}{1}\\
		   &= (q-1)(n-x) - (x-1)
\end{align*}
\begin{align*}
	\sum_{s=0}^1K_s(x) &= \left[1\right] + \left[(q-1)\binom{x}{0}\binom{n-x}{1} - \binom{x}{1}\binom{n-x}{0}\right]\\
					  &= 1 + (q-1)(n-x)-x\\
					  &= (q-1)(n-x)-(x-1)
\end{align*}

So when $t=1$, it is the case that $L_t(x)=\sum_{s=0}^tK_s(x)$, as desired.


\section{} % Section 2
Let $C$ be such a code, and let $w\in C$.
If $w$ is non-constant, then it produces at least two distinct cyclic shifts: $w$ itself, and $w$ shifted left by one.
We can view these cyclic shifts of $w$ as a cyclic group generated by the left shift operation.
For $w=w_1\cdots w_p$, consider the list of its cyclic shifts:\\
\begin{align*}
	w_1&\cdots w_{p-1}w_p\\
	w_2&\cdots w_pw_1\\
	   &\;\;\vdots\\
w_{p-1}&\cdots w_1w_2\\
\end{align*}
For any $w$, any repetitions in this list would imply the existence of a proper subgroup of the above mentioned cyclic group.
The above cyclic group has prime order, so its only subgroups are itself and the trivial subgroup, so all non-constant words must have full order $p$.
Each set of $p$ cyclic shifts in $C$ does not change $|C|\;(\text{mod } p)$, so we need only consider the remaining constant words in $C$.
$C$ is in particular a linear code, so if one nonzero constant word is in $C$, then all of them are, in which case $|C|\equiv q\;(\text{mod }p)$.
Otherwise, the only constant word is 0, in which case $|C|\equiv 1\;(\text{mod }p)$.


\section{} % Section 3
We know that if $C=\left\langle g(x)\right\rangle$ has length $n$, then $g(x)$ divides $x^n-1$.
$x^7+x+1$ divides $x^{127}-1$, and in fact one can verify that $x^7+x+1$ does not divide $x^i-1$ for any $i$ less than 127, so the smallest length binary code with generator polynomial $x^7+x+1$ has length 127.


\section{} % Section 4
\subsection{} % Section 4.a
$\dim C=n-\deg(g)=11-5=6$, so
$G=\begin{pmatrix}
	g(x)\\
	xg(x)\\
	x^2g(x)\\
	x^3g(x)\\
	x^4g(x)\\
	x^5g(x)
\end{pmatrix}$.
\newline
\newline
\newline
We get
\setcounter{MaxMatrixCols}{11}
$G=\begin{pmatrix}
	0 & 0 & 0 & 0 & 0 & 1 & 0 & 2 & 1 & 2 & 2\\
	0 & 0 & 0 & 0 & 1 & 0 & 2 & 1 & 2 & 2 & 0\\
	0 & 0 & 0 & 1 & 0 & 2 & 1 & 2 & 2 & 0 & 0\\
	0 & 0 & 1 & 0 & 2 & 1 & 2 & 2 & 0 & 0 & 0\\
	0 & 1 & 0 & 2 & 1 & 2 & 2 & 0 & 0 & 0 & 0\\
	1 & 0 & 2 & 1 & 2 & 2 & 0 & 0 & 0 & 0 & 0\\
\end{pmatrix}$


\subsection{} % Section 4.b
We know that if $C$ is a nontrivial linear cyclic code with generator polynomial $g(x)$, then $C^\perp$ is also a linear cyclic code with generator polynomial $g^*(x)$.
\newline
\newline
So the generator polynomial for $C^\perp$ is
\[g^*(x) = 1 + 2x^2 + x^3 + 2x^4 + 2x^5\]
From here, we know that if the generator polynomial is $g(x)$, then the check polynomial is $h(x)=\frac{x^n-1}{g(x)}$, so the check polynomial for $C^\perp$ is
\begin{align*}
	h(x) &= \frac{x^{11}-1}{g^*(x)}\\
		 &= \frac{x^{11}-1}{1 + 2x^2 + x^3 + 2x^4 + 2x^5}\\\\
		 &= 2x^6 + x^5 + x^4 + x^3 + 2x^2 + 2
\end{align*}


\section{} % Section 5
In this case, $q=2$ and $r=3$, so $n=q^r-1=2^3-1=7$, and $2\le d\le7$.
To find $\beta$, a primitive element of $\F_8$, we can simply take $\F_8\cong\F_2\left[x\right]/\langle x^3+x+1\rangle$ and let $\beta^3+\beta+1=0$.
\newline
\newline
We now find the minimal polynomials of $\beta,\ldots,\beta^7$:
\begin{align*}
	m_\beta(x) &= x^3+x+1\\
	m_{\beta^2}(x) &= x^3+x+1\\
	m_{\beta^3}(x) &= x^3+x^2+1\\
	m_{\beta^4}(x) &= x^3+x+1\\
	m_{\beta^5}(x) &= x^3+x^2+1\\
	m_{\beta^6}(x) &= x^3+x^2+1\\
	m_{\beta^7}(x) &= x+1\\
\end{align*}
\newline
\newline
Since $C=\left\langle g(x)\right\rangle$, where $g(x)=\lcm(m_\beta(x),\ldots,m_{\beta^{d-1}}(x))$, we can now find $C$ for each possible distance $d$.
\newline
\newline
For $d=2$, we have $C=\left\langle x^3+x+1\right\rangle$.
For $d=3,4,5,6$, we have $C=\left\langle (x^3+x+1)(x^3+x^2+1)\right\rangle$.
For $d=7$, we have $C=\left\langle(x^3+x+1)(x^3+x^2+1)(x+1)\right\rangle$.


\end{document}
