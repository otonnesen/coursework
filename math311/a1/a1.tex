\documentclass[11pt]{article}
\usepackage{fancyhdr}
\pagestyle{fancy}
\newcommand\course{MATH 311}
\newcommand\hwnumber{1}
\newcommand\duedate{September 24, 2019}

\lhead{Oliver Tonnesen\\V00885732}
\chead{\textbf{\Large Assignment \hwnumber}}
\rhead{\course\\\duedate}


\usepackage{amsmath,amsfonts}


\DeclareMathOperator{\Span}{Span}


\begin{document}


\renewcommand{\thesubsection}{\thesection.\alph{subsection}}


\section{} % Section 1
\subsection{} % Section 1.a
\begin{align*}
	F^3=\{(0,0,0),(0,0,1),(0,1,0),(0,1,1),(1,0,0),(1,0,1),(1,1,0),(1,1,1)\}
\end{align*}
In general, $F^n$ has $2^n$ elements.


\subsection{} % Section 1.b
If $(x,y,z)$ is a general vector in $\Span(S)$, then we have
$a(1,1,0)+b(1,0,1)+c(0,1,1)=(x,y,z)$ for some $a,b,c\in F$, not all zero. So we
have $x=a+b$, $y=a+c$, and $z=b+c$:
\begin{align*}
	&z=b+c\\
	&b=x-a\\
	&c=y-a\\
	\implies &z=(x-a)+(y-a)\\
	\implies &2a=x+y-z\\
	\implies &0=x+y-z\qquad\qquad\text{Since $F=\mathbb{Z}_2$}
\end{align*}
So $S$ does not span $F^3$, and $x+y-z=0$ is a non-trivial condition
for a vector $(x,y,z)\in F^3$ to be in $S$.


\subsection{} % Section 1.c
$\Span(S)=\{(0,0,0),(1,1,0),(1,0,1),(0,1,1)\}$


\section{} % Section 2
Consider the vector $1\in P_3(\mathbb{R})$. We show that there do not exist
$a,b,c\in\mathbb{R}$ such that $a(x^3-2x^2+1)+b(4x^2-x+3)+c(3x-2)=1$. We
have:
\begin{align}
	a&=0\\
	-2a+4b&=0\\
	-b+3c&=0\\
	a+3b-2c&=1
\end{align}
If we take $(2)+2\cdot(1)$, we have $4b=0$, so $b=0$. Now we can plug in $b=0$
to $-b+3c=0$, obtaining $3c=0$, or $c=0$. So $a=b=c=0$, but $a+3b-2c=1$, so
$0=1$, thus there exist no such $a,b,c$, and 1 is not in the span of the three
given vectors, and therefore they do not span $P_3(\mathbb{R})$.


\section{} % Section 3
We prove of $S$ that it satisfies the three conditions sufficient to show that
it is a subspace: (1): $0\in S$, (2): $v,w\in S\implies(v+w)\in S$, and
(3): $a\in F,v\in S\implies av\in S$. (For clarity, we denote $F$'s identity
by 0, and $V$'s by $\vec0$ for the remainder of this proof.)
\begin{align*}
	&\text{(1): } \lambda f(v)=f(\lambda v)\implies 0f(v)=f(0v)=f(\vec 0)=0\text{, so $\vec0\in S$.}\\
	&\text{(2): let $v,w\in S$. } f(v+w)=f(v)+f(w)=0+0=0\text{, so $v+w\in S$.}\\
	&\text{(3): let $v\in S,a\in F$. } f(av)=af(v)=a0=0\text{, so $av\in S$.}
\end{align*}
Thus $S$ is a subspace of $V$.


\section{} % Section 4
$(\Longrightarrow)$: Suppose for a contradiction that $\{u+v,u+w,v+w\}$ is
linearly dependent. Then $u+v=a(u+w)+b(v+w)$ for some $a,b\in F$ not both zero.
Then we have $(1-a)u+(1-b)v=(a+b)w$.
\newline
\newline
\underline{Case 1 ($a+b=0$)}: $a+b=0$, so $a=-b$. $F$ is not of characteristic
2, so if $a=1$, then $b\neq1$. Thus $(1-a)u+(1-b)v\neq0$. If
$(1-a)\neq0\neq(1-b)$, then we have $v=\frac{1-a}{1-b}u$, a contradiction since
$\{u,v,w\}$ is linearly independent. If one of $a$ or $b$ is 1, say WLOG a=1,
then we have $(1-b)u=0$, but $1-b\neq0$, so $u=0$, a contradiction since
$\{u,v,w\}$ is linearly independent.
\newline
\newline
\underline{Case 2 ($a+b\neq0$)}: $a+b\neq0$, so
$w=\frac{1-a}{a+b}u+\frac{1-b}{a+b}$. Again, not both of $1-a$ and $1-b$ can be
0, so we have a nontrivial linear combination, a contradiction since
$\{u,v,w\}$ is linearly independent.
\newline
\newline
$(\Longleftarrow)$: Suppose for a contradiction that $\{u,v,w\}$ is linearly
dependent. Then $u=av+bw$ for some $a,b\in F$ not both 0. So
$\{u+v,u+w,v+w\}=\{av+bw+v,av+bw+w,v+w\}=\{(a+1)v+bw,av+(b+1)w,v+w\}$. Each
vector in the set is a linear combination of $v$ and $w$, so it is clearly
linearly dependent, a contradiction since $\{u+v,u+w,v+w\}$ is linearly
dependent.
\newline
\newline
Thus we've shown both the forward and backward direction, and so $\{u,v,w\}$
is linearly independent if and only if $\{u+v,u+w,v+w\}$ is, as desired.


\end{document}
