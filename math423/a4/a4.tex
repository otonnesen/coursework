\documentclass[11pt]{article}
\usepackage{fancyhdr}
\pagestyle{fancy}
\newcommand\course{MATH 423}
\newcommand\hwnumber{4}
\newcommand\duedate{November 7, 2019}

\lhead{Oliver Tonnesen\\V00885732}
\chead{\textbf{\Large Assignment \hwnumber}}
\rhead{\course\\\duedate}


\usepackage{amsmath,mathtools}


\DeclarePairedDelimiter\abs{\lvert}{\rvert}%
\makeatletter
\let\oldabs\abs
\def\abs{\@ifstar{\oldabs}{\oldabs*}}


\begin{document}


\renewcommand{\thesubsection}{\thesection.\roman{subsection}}


\section{} % Section 1
We know that $\delta(G)<n(G)$, and if $\delta(G)=n(G)-1$, then $G=K_{n(G)}$,
so $\kappa(G)$ is defined to be $n(G)-1$. So we show that if
$\delta(G)=n(G)-2$, then $\kappa(G)=n(G)-2$:

Assume not for a contradiction that $\kappa(G)<n(G)-2$. Then there exists
$S\subseteq V(G)$ with $\abs{S}<n(G)-2$ such that $G-S$ is disconnected. For
any $v\in V(G)$, there is at most one vertex $u\in V(G)$ not adjacent to $v$
(not that $u$ is adjacent to every vertex other than $v$, otherwise $deg(u)\le
n(G)-3<\delta(G)$). So $G$ can be obtained by removing at most one edge
incident to any vertex in $K_{n(G)}$. Consider $v,u\in V(G)$, $v\not\sim u$.
To disconnect $v$ and $u$, we must remove all $n(G)-2$ edges incident to $v$
or $u$, and thus we must remove exactly $n(G)-2$ vertices (all vertices other
than $v$ and $u$). So $\kappa(G)=n(G)-2$.


\section{} % Section 2
$(\Longrightarrow)$: Let $F$ be the cut $[S,\overline{S}]$. For any cycle in
$G$, if we follow it starting from some vertex $v$ in $S$, whenever it crosses
to $\overline{S}$, it must cross back in order to return to $v$, so $F$
contains an even number of the cycle's edges.
\newline
\newline
$(\Longleftarrow)$: Take $G-F$ and contract each of its connected components
to a single vertex, and connect any two vertices whose original components
shared an edge in $F$. Call this new graph $G'$. We show that $G'$ is
bipartite:

Let $C$ be an arbitrary cycle in $G'$, $v$ a vertex in $C$. $v$ is incident to
two edges in $F$, $e_1$ and $e_2$. If $e_1$ and $e_2$ are incident to $u_1$
and $u_2$ in $G$, respectively, then (since $u_1$ and $u_2$ are in the same
component) there is a path of the form $e_1u_1\sim\ldots\sim u_2e_2$ in $G$.
But any cycle in $G$ has an even number of its edges in $F$, so the same must
be true of $C$ in $G'$. Since $G'$ only has edges from $F$, $C$ must then be
an even cycle, and so $G'$ is bipartite. So let $S$ be the set of vertices in
those components of $G-F$ corresponding to the vertices in one of $G'$'s
partite sets. Then $[S,\overline{S}]$ is an edge cut corresponding to $F$.


\section{} % Section 3
If $G-v$ is 2-connected, then we're done. So assume $G-v$ is not 2-connected.
Then $G-v$ has a cut vertex, say $c$. Let $C$ be a block of $G-v$ containing
$c$. $v$ has a neighbor in $C$, since if it didn't, $c$ would still be a cut
vertex, and so $G$ would not be 2-connected. If $v$'s neighbor in $C$ was $c$,
then $c$ would again still be a cut vertex. So $v$ has a neighbor in $C-c$,
say $u$. $C$ is a block, and thus would not be disconnected by the removal of
$u$. So $u$ is not a cut vertex in $G-v$, and so $G-v-u$ is connected, as
desired.


\section{} % Section 4
\subsection{} % Section 4.i
We decompose $G$ into ears. If the last ear added to make $G$ has more than
one edge, then it contains a vertex of degree 2, and we're done. If the last
ear has only one edge, then $G-e$ has an ear decomposition, and thus by
Whitney's theorem is 2-connected. But $G-e$ is not 2-connected by assumption,
so the last ear added must have a vertex of degree 2. No vertex added in any
previous ear can have degree less than 2, so $\delta(G)=2$, as desired.


\subsection{} % Section 4.ii
[Induction on $n(G)$]: The only such graph on 4 vertices is $C_4$, for which
the claim clearly holds.

Assume the claim holds for all $n(G)\le n$. Let $G$ be a graph on $n+1$
vertices. Consider an ear decomposition $P_0,\ldots,P_k$ of $G$. If $k=0$,
then $G=C_{n+1}$, and $m(G)=n+1$. $n+1\le2(n+1)-4$ is true for $n>4$, so we're
done. Otherwise, if $k>0$, then remove $P_k$ from $G$. If $P_k$ has $l$
vertices, then it has $l+1$ edges. $G-P_k$ is minimally 2-connected, since it
is 2-connected, and if $G-P_k-e$ was 2-connected, then $G-e$ would be
2-connected (removing $P_k$ doesn't make $G$ ``more'' connected). So by
the induction hypothesis, the claim holds for $G-P_k$. That is,
$m(G-P_k)\le2n(G-P_k)-4$.  We also have that $m(G-P_k)=m(G)-(l+1)$ and
$n(G-P_k)=n(G)-l$. Together this gives us:
\begin{align*}
	m(G)-(l+1)&\le2(n(G)-l)-4\\
	m(G)-l-1&\le2n(G)-2l-4\\
	m(G)&\le2n(G)-l-3
\end{align*}
As we saw in question 4.ii), $\delta(G)=2$, so if we simply use a
decomposition for $G$ whose final ear $P_k$ is just a vertex of degree 2, then
we get $l=1$, giving us $m(G)\le2n(G)-l-3=2n(G)-4$. Thus by induction the
claim holds.


\section{} % Section 5
\subsection{} % Section 5.i
We prove the contrapositive: if $k>5$, then there exists a pair of
nonintersecting odd cycles. Any colouring of $G$ has at least 6 colours.
Consider the vertices of $G$ coloured 4, 5, or 6. If the subgraph induced by
these vertices was 2-colourable, then that colouring along with the original
colouring for the vertices coloured 1, 2, or 3 would be a 5-colouring for $G$,
so the subgraph is not 2-colourable, hence not bipartite, and so it contains
an odd cycle. No vertex in this set is coloured 1, 2, or 3, so in particular
no vertex in the odd cycle is coloured 1, 2, or 3. We can use the same
argument to show the subgraph induced by the set of vertices coloured 1, 2, or
3 contains an odd cycle. Since these two sets of vertices are disjoint, the
two odd cycles are disjoint, as desired.


\subsection{} % Section 5.ii
Consider some $k$-colouring of $G$. Let $C_i$, $C_j$ be the subgraphs of $G$
containing those vertices coloured $i$ and $j$, respectively The graph induced
by $C_i$ and $C_j$ is bipartite, since no two vertices sharing a colour are
adjacent. No vertex in this graph has degree $3$ or more, otherwise it would
contain a claw (and hence so would $G$). So any vertex in this graph has
degree 1 or 2, and is thus either in a path or an even cycle. If
the number of vertices in $C_i$ and $C_j$ differ by more than one, then there
exist more odd paths beginning and ending in WLOG $C_i$ than there are
beginning and ending in $C_j$. If we recolour them such that the number of odd
paths beginning and ending in $C_i$ and $C_j$ differs by at most one (by
choosing a number of odd paths and simply colouring all vertices coloured $i$
with $j$ and colouring all vertices coloured $j$ with $i$), then the number of
vertices coloured $i$ and $j$ in the new colouring will differ by at most one.
This recolouring does not affect any other pairs of colours, and so this
process can be done between all pairs of colour classes to obtain a colouring
in which the size of any two colour classes differs by at most one, as desired.


\section{Bonus} % Section 6
Let $v_1,\ldots,v_n$ be some ordering of the vertices of $G$, and let $k$ be
the number of colours used in a greedy colouring of $G$. We want to find a
clique of size $k$ in $G$. We know $G$ has a clique, and that it has highest
colour $k$. So let $C$ be a maximum clique in $G$, and let its vertices be
$\{c_1,\ldots,c_l\}$. We show that $l=k$. If $l<k$, then since $c_l$ has
colour $k$, every vertex in $C$ must be adjacent to a vertex with colour
$k-l-1$. If each vertex in $C$ were adjacent to the same vertex of this
colour, we could add it to $C$ to get a larger clique, but $C$ is maximum so
no such vertex exists. Let $a$ have colour $k-l-1$ be adjacent to as many
vertices in $C$ as possible, and let $c_i$ be some vertex in $C$ that is not
adjacent to $a$. $c_i$ is adjacent to some vertex with colour $k-l-1$, say
$b$. Note that $a$ and $b$ are coloured the same, and hence are nonadjacent.
Neither $a$ nor $b$ are adjacent to all vertices in $C$, but $a$ is adjacent
to more vertices in $C$ than $b$ is, so choose some vertex in $C$ that is
adjacent to $a$ but not $b$ and that is not $c_i$, call it $c_j$. Then we have
four vertices: $a,c_j,c_i,b$. $a$ and $b$ are the same colour and thus not
adjacent, $c_i$ was chosen to be not adjacent to $a$, $c_j$ was chosen to be
not adjacent to $b$, and $a\sim c_j\sim c_i\sim b$, so $G$ contains $P_4$ as
an induced subgraph, a contradiction, and so our assumption that $l<k$ was
false, so $l=k$, and thus the largest clique in $G$ has the same size as the
number of colours used in a greedy colouring of, hence the greedy colouring
was optimal.


\end{document}
