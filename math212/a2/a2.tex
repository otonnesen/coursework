\documentclass[12pt]{article}
\usepackage{fancyhdr}
\usepackage{blindtext}
\usepackage[latin1]{inputenc}
\usepackage{amsmath}
\usepackage{amsfonts}
\usepackage{amssymb}
\usepackage{graphicx}
\usepackage{mathtools}
\setlength{\headheight}{55 pt}
\pagestyle{fancy}
\renewcommand{\sectionmark}[1]{\markright{#1}{}}
\fancyhead{}
\fancyfoot{}
\fancyhead[R]{Oliver Tonnensen\\Ashley Van Spankeren\\Selma Yazganoglu\\Colton Broughton}
\fancyhead[L]{Due: 05/02/2019}
\fancyfoot[C]{\thepage}
\fancyhead[C]{\textbf{M212 A2}}
\renewcommand{\headrulewidth}{0.1pt}
\renewcommand{\footrulewidth}{0pt}
\AtBeginDocument{\fancyhfoffset{0pt}}
\def\QEDmark{\ensuremath{\square}}
\def\proof{\paragraph{Proof:}}
\def\endproof{\hfill\QEDmark}
\begin{document}
	\subsection*{Question 1:}
	\begin{proof}
		Let our relation be:
		\[g\sim h \coloneqq g=khk^{-1}\]
		To prove this is an equivalence relation we need it to have the reflexivity, symmetry, and transitivity properties.
		\begin{enumerate}	
			
			\item [Reflexivity:]
			Since $\mathcal{G}$ is a group there is an identity, denoted $e$. For reflexivity let $k=e$ so:
			\[g=ege^{-1}=g\] as required.
			\item [Symmetry:]
			Need to show if $g\sim h$ then $h\sim g$\\ \\
			So $g=khk^{-1}\rightarrow h=pgp^{-1}$ taking:
			\[g=khk^{-1} \]
			Multiplying both sides (on the right) by $k$ gives:
			$$gk=kh$$
			doing the same on the left with $k^{-1}$ gives:
			$$k^{-1}gk=h$$ since $k^{-1},k\in\mathcal{G}$ then $p=k^{-1}$ and $p^{-1}=k$ and so it is symmetric.
			\item [Transitivity:] We need to show if $g\sim h$ and $h\sim j$ then $g\sim j$\\ \\
			Let $g=khk^{-1}$ and $h=pjp^{-1}$ where $g,k,h,p,j\in\mathcal{G}$\\
			So then $g=k(h)k^{-1}$ and since we know what $h$ is we sub it in to give $g=k(pjp^{-1})k^{-1}$. Since $\mathcal{G}$ is associative we use bracketing mastery to give $g=(kp)j(p^{-1}k^{-1})$ since $k,p\in\mathcal{G}$ their inverses are as well, thus. 
			$$g\sim j$$
			as required.
		\end{enumerate}
		Equivalence class of the identity is j conjugate to e such that $j=kek^{-1}$
		but this is simply $j=e$ since this is an equivalence relation and is symmetric the obverse is true as well. If $\mathcal{G}$ is abelian the relation is simply the traditional $"="$
	\end{proof}
\newpage
	\setlength{\headheight}{5pt}
	\fancyhead[R]{}
	\subsection*{Question 2:}
	\begin{proof}
		\begin{enumerate}
			\item[($\Rightarrow$)] If $\mathcal{G}$ is abelian then for all elements $g,h\in\mathcal{G}$ there exists an inverse denoted by $g^{-1}$ and $h^{-1}$. By proposition 1.2.9:
			$$(gh)^{-1}=h^{-1}g^{-1}$$
			Since $\mathcal{G}$ is abelian we can commute the right hand side giving:
			$$(gh)^{-1}=h^{-1}g^{-1}=g^{-1}h^{-1}$$
			as required.
			\item[$(\Leftarrow)$] By proposition 1.2.9 we have:
			$$(gh)^{-1}=h^{-1}g^{-1}$$
			similarly we have:
			$$(hg)^{-1}=g^{-1}h^{-1}$$
			if $(gh)^{-1}=g^{-1}h^{-1}=(hg)^{-1}$ then $\mathcal{G}$ is commutative. A commutative group is an abelian group.
		\end{enumerate}	
	\end{proof}
	\newpage
	\subsection*{Question 3:}
	ord(Z)/gcd(x,50) and solve for x in the case of gcd = 10 and gcd = 2
\newpage
	\subsection*{Question 4:}
	\begin{proof}
	%	For $\mathcal{G}$ to be finite order it means that for some $g$ in $\mathcal{G}$: $g^{ord\left(\mathcal{G}\right)}=e$\\
	%	We also know that for $g\ne e$  that if $\mathcal{G}$ is a group then every element has an inverse and so $gg^{-1}=e$. We know that $g^n =e =gg^{-1}$ Further for $g^{n-1}g^1 = g^n = e$ So either $g^{n-1}$ or $g^{1}$ is the inverse. But since both are in $\mathcal{G}$ and we need to show there exists at least one, we need not find which is the inverse.
	\end{proof}
	\newpage
	\subsection*{Question 5:}
	\begin{enumerate}
		\item 
	\end{enumerate}
	\newpage
	\subsection*{Question 6:}
	\begin{enumerate}
		\item TRUE. we defined the order as the least possible $k\in\mathbb{Z^+}$ such that for some $g$ in an arbitrary set $g^k=e$ for and identity denoted $e$, $e^1=e$ and so the order of $e$ is $1$
		\item FALSE. $gcd(k,6)=1$ for $0\le k\le 5$ is only true when $k=1,5$. when $k=2$ the gcd is 2, when $k=3$ the gcd is 3, and when $k=4$ the gcd is 2.
		\item FALSE for $n\ge3$ $S_n$ is not abelian:
		\begin{proof}
			define $\alpha$ as the permutation only switching 1 and 2, and $\beta$ as the permutation that only switches 1 and 3.Then:\\
			$\alpha\beta(1)=3$ and $\beta\alpha(1)=2$ and so it is not commutative and thus the group $S_n$ is not abelian. 	
		\end{proof}
	\end{enumerate}
	\newpage
	\subsection*{Question 7:}
	\begin{enumerate}
		\item[a] (1 9 6 5 7 8 3 4 2)
		\item[b]
		\item[c] 9
		\item[d] even
	\end{enumerate}
\end{document}