\documentclass{article}
\usepackage{amsmath,amsfonts,amssymb}
\usepackage[normalem]{ulem}
\title{MATH 212 Assignment 1}
\author{Colton Broughton\\
		Oliver Tonnesen\\
		Ashley Van Spankeren\\
		Selma Yazganoglu}
\date{February 25, 2019}
\begin{document}
\maketitle
\renewcommand{\thesubsection}{\thesection.\alph{subsection}}
\section{} % Section 1
$G:=\mathbb{Z}\oplus\mathbb{Z}=\{(0,0),(0,1),(1,0),(1,1)\}$\\
$\langle(0,0)\rangle=\{(0,0)\}$\\
$\langle(0,1)\rangle=\{(0,0),(0,1)\}$\\
$\langle(1,0)\rangle=\{(0,0),(1,0)\}$\\
$\langle(1,1)\rangle=\{(0,0),(1,1)\}$\\
So there exists no $g\in G$ with $\langle g\rangle=G$, and $G$ is therefore not
cyclic. The following are the subgroups of $G$:\\
$\{(0,0)\}$\\
$\{(0,0),(0,1)\}$\\
$\{(0,0),(1,0)\}$\\
$\{(0,0),(1,1)\}$\\
\newline
Note that each subgroup can be constructed by taking the union of the set
containing the identity -- $(0,0)$ -- and the set containing exactly one
element of $\mathbb{Z}\oplus\mathbb{Z}$. Additionally, the group generated by
$g\in G$ is exactly the vector space over $\{(x,y)\mid x,y\in GF(n)\}$ spanned
by $g$.
\section{} % Section 2
\begin{tabular}{c|cccc}
	* & a & b & c & d \\
	\hline
	a & c & d & a & b \\
	b & d & c & b & a \\
	c & a & b & c & d \\
	d & b & a & d & c \\
\end{tabular}
\newline
\newline
\newline
\underline{Nonempty}: True\\
\newline
\underline{Associative}:
\begin{align*}
	a(bd)=(ab)d=c\\
	b(cd)=(bc)d=a\\
	a(bc)=(ab)c=d\\
	a(cd)=(ac)d=b\\
\end{align*}
The table shows the group to be abelian, so this is true for all other
permutations.\\
\newline
\underline{Inverse}: Every element is its own inverse.\\
\newline
\underline{Binary operation}: All pairs of elements map to an element in
the set.\\
\section{} % Section 3
\subsection{} % Section 3.a
We know $h\neq e$ and $g\neq e$, so $gh$ can be neither $g$ nor $h$, and so has
to be $e$. Thus $gh=e=gg^{-1}$ and $h=g^{-1}$. The same can be said for $hg$,
and so $g=h^{-1}$. $hh$ cannot be $h$ since $h\neq e$, and $hh$ cannot be $e$
since $h^{-1}=g$, so $hh=g$. Similarly, $gg=h$. Thus we have completed the
binary operation table for $G$:
\newline
\newline
\begin{tabular}{c|ccc}
	& e & g & h \\
	\hline
	e & e & g & h \\
	g & g & h & e \\
	h & h & e & g \\
\end{tabular}
\newline
\newline
We can simply look at the table and see that it is symmetric about the diagonal,
and so $G$ is abelian.
\subsection{} % Section 3.b
Similarly to as in section 3.a, we can simply look at the binary operation
table and see that $\langle g\rangle=G$, and so $G$ is cyclic.
\section{} % Section 4
\subsection{} % Section 4.a
Recall \emph{Thorem 3.7.6}: $H\subseteq G$ is a subgroup of $G$ if and only if
$H\neq\emptyset$ and for all $h_1,h_2\in H$, $h_1^{-1}h_2\in H$.
\newline
\newline
We have $0\in H$, so $H\neq\emptyset$.\\
Suppose we have  $h_1=dk_1$, and $h_2=dk_2$ where $h_1,h_2\in d\mathbb{Z}$. We
know that $h_1^{-1}=-dk_1$, since $dk_1+(-dk_1)=e$. Thus,
$h_1^{-1}h_2=-dk_1+dk_2=d(k_2-k_1)$. $k_2-k_1\in\mathbb{Z}$, and so
$d(k_2-k_1)\in d\mathbb{Z}$. Thus we have satisfied both conditions of the
theorem, and so $H$ is a subset of $G$.
\subsection{} % Section 4.b
Suppose $H\subseteq\mathbb{Z}$ is a non-trivial subgroup of $\mathbb{Z}$. Let
$n\in H$ be the smallest positive element in $H$. $H$ is a group, and is
therefore closed under its binary operation. Thus $nk\in H$ for all
$k\in\mathbb{Z}$, and  $n\mathbb{Z}\subseteq H$.
Suppose for a contradiction that there exists an element in $H$ that is not of
the form $nk$, $k\in\mathbb{Z}$. By the division algorithm, there exist disinct
integers $q$ and $r$, $0\le r<n$ such that $m=qn+r$. Since $m\neq nk$ for all
$k\in\mathbb{Z}$, $r\neq0$. From $m=qn+r$, we have $m+(-qn)=r\in H$ since
$m,(-qn)\in H$. So $n>r\in H$, contradicting our initial supposition that $n$
is the smallest positive element in $H$.
\section{} % Section 5
\renewcommand{\thesubsection}{\thesection.\roman{subsection}}
\subsection{} % Section 5.i
True. Let $A$ be an abelian group. There exist $a,b\in A$ such that $ab=ba$.
For a subgroup $B$ of $A$, $a_B b_B=b_B a_B$, since all $a,b\in A$ commute.
\subsection{} % Section 5.ii
True. Let $\langle g\rangle=G$ for some $g\in G$. Then for any $g'\in G$, there
exists some $k\in\mathbb{Z}$ such that $g'=g^k$. Similarly, if $H\subseteq G$
is a subgroup of $G$, then for any $h\in H$, there exists some $m\in\mathbb{Z}$
such that $h=g^m$.
\newline
Let $n$ be the least positive integer such that $g^n\in H$. We wish to show
that $n\mid m$. That is, we wish to show that every $m$ can be written as $nq$
for some $q\in\mathbb{Z}$, and by extension, every $g^m$ can be written as
$(g^n)^q$. If this is the case, then $\langle g^n\rangle=H$.
\newline
By the division algorithm, we know that there exist distinct integers $q$ and
$r$, $0\le r<n$, such that $m=nq+r$. So
\begin{align*}
	g^m&=(g^n)^q\cdot g^r\\
	(g^n)^{-q}\cdot g^m&=(g^n)^{-q}\cdot(g^n)^q\cdot g^r\\
	(g^n)^{-q}\cdot g^m&=g^r\\
\end{align*}
By definition, $g^n,g^m\in H$, and so $(g^n)^q\cdot g^m=g^r\in H$. Recall
that $n$ was defined to be the smallest positive integer such that $g^n\in H$,
but $0\le r<n$. So $r=0$, and therefore it is the case that $n\mid m$, and so
$\langle g^n\rangle=H$, and $H$ is cyclic.
\subsection{} % Section 5.iii
False. The trivial subgroup containing only the identity is always abelian.
\subsection{} % Section 5.iv
False. The trivial subgroup containing only the identity is always cyclic.
\subsection{} % Section 5.v
True. Let $h,m\in G$, a cyclic group. We know there exists $g\in G$ such that
$\langle g\rangle=G$, and so $h=g^k$ and $m=g^l$ for some $k,l\in\mathbb{Z}$.
$hm=g^kg^l=g^{k+l}=g^{l+k}=g^lg^k=mh$.
\subsection{} % Section 5.vi
False. $(\mathbb{R},+)$ is abelian but not cyclic.
\end{document}
