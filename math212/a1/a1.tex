\documentclass{article}
\usepackage{amsmath,amsfonts,graphicx}
\usepackage[normalem]{ulem}
\title{MATH 212 Assignment 1}
\author{Colton Broughton\\
		Oliver Tonnesen\\
		Ashley Van Spankeren\\
		Selma Yazganoglu}
\date{January 22, 2019}
\begin{document}
\maketitle
\newcommand{\msout}[1]{\text{\sout{\ensuremath{#1}}}}
\newcommand{\heart}{\ensuremath\heartsuit}
\renewcommand{\thesubsection}{\thesection.\alph{subsection}}
\section{} % Section 1
\subsection{} % Section 1.a
No. Suppose $a=b\neq0$. Then $a-b=0$. But $0\not\in X$, so * is not a binary
operation on $X$.
\subsection{} % Section 1.b
Yes. Let $A\subseteq B\subseteq\mathcal{P}(\{1,2,3\})$. $A\cap B$ is defined
for all values of $A$ and $B$, and $A\cap B\in\mathcal{P}(\{1,2,3\})$ for all $A,B$.
\subsection{} % Section 1.c
No. Let $A=\{1\},B=\{1\}$. $A\times B=\{(1,1)\}$.
$\{(1,1)\}\not\in\mathcal{P}(\{1,2,3\})$, so $\times$ is not a binary operation.
\newline
In general, one must check that the range of * is contained in $X$, that * is
defined on all $x\in X\times X$, and that no input $x\in X\times X$ exists such
that it corresponds to two outputs.
\section{} % Section 2
Associativity:\\
Let $a=b=c=2\in\mathbb{Z}$. We show that $(a*b)*c\neq a*(b*c)$:
\begin{align*}
	(2*2)*2=(2\cdot2^2)\cdot2^2=32\\
	2*(2*2)=2\cdot(2\cdot2^2)^2=128
\end{align*}
So * is not associative.
\newline
\newline
Commutativity:\\
Let $a=2,b=3,a,b\in\mathbb{Z}$. We show that $a*b\neq b*a$.
\begin{align*}
	a*b=2\cdot3^2=18\\
	b*a=3\cdot2^2=12
\end{align*}
So * is not commutative.
\newline
\newline
Identity:\\
Suppose $e$ is an identity for * on $\mathbb{Z}$. Then $e*a=a=a*e$.
Then $e\cdot a^2=a=a\cdot e^2$. Consider the equation $e\cdot a^2=a$ where
$a\neq0$:
\begin{align*}
	&e\cdot a^2=a\\
	\iff&e\cdot a=1
\end{align*}
This clearly cannot hold for all $0\neq a\in\mathbb{Z}$, so $e$ cannot exist.
\section{} % Section 3
Symmetric:\\
All of $a,b,c,d\in\mathbb{Z}$, so then both of $ad,bc\in\mathbb{Z}$. By the
properties of = on $\mathbb{Z}$, we know that $ad=bc\implies bc=ad$, so
$((a,b),(c,d))\in\sim\iff((c,d),(a,b))\in\sim$.
\newline
\newline
Reflexive:\\
Similarly, we know that = on $\mathbb{Z}$ is reflexive, so
$ab=ab$ for all $(a,b)\in\mathbb{Z}\times\mathbb{Z}\setminus\{0\}$.
\newline
\newline
Transitive:\\
For all $(a,b),(c,d),(e,f)\in B$, suppose $ad=bc$ and $ef=de$.
Suppose $c\neq0$.
\begin{align*}
	ad=bc\\
	adcf=bcde&\qquad\text{($cf=de$)}\\
	a\msout{dc}f=b\msout{cd}e&\qquad\text{($c,d\neq0$)}\\
	af=be
\end{align*}
Now suppose $c=0$. Then $bc=cf=0$, and consequently, $ad=de=0$. $d\neq0$, so
$a=e=0$, so $af=be=0$. Thus, transitivity holds both when $c=0$ and when
$c\neq0$.
\section{} % Section 4
\subsection{} % Section 4.a
For $+$, $x$ is the identity, since $x+x=x$, and $x+y=y=y+x$, so $x:=0$.\\
For $\cdot$, $y$ is the identity, since $y\cdot y=y$ and $y\cdot x=x=x\cdot y$,
so $y:=1$.
\subsection{} % Section 4.b
For every $a\in\mathbb{Z}$, exactly one of the following holds: $a\in\mathbb{Z}^+$,
$-a\in\mathbb{Z}^+$, $a=0$. $x$ is defined to be in the set of positive integers,
but is also the additive identity. This contradicts the definition of $\mathbb{Z}$,
and so $X\neq\mathbb{Z}$.
\section{} % Section 5
\subsection{} % Section 5.a
No. $b*a=a\neq b=a*b$.
\subsection{} % Section 5.b
Yes.
\newline
\newline
\begin{tabular}{c|ccc}
	\# & a & b & c \\
	\hline
	a & a & b & b \\
	b & b & c & b \\
	c & b & b & c \\
\end{tabular}
\subsection{} % Section 5.c
Yes.
\newline
\newline
\begin{tabular}{c|ccc}
	* & a & b & c \\
	\hline
	a & a & b & a \\
	b & a & c & b \\
	c & a & b & c \\
\end{tabular}
\subsection{} % Section 5.d
No. None of $a,b,c$ can be an identity for \#:\\
$a$: $a\#c=b\neq c$\\
$b$: $b\#b=c\neq b$\\
$c$: $a\#c=b\neq a$\\
\subsection{} % Section 5.e
\heart\, is commutative on $Y$ if and only if the table representing \heart\, is
symmetric about the main diagonal. In other words, if the entry at row $x$ and
column $y$ is the same as that at row $y$ and column $x$, $\forall x,y\in Y$.
\newline
\newline
$c\in Y$ is \heart's identity if and only if its row and column exactly match
the order of the elements along the top and left of the table, respectively. In
other words,
\newline
\newline
\begin{tabular}{c|ccccc}
	\heart & a & b & c & d & e \\
	\hline
	a & $\boldsymbol\cdot$ && a && $\boldsymbol\cdot$  \\
	b && $\boldsymbol\cdot$ & b & $\boldsymbol\cdot$ & \\
	c & a & b & c &d & e \\
	d && $\boldsymbol\cdot$ & d & $\boldsymbol\cdot$ & \\
	e & $\boldsymbol\cdot$ && e && $\boldsymbol\cdot$ \\
\end{tabular}
\end{document}
