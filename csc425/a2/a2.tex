\documentclass[11pt]{article}

\usepackage[margin=1.5in]{geometry}

\usepackage{fancyhdr}
\pagestyle{fancy}
\newcommand\course{CSC 425}
\newcommand\hwnumber{2}
\newcommand\duedate{October 16, 2020}

\lhead{Oliver Tonnesen\\V00885732}
\chead{\textbf{\Large Assignment \hwnumber}}
\rhead{\course\\\duedate}


\usepackage{tikz}


\usepackage{amsmath, amssymb, mathtools}

\DeclarePairedDelimiter\abs{\lvert}{\rvert}%
\makeatletter
\let\oldabs\abs
\def\abs{\@ifstar{\oldabs}{\oldabs*}}

\usepackage{algorithm}
\usepackage[noend]{algpseudocode}

\newcommand{\ngets}{
	\ifmmode
	\gets
	\else
	$\gets$
	\fi
}

\begin{document}
\section{} % Section 1
Given two sorted arrays, \emph{left} and \emph{right}, we can count the number of significant inversions between the two as follows:

\begin{algorithm}
\begin{algorithmic}

\newcommand{\varleft}{\emph{left}}
\newcommand{\varright}{\emph{right}}


\Function{Count-Sig-Invs}{\varleft, \varright}
\State $i,j,s \ngets 0,0,0$
\While{$i < \abs{\varleft}$ and $j < \abs{\varright}$}
	\If{$\varleft[i] > 2 \cdot \varright[j]$}
		\State $s \ngets s + \abs{left} - i$
		\State $j \ngets j + 1$
	\Else
		\State $i \ngets i + 1$
	\EndIf
\EndWhile
\State \Return $s$
\EndFunction
\end{algorithmic}
\end{algorithm}

Since the lists are sorted, if $a_i$ in the left list and $a_j$ in the right list form a significant inversion (that is, $a_i > 2a_j$), then since each of $a_{i+1}, \ldots, a_{\abs{\emph{left}}}$ are larger than $a_i$, each of them also forms a significant inversion with $a_j$.
If $(a_i, a_j)$ is a significant inversion, then we count all of $(a_i, a_j), \ldots, (a_{\abs{\emph{left}}}, a_j)$ at once, the nchoose a larger $j$, and thus a larger $a_j$ to see if it is still forms a significant inversion with $a_i$.
If $(a_i, a_j)$ is not a significant inversion, then we choose a larger $i$, and thus a larger $a_i$ to see if it forms a significant inversion with $a_j$.

We can plug the above function into a the merge step of a regular implementation of merge sort, roughly as follows:

\begin{algorithm}
\begin{algorithmic}

\newcommand{\varlist}{\emph{list}}
\newcommand{\varleft}{\emph{left}}
\newcommand{\varright}{\emph{right}}


\Function{Merge-Sort-Count-Invs}{\varlist}
\If{$\abs{\varlist} = 0$}
	\State \Return 0
\EndIf
\State Split list into \varleft{} and \varright{}.
\State invs$_\varleft$ \ngets \textproc{Merge-Sort-Count-Invs}(\varleft)
\State invs$_\varright$ \ngets \textproc{Merge-Sort-Count-Invs}(\varright)
\State invs$_\emph{cross}$ \ngets \textproc{Merge}(\varleft, \varright, \varlist)
\State \Return invs$_\varleft$ + invs$_\varright$ + invs$_\emph{cross}$
\EndFunction
\Function{Merge}{\varleft, \varright, \varlist}
\State invs \ngets \textproc{Count-Sig-Invs}(\varleft, \varright)
\State Continue merge sort as normal, merging \varleft{} and \varright{} into \varlist{} in place.
\State \Return invs
\EndFunction
\end{algorithmic}
\end{algorithm}


\section{} % Section 2
The only vertices connected to a given vertex $v$ are its parent and its two children.
Our goal is simply to find a vertex whose value is lower than those of its parent and its children.
Consider the following algorithm:
\begin{algorithm}
\begin{algorithmic}

\newcommand{\varroot}{\emph{root}}


\Function{Find-Local-Min}{\varroot}
\State $v \ngets \varroot$
\While{$v$ not a leaf}
	\State $l,r \ngets v_\textrm{left}$, $v_\textrm{right}$
	\If{$x_l < x_v$}
		\State $v \ngets l$
	\ElsIf{$x_r < x_v$}
		\State $v \ngets r$
	\Else
		\State \Return $v$
	\EndIf
\EndWhile
\EndFunction
\end{algorithmic}
\end{algorithm}


We claim the above algorithm returns a local minimum, and give the following proof:

Let $v_1, \ldots, v_k$ be the sequence of vertices taken on by the variable $v$ throughout the algorithm.
Note that $x_{v_1}, \ldots, x_{v_k}$ is necessarily a decreasing sequence by the nature of how the algorithm selects $v_{i+1}$ given $v_i$.
The loop in the algorithm only exits when $v$ is either a local minimum or a leaf.
If $v$ is a local minimum, then we're already done, so suppose $v_k$ is a leaf.
The only vertex joined to a leaf is that vertex's parent, in this case $v_{k-1}$.
Recall that $x_{v_1}, \ldots, x_{v_k}$ is a strictly decreasing sequence, so $x_{v_{k-1}} > x_{v_k}$.
In other words, $v_k$ is a local minimum, as desired.

\section{} % Section 3
Our goal is to find $i < j$ such that $p(j) - p(i)$ is maximized.

We can solve this problem using divide and conquer.
If the list has length 2, then we either return $i=0$ and $j=1$ if the price is lower on day $i$ than it is on day $j$, and $i=-1$ and $j=-1$ otherwise.
If the list has length greater than 2, then we begin by dividing the list into a left and right sublist.
We recursively find the two days giving the largest profit in either list, then we find the two days giving the largest profit when buying on a day in the left list and selling on a day in the right list.
For the latter step, we can simply perform a linear search on the left and right sublists to find the lowest and highest prices, respectively.
The difference in price between these two days is the largest profit between the two lists.

The following pseudocode corresponds to the above:


\begin{algorithm}[H]
\begin{algorithmic}

\newcommand{\varlist}{\emph{list}}
\newcommand{\varleft}{\emph{left}}
\newcommand{\varright}{\emph{right}}
\newcommand{\varmid}{\emph{mid}}
\newcommand{\varoffset}{\emph{offset}}


\Function{Find-Delta}{\varlist[, \varoffset$=0$]}
\If{$j - i = 1$}
	\If{$\varlist[i] < \varlist[j]$}
		\State \Return $i, j$
	\Else
		\State \Return $-1, -1$
	\EndIf
\EndIf
\State $\varleft, \varright \ngets \varlist$
\State $\varmid \ngets \abs{\varlist} / 2$
\State $\delta_\varleft, i_\varleft, j_\varleft \ngets \textproc{Find-Delta}(\varleft, \varoffset)$
\State $\delta_\varright, i_\varright, j_\varright \ngets \textproc{Find-Delta}(\varright, \varoffset + \varmid)$
\State $\delta_\emph{cross}, i_\emph{cross}, j_\emph{cross} \ngets \textproc{Merge}(\varleft, \varright, \varoffset, \varmid)$
\State \Return $\delta_x, i_x, j_x$ corresponding to $\max\{\delta_\varleft, \delta_\varright, \delta_\emph{cross}\}$
\EndFunction

\Function{Merge}{\varleft, \varright, \varoffset, \varmid}
\State $i$ \ngets index of smallest element in \varleft + \varoffset
\State $j$ \ngets index of largest element in \varright + \varoffset + \varmid
\State \Return $\varlist[j] - \varlist[i], i, j$
\EndFunction
\end{algorithmic}
\end{algorithm}

The \textproc{Merge} step runs in $O(n)$ time, and is called $O(\log n)$ times, so the above algorithm runs in $O(n \log n)$ time, as desired.


\section{} % Section 4
We give a solution using dynamic programming.
Given an index $i$, our sub-problem is to find the maximum weight independent set only using vertices from $v_1, \ldots, v_i$.

At index $i$, we have two options for $OPT(i)$, either:
\begin{enumerate}
	\item $v_i$ is part of the independent set, in which case $OPT(i) = w_i + OPT(i-2)$, or
	\item $v_i$ is not part of the independent set, in which case $OPT(i) = OPT(i-1)$.
\end{enumerate}

\noindent
More concisely,
\[OPT(i) = \begin{cases}
	0 & \textrm{if } i = 0\\
	\max \begin{cases}
		w_i + OPT(i-2)\\
		OPT(i-1)
	\end{cases} & \textrm{otherwise}
\end{cases}\]

Our goal is to find $OPT(n)$, and use it to reconstruct the set of vertices used to get it.
We give an algorithm to demonstrate this:

\begin{algorithm}[H]
\begin{algorithmic}
\Function{Max-Wt-Indep-Set}{$v_1, \ldots, v_n, w_1, \ldots, w_n$}
\State $M[0] = 0$
\State $\textproc{OPT}(n, M)$
\State \Return $\textproc{Reconstruct}(M)$
\EndFunction
\Function{OPT}{$n, M$}
\If{$M[i]$ uninitialized}
	\State $M[i] \ngets \max\{\textproc{OPT}(i-1,M), w_i + \textproc{OPT}(i-2,M)\}$
\EndIf
\State \Return $M[i]$
\EndFunction
\Function{Reconstruct}{$M$}
\State $U \ngets \emptyset$
\State $i \ngets n$
\While{$i > 0$}
	\If{$M[i] = M[i-1]$}
		\State $i \ngets i-1$
	\Else
		\State $U \ngets U \cup \{i\}$
		\State $i \ngets i-2$
	\EndIf
\EndWhile
\Return $U$
\EndFunction
\end{algorithmic}
\end{algorithm}


\section{} % Section 5
We view the sub-problems as follows:

Find the maximum possible profit when starting on day $i$ with the machine rebooted $j$ days ago.

The base case is when $i = n$, since we never want to reboot on the last day.
When $i = n$, the maximum possible profit is simply the minimum of $x_i$ and $s_j$.

In the case where $i \neq n$, the decision we must make when finding $OPT(i, j)$ is whether or not to reboot the machine.
If we do reboot it, then our profit will be $OPT(i+1, 1)$.
If we do not reboot it, then our profit will be $\min\{x_i, s_j\} + OPT(i+1, j+1)$

This gives us the following Bellman equation:
\[OPT(i, j) = \begin{cases}
	\min\{x_i, s_j\} & \textrm{if } i = n\\
	\max \begin{cases}
		OPT(i+1, 1)\\
		\min\{x_i, s_j\} + OPT(i+1, j+1)
	\end{cases} & \textrm{otherwise}
\end{cases}\]

Given this, we can easily construct a dynamic programming algorithm solving the problem:

\begin{algorithm}[H]
\begin{algorithmic}
\Function{Max-Profit}{$x_1, \ldots, x_n, s_1, \ldots, s_n$}
\For{$j = 1, \ldots, n$}
	\State $M[n][j] \ngets \min\{x_n, s_j\}$
\EndFor
\State $\textproc{OPT}(1, 1, M)$
\State \Return $\textproc{Reconstruct}(M)$
\EndFunction
\Function{OPT}{$i, j, M$}
\If{$M[i][j]$ uninitialized}
	\State $M[i][j] \ngets \max\{\textproc{OPT}(i+1, 1, M), \textproc{OPT}(i+1, j+1, M)\}$
\EndIf
\State \Return $M[i][j]$
\EndFunction
\Function{Reconstruct}{$M$}
\State $U \ngets \emptyset$
\State $i \ngets 1$
\State $j \ngets 1$
\While{$i \neq n$}
	\If{$M[i][j] = M[i+1][1]$}
		\State $i \ngets i + 1$
		\State $j \ngets 1$
		\State $U \ngets U \cup \{i\}$
	\Else
		\State $i \ngets i + 1$
		\State $j \ngets j + 1$
	\EndIf
\EndWhile
\State \Return $U$
\EndFunction
\end{algorithmic}
\end{algorithm}


\end{document}
