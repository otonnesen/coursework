\documentclass[11pt]{article}

\usepackage[margin=1.5in]{geometry}

\usepackage{fancyhdr}
\pagestyle{fancy}
\newcommand\course{CSC 425}
\newcommand\hwnumber{3}
\newcommand\duedate{November 16, 2020}

\lhead{Oliver Tonnesen\\V00885732}
\chead{\textbf{\Large Assignment \hwnumber}}
\rhead{\course\\\duedate}


\usepackage{tikz}


\usepackage{amsmath, amssymb, mathtools}

\DeclarePairedDelimiter\abs{\lvert}{\rvert}%
\makeatletter
\let\oldabs\abs
\def\abs{\@ifstar{\oldabs}{\oldabs*}}

\usepackage{algorithm}
\usepackage[noend]{algpseudocode}

\begin{document}
\section{} % Section 1
We construct a graph in polynomial time to which we can apply a network flow algorithm to solve the problem.

First calculate the distance between each pair of clients and stations.
Add a vertex for each client and each station.
Add an edge between any client-server pair within $r$ of each other, and give it capacity 1.
Finally add a source vertex adjacent to each client vertex with edge capacities of 1, and add a sink vertex adjacent to each station with edge capacities of $L$.

We run a network flow algorithm on this graph to obtain the value of its maximum flow, say $v$.
If $v$ is equal to the number of clients, then our algorithm returns true, otherwise false.


\section{} % Section 2
Our main idea is to construct an edge-capacited graph $H$ whose maximum flow (in the original sense) is the same as $G$'s maximum flow (in the vertex-capacited sense).

Roughly, we do this by ``doubling up'' all the vertices and transforming each vertex into an edge.
For each vertex $v$, we add a vertex $v'$ with an edge from $v$ to $v'$ with capacity $c_v$.
Any edge going to $v$ still do so, but any edges coming from $v$ now instead come from $v'$.
That is, if $(i,j)$ is an edge in $G$, it becomes $(i',j)$ in $H$.
The edges between $v$ and $v'$ account for all the bottlenecks in $G$, so we can give the remaining edges infinite capacity (or more practically, $\abs{v} \cdot \max\{c_v \vert v \in V\}$).

We define an $s-t$ cut in a vertex-capacited network analogously to an edge-capacited network: a partition of the edges into two sets.

Our cut-set analogue is the set of vertices with at least one edge in either partite set.

\textbf{Theorem}: \emph{Max-flow min-cut for vertex-capacited graphs}:
\[\max\{f \;\vert\; \textrm{$f$ a feasible flow}\} = \min\{c(S,T) \;\vert\; \textrm{$(S,T)$ an $s-t$ cut}\}\]
We know the original max-flow min-cut theorem holds for $H$.
Furthermore, a min-cut for $H$ will certainly not use any of the edges with infinite capacity, and the rest of $H$'s edges are in bijection with the vertices of $G$.
This means that a min-cut in $H$ can be translated into a min-cut for $G$.
This -- along with the fact that the max-flows in $G$ and $H$ have equal values -- proves the max-flow min-cut theorem for vertex-capacited graphs.


\section{} % Section 3
We give a reduction from a known NP-complete problem: $\textproc{Vertex Cover} \le_P \textproc{Hitting Set}$.

Let $(V, E, k)$ be an input to \textproc{Vertex Cover}.
Let $A = V$, and for each $(v_i, u_i) \in E$, let $B_i = \{v_i, u_i\}$.

We show that $(A, B_1, \ldots, B_m, k)$ is an accepting input to \textproc{Hitting Set} if and only if $(V, E, k)$ is an accepting input to \textproc{Vertex Cover}:
\newline

\noindent
$(\Longrightarrow)$:
Let $H$ be a hitting set of size at most $k$ for $B_1, \ldots, B_m$
Now consider the subset $C \subseteq V$ corresponding to $H$.
For any edge $e_i$ corresponding to $B_i$, $H \cap B_i \neq \emptyset$, so $C$ covers that edge.
Thus $C$ is a vertex cover with size at most $k$, and so $(V, E, k)$ is indeed an accepting input to \textproc{Vertex Cover}.
\newline

\noindent
$(\Longleftarrow)$:
Let $C \subseteq V$ be a vertex cover of size at most $k$ for $(V, E)$.
Now consider the subset $H \subseteq A$ corresponding to $C$.
For any subset $B_i$ corresponding to edge $e_i$, at least one endpoint of $e_i$ is in $C$, so at least one element of $B_i$ is in $H$.
Thus $H$ is a hitting set with size at most $K$, and so $(A, B_1, \ldots, B_m, k)$ is indeed an accepting input to \textproc{Hitting Set}.


\section{} % Section 4
We give a reduction from a known NP-complete problem: $\textproc{Independent Set} \le_P \textproc{Path Selection}$:

Let $G = (V, E)$ be an input to \textproc{Independent Set}.
We construct the input $(D, P_1, \ldots, P_c, k)$ (where $D = (V, E)$ is a directed graph) as follows:

Order the edges of $G$ arbitrarily by $e_1, \ldots, e_m$, and for each edge $e_i$, add a vertex $u_i$ to $D$.
For each vertex $v_i$ in $G$, add two more vertices to $D$: $w_i$ and $x_i$.
Draw a path from $w_i$ to $x_i$, using (in ascending order) the vertices in $D$ corresponding to the edges adjacent to $v_i$ in $G$.
Let $P_1, \ldots, P_n$ be the paths we constructed between each $w_i$ and $x_i$.

We claim that $G$ has an independent set of size $k$ if and only if $D$ has $k$ nonintersecting paths.
We prove our claim in two parts:
\newline

\noindent
$(\Longrightarrow)$:
Let $S$ be an independent set of size $k$ in $G$.
Let $v_1, \ldots, v_k \in S$.
None of $v_1, \ldots, v_k$ share any edges, so none of the $k$ paths $w_1-x_1, \ldots, w_k-x_k$ in $D$ share any vertices.
Thus they form a set of $k$ nonintersecting paths in $D$.
\newline

\noindent
$(\Longleftarrow)$:
Let $P_1, \ldots, P_k$ be a set of $k$ nonintersecting paths in $D$.
We claim $S = \{v_1, \ldots, v_k\}$ is an independent set in $G$.
Assume for a contradiction that this is not the case.
Then there exist vertices $v_i, v_j \in S$ that share an edge.
Let $e = (v_i, v_j)$, and let $u$ be the vertex in $D$ corresponding to $e$.
By the construction of $D$, $u \in P_i$ and $u \in P_j$, a contradiction, since all $k$ paths are nonintersecting.
So indeed $S$ is an independent set in $G$.


\section{} % Section 5

\textbf{(1)}
Assume not.
Then $u \not\in C$, so each of $u$'s $>k$ neighbours must be included in $C$ to cover all the edges incident to $u$.
This is a contradiction, since $\abs{C} \le k$, and so it must be the case that $u \in C$.
\newline

\noindent
\textbf{(2)}
A set of vertices can cover no more than $k$ edges per vertex in the set, so any set of $k'$ vertices in $G$ covers at most $kk'$ edges.
Thus if $G$ has more than $kk'$ edges, a vertex cover cannot have fewer than $k'+1$ vertices.
\newline

\noindent
We use the above two facts in the design of an algorithm for the fixed parameter variant of \textproc{Vertex Cover}.
The main idea is this:
If $m$ is the number of edges in $G$, and $k'$ is the largest degree, then if $\frac{m}{k'} > k$, then there exists no vertex cover of size $k$ or smaller.
Otherwise, we can remove a maximum-degree vertex from $G$ and check the above criteria again.
We keep repeating this process until at most a constant number of vertices remain, at which point we can simply use a brute force search of the remaining graph to check if a vertex cover of the desired exists in constant time.
A more concrete algorithm is given below:
\newline
\newline
Note: For brevity, we use the notation $\Delta(G)$ to denote the maximum vertex degree in $G$.

\begin{algorithm}
\begin{algorithmic}


\Function{Vertex-Cover-FPT}{$G, k$}
\State $C \gets \emptyset$
\While{$\Delta(G-C) > k - \abs{C}$}
\State $v \gets$ vertex with maximum degree
\State $C \gets C \cup \{v\}$
\EndWhile

\If{$\left(k - \abs{C}\right)^2 < m$}
	\State \Return false
\Else
	\State check all subsets of $G-C$
	\If{$G-C$ has a vertex cover of size at most $k - \abs{C}$}
		\State \Return true
	\Else
		\State \Return false
	\EndIf
\EndIf
\EndFunction
\end{algorithmic}
\end{algorithm}

If we're able to find a vertex cover of $G-C$ with size $k - \abs{C}$, then we can add its vertices to $C$ to get a vertex cover for all of $G$ with size at most $k$, so our algorithm is correct.

In the while loop, we look at any edge or vertex $O(1)$ times, so the entire loop runs in $O(n + m)$.
After the loop, $G-C$ has fewer than $(k - \abs{C})^2$ edges left, so no more than $2(k - \abs{C})^2$ vertices remain, and checking all subsets takes $O(2^{2(k - \abs{C})^2}) = O(2^{2k^2})$.
Thus our algorithm runs in time $O(n + m + 2^{2k^2})$, as desired.


\end{document}
