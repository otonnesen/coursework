\documentclass[11pt]{article}

\usepackage[margin=1.5in]{geometry}

\usepackage{fancyhdr}
\pagestyle{fancy}
\newcommand\course{CSC 425}
\newcommand\duedate{October 22, 2020}

\lhead{Oliver Tonnesen\\V00885732}
\chead{\textbf{\Large Final}}
\rhead{\course\\\duedate}


\usepackage{tikz}


\usepackage{amsmath, amssymb, mathtools}

\DeclarePairedDelimiter\abs{\lvert}{\rvert}%
\makeatletter
\let\oldabs\abs
\def\abs{\@ifstar{\oldabs}{\oldabs*}}

\usepackage{algorithm}
\usepackage[noend]{algpseudocode}

\begin{document}
\renewcommand{\thesubsection}{\thesection.\alph{subsection}}
\section{} % Section 1
Suppose not.
Then there is some pair $m_1 - w_1$ of a good man and a bad woman, and since there are an equal number of good men and women, there must also be a pair $m_2 - w_2$ of a bad man and a good woman.

Since $m_1$ is good and $m_2$ is bad, $m_1$ is ranked higher on $w_2$'s preference list than $m_2$.
Similarly, since $w_1$ is bad and $w_2$ is good, $w_2$ is ranked higher on $m_1$'s preference list than $w_1$.
Thus the pair $m_1 - w_2$ is unstable, a contradiction to our assumption that the matching is stable, and so it must be the case that each good man is married to a good woman, as desired.


\section{} % Section 2
\subsection{} % Section 2.a
Consider a situation where one very long job starts before any other job and ends after every other job.
The algorithm will chooes this job


\subsection{} % Section 2.b


\section{} % Section 3


\section{} % Section 4
Below is our table after we've filled it out:
\begin{center}
\renewcommand{\arraystretch}{1.6}
\begin{tabular}{| c |c |c |c |c |c |c | c |}
	\hline
	  &   & P & A & L & A & T & E \\ \hline
	  & 0 & 2 & 4 & 6 & 8 & 10 & 12 \\ \hline
	P & 2 & 0 & 2 & 4 & 6 & 8 & 10 \\ \hline
	A & 4 & 2 & 0 & 2 & 4 & 6 & 8 \\ \hline
	L & 6 & 4 & 2 & 0 & 2 & 4 & 6 \\ \hline
	E & 8 & 6 & 4 & 2 & 1 & 3 & 5 \\ \hline
	T & 10 & 8 & 6 & 4 & 3 & 1 & 3 \\ \hline
	T & 12 & 10 & 8 & 6 & 5 & 3 & 2 \\ \hline
	E & 14 & 12 & 10 & 8 & 7 & 5 & 3 \\ \hline
\end{tabular}
\end{center}
Our final answer is 3, and it is obtained with the following alignment:
\newline
P A L A T -- E
\newline
P A L E T T E


\end{document}
