\documentclass{article}
\usepackage{amsmath,amsfonts,tikz,cancel}
\usepackage[margin=1in]{geometry}
\title{MATH 322 Assignment 3}
\author{Oliver Tonnesen\\V00885732}
\date{March 11, 2019}
\begin{document}
\maketitle
\renewcommand{\thesubsection}{\thesection.\roman{subsection}}
\section{} % Section 1
\subsection{} % Section 1.i
\begin{align*}
	b&=\frac{\lambda n(n-1)}{k(k-1)}\\
	&=\frac{3\cdot25(24}{10(9)}\\
	&=20\\\\
	r&=\frac{\lambda(n-1)}{k-1}\\
	&=\frac{3\cdot24}{9}\\
	&=8
\end{align*}
By Fisher's Inequality, we see that this design cannot exist, since $b<n$.
\subsection{} % Section 1.ii
\begin{align*}
	&n(n-1)=\frac{bk(k-1)}{\lambda}\\
	&\Rightarrow n^2-n-\frac{bk(k-1)}{\lambda}=0\\
	&\Rightarrow n^2-n-90=0\\
	&\Rightarrow n = \frac{1\pm\sqrt{361}}{2}\\
	&= \cancel{-9}, 10\\
	&= 10\\
\end{align*}
\begin{align*}
	r&=\frac{\lambda(n-1)}{k-1}\\
	&=\frac{2(9)}{3}\\
	&=6
\end{align*}
\subsection{} % Section 1.iii
\begin{align*}
	n&=\frac{r(k-1)}{\lambda}+1\\
	&=\frac{13\cdot5}{1}+1\\
	&=66\\\\
	b&=\frac{\lambda n(n-1)}{k(k-1)}\\
	&=\frac{1\cdot66(65)}{6(5)}=143
\end{align*}
\section{} % Section 2
Our $n$ parameter is clearly 16.
\newline
For the $k$ parameter, notice that each block
$B_i$ consists of each element in the same row as $i$ as well as each element
in the same column as $i$, but not $i$. Each row and column has 4 elements, so
each block thus contains 6 elements, and $k=6$.
\newline
For the $\lambda$ parameter, consider any pair of elements, say $A_{ij}$ and
$A_{kl}$. It's clear to see that the only two elements whose rows or columns
contain both $A_{ij}$ and $A_{kl}$ are $A_{il}$ and $A_{kj}$. Any other does
not contain one or both of $A_{ij}$ and $A_{kl}$ in its block, so clearly each
pair occurs in exactly two blocks (specifically, $B_{A_{il}}$ and $B_{A_{kj}}$).
\section{} % Section 3
Take an arbitrary pair of points, $x_1$ and $x_2$. $x_1$ and $x_2$ are contained
in $\lambda$ blocks, and $x_1$ or $x_2$ (inclusive) are contained in $2r$ blocks.
Thus, the number of blocks containing either $x_1$ or $x_2$ (exclusive) is
$2r-\lambda$. Let $y$ be the number of blocks containing neither $x_1$ nor $x_2$.
Then $b=y+2r-\lambda\Rightarrow y=b-2r+\lambda$. Thus if $y$ is the number of
blocks in $\mathcal{B}$ not containing $x_1$ or $x_2$, then $y$ is the number
of blocks containing both $x_1$ and $x_2$ in $\mathcal{B}'$. Thus
$(X,\mathcal{B}')$ is an $(n,n-k,b-2r+\lambda)$-design.
\section{} % Section 4
\subsection{} % Section 4.i
\begin{align*}
&1-2=14\qquad&1-3=13\qquad&1-5=11\qquad&1-6=10\qquad&1-9=7\qquad\\
&1-11=5\qquad&2-1=1\qquad&2-3=14\qquad&2-5=12\qquad&2-6=11\qquad\\
&2-9=8\qquad&2-11=6\qquad&3-1=2\qquad&3-2=1\qquad&3-5=13\qquad\\
&3-6=12\qquad&3-9=9\qquad&3-11=7\qquad&5-1=4\qquad&5-2=3\qquad\\
&5-3=2\qquad&5-6=14\qquad&5-9=11\qquad&5-11=9\qquad&6-1=5\qquad\\
&6-2=4\qquad&6-3=3\qquad&6-5=1\qquad&6-9=12\qquad&6-11=10\qquad\\
&9-1=8\qquad&9-2=7\qquad&9-3=6\qquad&9-5=4\qquad&9-6=3\qquad\\
&9-11=13\qquad&11-1=10\qquad&11-2=9\qquad&11-3=8\qquad&11-5=6\qquad\\
&11-6=5\qquad&11-9=2
\end{align*}
We can clearly see that this is indeed a $(15,7,3)$-difference set.
\subsection{} % Section 4.ii
$D=\{1,4,5,6,7,9,11,16,17\}$
\newline
$\mathbb{Z}_{19}\setminus D=\{0,2,3,8,10,12,13,14,15,18\}$
\begin{align*}
&0-2=17\qquad&0-3=16\qquad&0-8=11\qquad&0-10=9\qquad&0-12=7\qquad\\
&0-13=6\qquad&0-14=5\qquad&0-15=4\qquad&0-18=1\qquad&2-0=2\qquad\\
&2-3=18\qquad&2-8=13\qquad&2-10=11\qquad&2-12=9\qquad&2-13=8\qquad\\
&2-14=7\qquad&2-15=6\qquad&2-18=3\qquad&3-0=3\qquad&3-2=1\qquad\\
&3-8=14\qquad&3-10=12\qquad&3-12=10\qquad&3-13=9\qquad&3-14=8\qquad\\
&3-15=7\qquad&3-18=4\qquad&8-0=8\qquad&8-2=6\qquad&8-3=5\qquad\\
&8-10=17\qquad&8-12=15\qquad&8-13=14\qquad&8-14=13\qquad&8-15=12\qquad\\
&8-18=9\qquad&10-0=10\qquad&10-2=8\qquad&10-3=7\qquad&10-8=2\qquad\\
&10-12=17\qquad&10-13=16\qquad&10-14=15\qquad&10-15=14\qquad&10-18=11\qquad\\
&12-0=12\qquad&12-2=10\qquad&12-3=9\qquad&12-8=4\qquad&12-10=2\qquad\\
&12-13=18\qquad&12-14=17\qquad&12-15=16\qquad&12-18=13\qquad&13-0=13\qquad\\
&13-2=11\qquad&13-3=10\qquad&13-8=5\qquad&13-10=3\qquad&13-12=1\qquad\\
&13-14=18\qquad&13-15=17\qquad&13-18=14\qquad&14-0=14\qquad&14-2=12\qquad\\
&14-3=11\qquad&14-8=6\qquad&14-10=4\qquad&14-12=2\qquad&14-13=1\qquad\\
&14-15=18\qquad&14-18=15\qquad&15-0=15\qquad&15-2=13\qquad&15-3=12\qquad\\
&15-8=7\qquad&15-10=5\qquad&15-12=3\qquad&15-13=2\qquad&15-14=1\qquad\\
&15-18=16\qquad&18-0=18\qquad&18-2=16\qquad&18-3=15\qquad&18-8=10\qquad\\
&18-10=8\qquad&18-12=6\qquad&18-13=5\qquad&18-14=4\qquad&18-15=3\qquad\\
\end{align*}
Thus $\mathbb{Z}_{19}\setminus D$ is a $(19,10,5)$-difference set.
\section{} % Section 5
Fix $x\in\mathbb{Z}_n$. $x$ can be written as the difference of two elements in
$\mathbb{Z}_n$ in exactly $n$ ways, as the difference of two elements in $D$ in
$\lambda$ ways, and as the difference of two elements in $\overline{D}$ in
$n-2k+\lambda$ ways. Thus, we have that $x$ can be written as the difference of
two elements, one in $D$ and the other in $\overline{D}$ in
\[n-\lambda-(n-2k+\lambda)=2k-2\lambda=2(k-\lambda)\]
ways.
\section{} % Section 6
\subsection{} % Section 6.i
$q=7$ is the only multiplier satisfying the Multiplier Theorem. Let $D$ be a
difference set fixed by $q$, i.e. $7D=D$. $D$ must be some union of the
following sets:
\begin{align*}
&\{0\}\\
&\{1,7,9,10,12,16,26,33,34\}\\
&\{2,14,15,18,20,24,29,31,32\}\\
&\{3,4,11,21,25,27,28,30,36\}\\
&\{5,6,8,13,17,19,22,23,35\}\\
\end{align*}
We see that $\{1,7,9,10,12,16,26,33,34\}$ is a $(37,9,2)$-difference set.
\subsection{} % Section 6.ii
$q=3$ and $q=9$ are the possible multipliers satisfying the Multiplier Theorem.
Let $D$ be a difference set fixed by $q$, i.e. $3D=D$ or $9D=D$. $D$ must be
some union of the either these sets ($q=3$):
\begin{align*}
&\{0\}\qquad&\{1,3,9,27,25,19\}\qquad&\{2,6,18,54,50,38\}\qquad\\
&\{4,12,36,52,44,20\}\qquad&\{5,15,45,23,13,39\}\qquad&\{7,21\}\qquad\\
&\{8,24,16,48,32,40\}\qquad&\{10,30,34,46,26,22\}\qquad&\{11,33,43,17,51,41\}\qquad\\
&\{14,42\}\qquad&\{28\}\qquad&\{29,31,37,55,53,47\}\qquad\\
\end{align*}
or of these sets ($q=9$):
\begin{align*}
&\{0\}\qquad&\{1,9,25\}\qquad&\{2,18,50\}\qquad&\{3,27,19\}\qquad\\
&\{4,36,44\}\qquad&\{5,45,13\}\qquad&\{6,54,38\}\qquad&\{7\}\qquad\\
&\{8,16,32\}\qquad&\{10,34,26\}\qquad&\{11,43,51\}\qquad&\{12,52,20\}\qquad\\
&\{14\}\qquad&\{15,23,39\}\qquad&\{17,41,33\}\qquad&\{21\}\qquad\\
&\{22,30,46\}\qquad&\{24,48,40\}\qquad&\{28\}\qquad&\{29,37,53\}\qquad\\
&\{31,55,47\}\qquad&\{35\}\qquad&\{42\}\qquad&\{49\}\qquad\\
\end{align*}
We see that no such union gives a difference set, and so no difference set exists.
\end{document}
