\documentclass{article}
\usepackage{amsmath,amssymb,amsfonts,tikz,cancel}
\usepackage[margin=1in]{geometry}
\title{MATH 322 Assignment 4}
\author{Oliver Tonnesen\\V00885732}
\date{March 21, 2019}
\begin{document}
\maketitle
\renewcommand{\thesubsection}{\thesection.\alph{subsection}}
\section{} % Section 1
\subsection{} % Section 1.a
\underline{Reflexive}: By definition, $\preceq$ is reflexive.
\newline
\underline{Antisymmetric}: Suppose
\begin{enumerate}
	\item $x\preceq y$
	\item $y\preceq x$
\end{enumerate}
Then by 1., $x$ is an odd string, and by 2., $y$ is an odd string. We now now
that either $x=y$ or $x$ and $y$ are adjacent in $Q_n$. Both $x$ and $y$ are
odd, and can thus not be adjacent in $Q_n$, so $x=y$.
\newline
\underline{Transitive}: Suppose $x\preceq y$ and $y\preceq z$, and that
$x\neq y$. Then $x$ is an odd string, meaning $y$ must be an even string. The
only $b$ such that there exists an even string $a$ with $a\preceq b$ is $b=a$,
so $y\preceq z\implies y=z$. We now have that
$x\preceq y$ and $y\preceq z\iff x\preceq y$ and $y\preceq y$. Thus transitivity
holds when $x\neq y$, and we can see that this is trivially true when $x=y$.
\subsection{} % Section 1.b
Let $A=\{x\in X\mid\text{$x$ is an even string}\}$. Then $A$ is a maximum
antichain, since $A=\frac{|X|}{2}$ and $Q_n$ is bipartite.
\subsection{} % Section 1.c
Construct our chain cover as follows:
\begin{itemize}
	\item Take the lexicographically smallest unmarked element in $X$, and
		place it in a chain with the next element in lexicographic order.
	\item Mark both elements
	\item Repeat until no unmarked elements remain.
\end{itemize}
Each chain constructed in this fashion contains two elements, and so we end up
with $\frac{|X|}{2}$ chains in our covering -- the as the number of elements in
our maximum antichain.
\section{} % Section 2
\[\mathcal{R}\subseteq\mathcal{R}^*\iff\nexists x,y\in X,x\neq y\text{, s.t. }x\mathcal{T}y\text{ and }y\mathcal{T}x\]
in $\mathcal{T}$, $\mathcal{R}$'s transitive closure.
\newline
($\Longrightarrow$)
\newline
The proof of the forward direction is trivial.
\newline
($\Longleftarrow$)
\newline
We show the contrapositive is true:\\
$\mathcal{R}\nsubseteq\mathcal{R}^*\Longrightarrow\exists x,y\in X,x\neq y$, s.t.
$x\mathcal{T}y$ and $y\mathcal{T}x$ in $\mathcal{T}$, $\mathcal{R}$'s transitive
closure.
\newline
\newline
$\mathcal{R}$ is not the subset of a partial order, so it must be the case that
there are $x,y\in X, x\neq y$ such that $x\mathcal{R}y$ and $y\mathcal{R}x$.
These $x,y$ satisfy the above statement, and so our condition holds.
\newline
\newline
To obtain $\mathcal{R}^*$, add all elements of the form $x\mathcal{R}x$ and also
ensure that if $x\mathcal{R}y$ and $x\mathcal{R}y$, then $x\mathcal{R}z$ for all
$x,y,z\in X$.
\section{} % Section 3
Enumerate the subsets of size $\big\lfloor\frac{n}{2}\big\rfloor$. For each
subset, pair it with the subset obtained by adding the least element that does
not form a set we've already created. A partial example is given below with $n=5$:
\newline
\newline
For $\{1,2\}$, the least element we can add is $3$, so we pair $\{1,2\}$ with
$\{1,2,3\}$.\\
For $\{1,3\}$, the least element we can add is not $2$, since we've already
created the set $\{1,2,3\}$, so we instead add $4$, pairing $\{1,3\}$ with
$\{1,3,4\}$.
For $\{1,4\}$, the least element we can add is $2$, so we pair $\{1,4\}$ with
$\{1,2,4\}$.\\
For $\{1,5\}$, the least element we can add is $2$, so we pair $\{1,5\}$ with
$\{1,2,5\}$.\\
For $\{2,3\}$, the least element we can add is not $1$, since we've already
created the set $\{1,2,3\}$, so we instead add $4$, pairing $\{2,3\}$ with
$\{2,3,4\}$.\\
And continuing on following the same pattern.
\newline
\newline
This method works since if we fix any $x\in[n]$, there are exactly $n-1$
subsets of size $\big\lfloor\frac{n}{2}\big\rfloor$ containing it. Similarly,
there are $n-1$ other elements of $[n]$ to add to some subset, so for each
subset we consider we can create a supserset by adding a single element.
\section{} % Section 4
Assume for a contradiction that no such $S$ exists. We know that there exists a
subset $Y$ not in $\mathcal{F}$ whose complement is also not in $\mathcal{F}$.
In other words, both $\mathcal{F}\cup Y$ and $\mathcal{F}\cup([n]\setminus Y)$
are not intersecting families. This means that there are sets $U,V\in\mathcal{F}$
such that $U\cap Y=\emptyset$ and $V\cap([n]\setminus Y)=\emptyset$, a
contradiction since $U$ and $V$ must intersect by definition. So our assumption
was false and such an $S$ indeed exists for any $|\mathcal{F}|<2^{n-1}$.
\section{} % Section 5
Fix some $x\in[n]$, and choose every $k$ subset of $[n]$ containing $x$. After
fixing $x$, there remain $n-1$ elements from which to choose the $k-1$ elements
of each $k$-subset. Thus our intersecting family $\mathcal{F}$ contains exactly
$\binom{n-1}{k-1}$ subsets of $[n]$.
\end{document}
