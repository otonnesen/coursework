\documentclass{article}
\usepackage{amsmath,amsfonts,tikz}
\title{MATH 322 Assignment 2}
\author{Oliver Tonnesen\\V00885732}
\date{February 7, 2019}
\begin{document}
\maketitle
\renewcommand{\thesubsection}{\thesection.\alph{subsection}}
\section{} % Section 1
\subsection{} % Section 1.a
An idempotent symmetric Latin square of order 5:
\newline
\newline
\begin{tabular}{ccccc}
	1 & 3 & 5 & 2 & 4\\
	3 & 2 & 4 & 5 & 1\\
	5 & 4 & 3 & 1 & 2\\
	2 & 5 & 1 & 4 & 3\\
	4 & 1 & 2 & 3 & 5\\
\end{tabular}
\newline
\newline
Using the Bose Construction, we have
\begin{alignat*}{3}
	&S=\{&&x_i,y_i,z_i\;|\;1\le i\le5\}\\
	&T=\{&&\{x_1,y_1,z_1\},
	\{x_2,y_2,z_2\},
	\{x_3,y_3,z_3\},
	\{x_4,y_4,z_4\},
	\{x_5,y_5,z_5\},
	\{x_1,x_2,y_3\},\\
	&&&\{y_1,y_2,z_3\},
	\{z_1,z_2,x_3\},
	\{x_1,x_3,y_5\},
	\{y_1,y_3,z_5\},
	\{z_1,z_3,x_5\},
	\{x_1,x_4,y_2\},\\
	&&&\{y_1,y_4,z_2\},
	\{z_1,z_4,x_2\},
	\{x_1,x_5,y_4\},
	\{y_1,y_5,z_4\},
	\{z_1,z_5,x_4\},
	\{x_2,x_3,y_4\},\\
	&&&\{y_2,y_3,z_4\},
	\{z_2,z_3,x_4\},
	\{x_2,x_4,y_5\},
	\{y_2,y_4,z_5\},
	\{z_2,z_4,x_5\},
	\{x_2,x_5,y_1\},\\
	&&&\{y_2,y_5,z_1\},
	\{z_2,z_5,x_1\},
	\{x_3,x_4,y_1\},
	\{y_3,y_4,z_1\},
	\{z_3,z_4,x_1\},
	\{x_3,x_5,y_2\},\\
	&&&\{y_3,y_5,z_2\},
	\{z_3,z_5,x_2\},
	\{x_4,x_5,y_3\},
	\{y_4,y_5,z_3\},
	\{z_4,z_5,x_3\}
	\}
\end{alignat*}
\subsection{} % Section 1.b
A symmetric half-idempotent Latin square of order 6:
\newline
\newline
\begin{tabular}{cccccc}
	1 & 4 & 5 & 6 & 3 & 2\\
	4 & 2 & 6 & 3 & 5 & 1\\
	5 & 6 & 3 & 2 & 1 & 4\\
	6 & 3 & 2 & 1 & 4 & 5\\
	3 & 5 & 1 & 4 & 2 & 6\\
	2 & 1 & 4 & 5 & 6 & 3\\
\end{tabular}
\newline
\newline
Using the Skolem Construction, we have
\begin{alignat*}{3}
	&S=\{&&x_i,y_i,z_i\;|\;1\le i\le6\}\cup\{w\}\\
	&T=\{&&\{x_1,y_1,z_1\},
	\{x_2,y_2,z_2\},
	\{x_3,y_3,z_3\},
	\{x_1,z_4,w\},
	\{y_1,x_4,w\;\},
	\{z_1,y_4,w\;\},\\
	&&&\{x_2,z_5,w\},
	\{y_2,x_5,w\;\},
	\{z_2,y_5,w\;\},
	\{z_3,y_6,w\;\},
	\{z_3,y_6,w\;\},
	\{z_3,y_6,w\;\},\\
	&&&\{x_1,x_2,y_4\},
	\{y_1,y_2,z_4\},
	\{z_1,z_2,x_4\},
	\{x_1,x_3,y_5\},
	\{y_1,y_3,z_5\},
	\{z_1,z_3,x_5\},\\
	&&&\{x_1,x_4,y_6\},
	\{y_1,y_4,z_6\},
	\{z_1,z_4,x_6\},
	\{x_1,x_5,y_3\},
	\{y_1,y_5,z_3\},
	\{z_1,z_5,x_3\},\\
	&&&\{x_1,x_6,y_2\},
	\{y_1,y_6,z_2\},
	\{z_1,z_6,x_2\},
	\{x_2,x_3,y_6\},
	\{y_2,y_3,z_6\},
	\{z_2,z_3,x_6\},\\
	&&&\{x_2,x_4,y_3\},
	\{y_2,y_4,z_3\},
	\{z_2,z_4,x_3\},
	\{x_2,x_5,y_5\},
	\{y_2,y_5,z_5\},
	\{z_2,z_5,x_5\},\\
	&&&\{x_2,x_6,y_1\},
	\{y_2,y_6,z_1\},
	\{z_2,z_6,x_1\},
	\{x_3,x_4,y_2\},
	\{y_3,y_4,z_2\},
	\{z_3,z_4,x_2\},\\
	&&&\{x_3,x_5,y_1\},
	\{y_3,y_5,z_1\},
	\{z_3,z_5,x_1\},
	\{x_3,x_6,y_4\},
	\{y_3,y_6,z_4\},
	\{z_3,z_6,x_4\},\\
	&&&\{x_4,x_5,y_4\},
	\{y_4,y_5,z_4\},
	\{z_4,z_5,x_4\},
	\{x_4,x_6,y_5\},
	\{y_4,y_6,z_5\},
	\{z_4,z_6,x_5\},\\
	&&&\{x_5,x_6,y_6\},
	\{y_5,y_6,z_6\},
	\{z_5,z_6,x_6\}
	\}
\end{alignat*}
\section{} % Section 2
Due to the nature of sets (namely the fact that $\{i,j,k\}=\{j,i,k\}$),
for any $i,j\in[n],i\neq j$, $a_{ij}=a_{ji}$, satisfying the property that the
matrix must be a \textit{symmetric} Latin square. Since $(S,T)$ is a Steiner
triple system, we know that any $i,j,k\in[n]$ are contained in exactly one
triple together in $T$. In other words, if we fix $i$, we will find a different
$k$ for any choice of $j$, and if we fix $j$, we will find a different $k$ for
any choice of $i$. Thus for any row or column of $A$, no two numbers will occur
twice. Thus we have proven that the matrix $A$ as defined would be a symmetric
Latin square.
\section{} % Section 3
As we know, the number of triples containing any $x\in S$ in a Steiner triple
system is $\frac{|S|-1}{2}$, so in our \textit{2-fold triple system} we have
$|S|-1$ triples containing any $x\in S$. We also know that each triple contains
$\binom{3}{2}$ pairs, so in total, we have $\frac{|S|(|S|-1)}{\binom{3}{2}}=
\frac{|S|(|S|-1)}{3}$ triples. We know that $|S|(|S|-1)$ is even, since
$\frac{|S|(|S|-1)}{2}$ is an integer. Thus, since $\frac{|S|(|S|-1)}{3}$ is also
an integer, we know that it must be even. So we check to see for which numbers
this property holds:
\begin{align*}
	|S|&\equiv0\text{ (mod 3)}:\ \frac{3l(3l-1)}{3}=3l^2-l&\qquad\text{ Always even, so holds.}\\
	|S|&\equiv1\text{ (mod 3)}:\ \frac{(3l+1)(3l)}{3}=3l^2+l&\qquad\text{ Always even, so holds}\\
	|S|&\equiv2\text{ (mod 3)}:\ \frac{(3l+2)(3l+1)}{3}=\frac{9l^2+9l+3}{3}=3l^2+3l+1&\qquad\text{ Always odd, so does not hold}\\
\end{align*}
So we see that $|S|\equiv0\text{ or }1\text{ (mod 3)}$ is a necessary condition for the
existence of a \textit{2-fold triple system}.
\section{} % Section 4
The system has at a total of $\binom{10}{2}$ pairs. We can also count this number
as $3|T|$, since each triple counts three distinct pairs. Thus,
$\binom{10}{2}\le3|T|$. We also know that there exists no Steiner triple system
of order 10, so not all pairs can be counted in our packing. In other words, we
must miss at least one pair, so we update our inequality:
$\binom{10}{2}-1\le3|T|$. And so
$\Big\lfloor\frac{\binom{10}{2}-1}{3}\Big\rfloor\le|T|$. So $|T|$ can be at most
$\big\lfloor\frac{44}{3}\big\rfloor=14$.\\
Below is a maximal packing on 10 elements, with $|T|=13$.
\begin{align*}
	S=&\{1,2,3,4,5,6,7,8,9,10\}\\
	T=&\{\{2,3,4\},\{1,4,8\},\{6,7,8\},\{2,6,9\},\{3,7,10\},\\
	&\{3,5,9\},\{1,5,7\},\{1,2,10\},\{8,9,10\},\{1,3,6\},\\
	&\{4,7,9\},\{4,5,6\},\{2,5,8\}\}
\end{align*}
\section{} % Section 5
$L$ and $L'$ both have at least 3 points, and share exactly 1. Consider a point
on $L$ but not on $L'$; this point must be crossed by another line, $L''$. $L''$
has at least 2 points not on $L$, by $(P_2)$. Both of these points cannot be on
$L'$, by $(P_4)$. Thus, at least one of these must be on neither $L$ nor $L'$.
\section{} % Section 6
We aim to show that a projective plane $(\mathcal{P},\mathcal{L})$ satisfies
$(P_1),(P_2),(P_3),(P_4),(P_5)$ if and only if it satisfies $(Q),(P_4),(P_5)$.
We prove the two directions separately:
\newline
\newline
$(\implies)$: By $(P_1)$, there exists at least one line, $L$. By $(P_2)$, $L$ has at least
three points. By $(P_3)$, not all points are on $L$, so there exists
another line, $L'$, also having at least three points. By $(P_5)$, $L$ and $L'$
share exactly one point, so $L'$ has at least two points not on $L$. At this
point, we have a set of four points, two on $L$ and two on $L'$. By $(P_4)$, the
two points on $L$ cannot share another line, and similarly, the two points on
$L'$ cannot share another line, so no three of these fours points are colinear,
satisfying $(Q)$.
\newline
\newline
$(\impliedby)$: By $(Q)$, we have four points, $a,b,c,d$, no three of which
share a line. Now to show the implication holds, we must do so for any possible
combination of $a,b,c,\text{ and }d$, so we have three cases:\\
Case 1: We have two lines, each containing exactly two of $a,b,c,d$.\\
Case 2: We have three lines, one of which containing two of $a,b,c,d$, and the
other two containing one each of the two points not on the first line.\\
Case 3: We have four lines, each containing exactly one of $a,b,c,d$.\\
\newline
Case 1: Suppose WLOG that $a,b\in L_1$ and $c,d\in L_2$. By $(P_5)$, $L_1$ and
$L_2$ must intersect at a point. By $(Q)$, this point cannot be any of $a,b,c,d$,
so let $e\in L_1\cap L_2$. $L_1$ and $L_2$ now both have at least $3$ points,
and so all of $(P_1)$, $(P_2)$, and $(P_3)$ are satisfied.\\
\newline
Case 2: Suppose WLOG that $a,b\in L_1$, $c\in L_2$, and $d\in L_3$. By $(P_5)$,
$L_1$ intersects $L_2$ at some point not $c$ or $d$, so again we define
$e\in L_1\cap L_2$. Note that $|L_1|\ge3$. We know that for a projective plane
$(\mathcal{P},\mathcal{L})$, $|L_1|=|L_2|$ for all $L_1,L_2\in\mathcal{L}$, so
again, our conditions $(P_1)$, $(P_2)$, and $(P_3)$ are satisfied.\\
\newline
Case 3: We have four lines, each of which contains exactly one of $a,b,c,d$.
Let $a\in L_1$, $b\in L_2$, and $c\in L_3$. By $(P_4)$, $L_1$ and $L_2$
intersect, but not at any of $a,b,c,d$, by our assumption. Similarly, $L_1$ and
$L_3$ must also intersect, also at none of $a,b,c,d$. Note that $L_1$ cannot
intersect $L_3$ at the same point as that at which it intersects $L_2$, so
$|L_1|\ge3$. As before, we know that any two lines in a projective plane have
the same number of points, and so our conditions $(P_1)$, $(P_2)$, and $(P_3)$
are once again satisfied.\\
\newline
Thus we have shown that a projective plane $(\mathcal{P},\mathcal{L})$ satisfies
$(P_1),(P_2),(P_3),(P_4),(P_5)$ if and only if it satiesfies $(Q),(P_4),(P_5)$.
\end{document}
