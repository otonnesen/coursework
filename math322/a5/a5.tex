\documentclass{article}
\usepackage{amsmath,amssymb,amsfonts,tikz,cancel,mathtools}
\usepackage[margin=1in]{geometry}
\title{MATH 322 Assignment 5}
\author{Oliver Tonnesen\\V00885732}
\date{April 4, 2019}
\begin{document}
\maketitle
\renewcommand{\thesubsection}{\thesection.\alph{subsection}}
\section{} % Section 1
\subsection{} % Section 1.a
A generator matrix for $C$:
\begin{align*}
\begin{bmatrix}
	1 & 1 & 0 & 0 & 0\\
	0 & 1 & 1 & 1 & 1\\
	1 & 1 & 1 & 1 & 0\\
	0 & 1 & 0 & 1 & 0
\end{bmatrix}
&\longrightarrow
\begin{bmatrix}
	1 & 0 & 0 & 0 & 1\\
	0 & 1 & 0 & 0 & 1\\
	0 & 0 & 1 & 0 & 1\\
	0 & 0 & 0 & 1 & 1
\end{bmatrix}
\end{align*}
A parity check matrix for $C$:
\begin{align*}
	\begin{bmatrix}
		1 & 1 & 1 & 1 & 1
	\end{bmatrix}
\end{align*}
\subsection{} % Section 1.b
$dim(C)=4$, $dim(C^\perp)=1$.
\subsection{} % Section 1.c
$min dist(C)=2$, $min dist(C^\perp)=5$.

\section{} % Section 2
($\Longrightarrow$) $C$ has minimum distance $d$, so there exists a codeword
$w\in C$ with weight $d$. Since $H\cdot w^\mathsf{T}=0$, we can use $w$ to
construct a linear combination of $d$ columns in $H$ which sums to 0, thus
there exists a set of $d$ linearly dependent columns of $H$. Similarly, since
there does \underline{not} exist a codeword of weight $d-1$ in $C$, such a
linear combination cannot be constructed using only $d-1$ columns of $H$.
\newline
\newline
($\Longleftarrow$) There exists a set of $d$ linearly dependent columns of $H$.
Thus there exists a linear combination of these columns summing to 0. We can
simply construct a codeword of weight $d$ using this linear combination. Any
set of $d-1$ columns of $H$ are linearly dependent, so a codeword cannot have
weight less than $d$.
\section{} % Section 3
By definition, $C^\perp=\{v\in H_n\mid u\cdot v=0\;\forall u\in S\}$.
$C=C^\perp$, so we have that $w\cdot w=0$ for any $w\in C$. Thus $x$ must have
even weight, and so all codewords in $C$ must have even weight.

\section{} % Section 4
\subsection{} % Section 4.a
We row reduce $H_1$ and $H_2$, then, using their standard form (or equivalent
permuted standard form) we construct their corresponding generator matrices:
\begin{align*}
	H_1:
	&\begin{bmatrix}
		1 & 1 & 1 & 0 & 1 & 0 & 0\\
		1 & 1 & 0 & 1 & 0 & 1 & 0\\
		1 & 0 & 1 & 1 & 0 & 0 & 1
	\end{bmatrix}
	\longrightarrow
	\begin{bmatrix}
		1 & 0 & 0 & 0 & 1 & 1 & 1\\
		0 & 1 & 0 & 1 & 1 & 0 & 1\\
		0 & 0 & 1 & 1 & 1 & 1 & 0
	\end{bmatrix}\\
	\Longrightarrow G_1=
	&\begin{bmatrix}
		0 & 1 & 1 & 1 & 0 & 0 & 0\\
		1 & 1 & 1 & 0 & 1 & 0 & 0\\
		1 & 0 & 1 & 0 & 0 & 1 & 0\\
		1 & 1 & 0 & 0 & 0 & 0 & 1
	\end{bmatrix},\text{ the generator matrix for $C_1$}
\end{align*}
\newline
\begin{align*}
	H_2:
	&\begin{bmatrix}
		1 & 0 & 1 & 0 & 1 & 0 & 1\\
		0 & 1 & 1 & 0 & 0 & 1 & 1\\
		0 & 0 & 0 & 1 & 1 & 1 & 1
	\end{bmatrix}
	\qquad\stackrel{\mathclap{\normalfont\mbox{$\sigma=(34657)$}}}{\longrightarrow}\qquad
	\begin{bmatrix}
		1 & 0 & 0 & 0 & 1 & 1 & 1\\
		0 & 1 & 0 & 1 & 1 & 0 & 1\\
		0 & 0 & 1 & 1 & 1 & 1 & 0
	\end{bmatrix}\\
	\Longrightarrow G_2'=
	&\begin{bmatrix}
		0 & 1 & 1 & 1 & 0 & 0 & 0\\
		1 & 1 & 1 & 0 & 1 & 0 & 0\\
		1 & 0 & 1 & 0 & 0 & 1 & 0\\
		1 & 1 & 0 & 0 & 0 & 0 & 1
	\end{bmatrix}
\end{align*}
We see that $G_1=G_2'$, and so $C_1$ and $C_2$ are indeed equivalent.
\subsection{} % Section 4.b
\begin{align*}
	G_2'=
	\begin{bmatrix}
		0 & 1 & 1 & 1 & 0 & 0 & 0\\
		1 & 1 & 1 & 0 & 1 & 0 & 0\\
		1 & 0 & 1 & 0 & 0 & 1 & 0\\
		1 & 1 & 0 & 0 & 0 & 0 & 1
	\end{bmatrix}
	\qquad\;\stackrel{\mathclap{\normalfont\mbox{$\sigma^{-1}=(75643)$}}}{\longrightarrow}\qquad\;
	\begin{bmatrix}
		1 & 1 & 0 & 1 & 0 & 1 & 0\\
		1 & 0 & 0 & 1 & 0 & 0 & 1\\
		0 & 1 & 0 & 1 & 1 & 0 & 0\\
		1 & 1 & 1 & 0 & 0 & 0 & 0
	\end{bmatrix}
\end{align*}

\section{} % Section 5
We begin with the given parity check matrix and permute it to standard form:
\begin{align*}
	\begin{bmatrix}
		0 & 0 & 0 & 1 & 1 & 1 & 1\\
		0 & 1 & 1 & 0 & 0 & 1 & 1\\
		1 & 0 & 1 & 0 & 1 & 0 & 1
	\end{bmatrix}
	\qquad\stackrel{\mathclap{\normalfont\mbox{$\sigma=(143)$}}}{\longrightarrow}\qquad
	\begin{bmatrix}
		1 & 0 & 0 & 0 & 1 & 1 & 1\\
		0 & 1 & 0 & 1 & 0 & 1 & 1\\
		0 & 0 & 1 & 1 & 1 & 0 & 1
	\end{bmatrix}\\
\end{align*}
We then construct the corresponding generator matrix and apply the inverse
permutation to get the desired generator matrix:
\begin{align*}
	\begin{bmatrix}
		0 & 1 & 1 & 1 & 0 & 0 & 0\\
		1 & 0 & 1 & 0 & 1 & 0 & 0\\
		1 & 1 & 0 & 0 & 0 & 1 & 0\\
		1 & 1 & 1 & 0 & 0 & 0 & 1
	\end{bmatrix}
	\qquad\stackrel{\mathclap{\normalfont\mbox{$\sigma^{-1}=(341)$}}}{\longrightarrow}\qquad
	\begin{bmatrix}
		1 & 1 & 1 & 0 & 0 & 0 & 0\\
		1 & 0 & 0 & 1 & 1 & 0 & 0\\
		0 & 1 & 0 & 1 & 0 & 1 & 0\\
		1 & 1 & 0 & 1 & 0 & 0 & 1
	\end{bmatrix}
\end{align*}

\end{document}
